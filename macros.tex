%%%%%%%%%%%%%%%%%%%%%%%%%%%%%%%%%%%%%%%%%%%%%%%%%%%%%%%%%%%%%%%%%%%
% Michael's macros                                                %
% Version of 2021 July 25                                         %
%%%%%%%%%%%%%%%%%%%%%%%%%%%%%%%%%%%%%%%%%%%%%%%%%%%%%%%%%%%%%%%%%%%

%%%%%%%%%%%%%%%%%%%%%%%%%%%%%%%%%%%%%%%%%%%%%%%%%%%%%%%%%%%%%%%%%%%
%                          Theorems etc.                          %
%%%%%%%%%%%%%%%%%%%%%%%%%%%%%%%%%%%%%%%%%%%%%%%%%%%%%%%%%%%%%%%%%%%

\theoremstyle{plain}
% \newtheorem{theorem}{Theorem}%[section]
\newtheorem{proposition}[theorem]{Proposition}
\newtheorem{observation}[theorem]{Observation}
%\newtheorem{claim}[theorem]{Claim}
%\newtheorem{lemma}[theorem]{Lemma}
\newtheorem{corollary}[theorem]{Corollary}
\newtheorem{conjecture}[theorem]{Conjecture}
\newtheorem{conjollary}[theorem]{Conjollary}

\theoremstyle{definition}
\newtheorem{definition}[theorem]{Definition}
\newtheorem{example}[theorem]{Example}
\newtheorem{question}[theorem]{Question}
\newtheorem{problem}[theorem]{Problem}
\newtheorem{remark}[theorem]{Remark}

%%%%%%%%%%%%%%%%%%%%%%%%%%%%%%%%%%%%%%%%%%%%%%%%%%%%%%%%%%%%%%%%%%%
%                         Commands, etc.                          %
%%%%%%%%%%%%%%%%%%%%%%%%%%%%%%%%%%%%%%%%%%%%%%%%%%%%%%%%%%%%%%%%%%%

\newcommand{\st}{\::\:}
\newcommand{\Age}{   \operatorname{Age} }
\newcommand{\Av}{    \operatorname{Av}  }
% \renewcommand{\c}{   \operatorname{C}   } % Don't do this! Need it for that fancy French ç.
\newcommand{\conj}{  \operatorname{conj}}
\newcommand{\cont}{  \operatorname{c}   }
\newcommand{\Cont}{  \operatorname{C}   }
\newcommand{\g}{     \operatorname{G}   }
\newcommand{\Geom}{  \operatorname{Geom}}
\newcommand{\gr}{    \operatorname{gr}  }
\newcommand{\Grid}[1]{\operatorname{Grid}\left(#1\right)}
\renewcommand{\hom}{ \operatorname{hom} }
\newcommand{\height}{\operatorname{height}}
% \newcommand{\length}{\operatorname{length}}
\newcommand{\len}{   \operatorname{len} }
\newcommand{\mmd}{   \operatorname{mmd} }
\newcommand{\rev}{   \operatorname{rev} }
\newcommand{\comp}{  \operatorname{comp}}
\renewcommand{\u}{   \operatorname{u}   }
\newcommand{\w}{     \operatorname{w}   }

\newcommand{\worda}{\textsf{a}}
\newcommand{\wordb}{\textsf{b}}

\newcommand{\rir}{\text{\textsc{rir}}}

\newcommand{\A}{  \mathcal{A}}
\newcommand{\C}{  \mathcal{C}}
\newcommand{\F}{  \mathcal{F}}
\newcommand{\G}{  \mathcal{G}}
\renewcommand{\P}{\mathcal{P}}
\newcommand{\Q}{  \mathcal{Q}}
\newcommand{\Lay}{\mathcal{L}}
\newcommand{\R}{  \mathcal{R}}
\renewcommand{\S}{\mathcal{S}}
\newcommand{\SF}{\mathcal{SF}}
\newcommand{\SR}{\mathcal{SR}}
\newcommand{\T}{  \mathcal{T}}
\newcommand{\U}{  \mathcal{U}}
\newcommand{\W}{  \mathcal{W}}


\newcommand{\V}{  \textsf{V}  }
\newcommand{\X}{  \textsf{X}  }
\newcommand{\Sep}{\textsf{Sep}}

\newcommand{\sumind}{\notdirectsum}
\newcommand{\skewind}{\notskewsum}

\newcommand{\ulemph}[1]{\underline{#1}}
\renewcommand{\emph}[1]{\ulemph{#1}}

\newcommand{\union}{+}
\newcommand{\join}{\times}
\newcommand{\conn}{\notdirectsum}

\newcommand{\size}[1]{\left\lvert{#1}\right\rvert}
\newcommand{\magnitude}[1]{\left\lvert\left\lvert{#1}\right\rvert\right\rvert}
\newcommand{\floor}[1]{{\left\lfloor{#1}\right\rfloor}}
\newcommand{\ceil}[1]{{\left\lceil{#1}\right\rceil}}

\newcommand{\oTheta}[1]{\Theta{\left(#1\right)}}
\newcommand{\oOmega}[1]{\Omega{\left(#1\right)}}
\newcommand{\oO}[1]{O{\left(#1\right)}}
\newcommand{\oo}[1]{o{\left(#1\right)}}

\newcommand{\rotle}{\rotatebox[origin=c]{-90}{$\leq$}}

\newcommand{\maybeeq}{\mathrel{\rlap{\. ?}{=}}}

% The OEIS links:
\newcommand{\OEISlink}[1]{\href{https://oeis.org/#1}{#1}}
%\newcommand{\OEISref}{\href{http://www.research.att.com/\~njas/sequences/}{OEIS}~\cite{sloane:the-on-line-enc:}}
%\newcommand{\OEIS}[1]{(Sequence \OEISlink{#1} in the \OEISref.)}

%
\newcommand{\first}{^{\mbox{\scriptsize st}}}
\newcommand{\nd}{^{\mbox{\scriptsize nd}}}
\renewcommand{\th}{^{\mbox{\scriptsize th}}}

%%%%%%%%%%%%%%%%%%%%%%%%%%%%%%%%%%%%%%%%%%%%%%%%%%%%%%%%%%%%%%%%%%%
%                           Bookkeeping                           %
%%%%%%%%%%%%%%%%%%%%%%%%%%%%%%%%%%%%%%%%%%%%%%%%%%%%%%%%%%%%%%%%%%%

\newcounter{todocounter}
\newcommand{\todo}[1]{
	\addtocounter{todocounter}{1}
	\bigskip
	\noindent{\bf $\ll$ To-do \#\arabic{todocounter}:\rule{10pt}{0pt}#1 $\gg$}\bigskip
}

%%%%%%%%%%%%%%%%%%%%%%%%%%%%%%%%%%%%%%%%%%%%%%%%%%%%%%%%%%%%%%%%%%%
%                             Symbols                             %
%%%%%%%%%%%%%%%%%%%%%%%%%%%%%%%%%%%%%%%%%%%%%%%%%%%%%%%%%%%%%%%%%%%

\newcommand{\directsum}{\oplus}
\newcommand{\skewsum}{\ominus}
\newcommand{\bigdirectsum}{\bigoplus}
% \newcommand{\bigskewsum}{\bigominus}
\newcommand{\bigskewsum}{\ominus}

\usepackage{cancel}
\newcommand{\notdirectsum}{\cancel{\oplus}}
\newcommand{\notskewsum}{\cancel{\ominus}}
\newcommand{\bignotdirectsum}{\cancel{\bigoplus}}
% \newcommand{\bignotskewsum}{\cancel{\bigominus}}
\newcommand{\bignotskewsum}{\cancel{\ominus}}

%%%%%%%%%%%%%%%%%%%%%%%%%%%%%%%%%%%%%%%%%%%%%%%%%%%%%%%%%%%%%%%%%%%
%                          Custom grids                           %
%%%%%%%%%%%%%%%%%%%%%%%%%%%%%%%%%%%%%%%%%%%%%%%%%%%%%%%%%%%%%%%%%%%

\newcommand{\CustomGridOne}[1][0.5]{
	\begin{tikzpicture}[scale=1]
		\pgftransformxscale{#1};
		\pgftransformyscale{#1};

		\draw [semithick, line cap = round] (0,0) rectangle (2,2);
		\draw [semithick, line cap = round] (0,1) -- ++(2,0);
		\draw [semithick, line cap = round] (1,0) -- ++(0,2);

		\draw [semithick, line cap = round] (0,0) -- (1,1);
		\draw [semithick, line cap = round] (1,1) -- (2,2);

		\draw [draw=black, fill=black] (0,1) circle (4pt);

	\end{tikzpicture}
}

\newcommand{\CustomGridTwo}[1][0.25]{
	\begin{tikzpicture}[scale=1, baseline = (current bounding box.center)]
		\pgftransformxscale{#1};
		\pgftransformyscale{#1};

		\draw [semithick, line cap = round] (0,0) rectangle (5,5);

		\draw [semithick, line cap = round] (0,1) -- ++(5,0);
		\draw [semithick, line cap = round] (0,2) -- ++(5,0);
		\draw [semithick, line cap = round] (0,3) -- ++(5,0);
		\draw [semithick, line cap = round] (0,4) -- ++(5,0);

		\draw [semithick, line cap = round] (1,0) -- ++(0,5);
		\draw [semithick, line cap = round] (2,0) -- ++(0,5);
		\draw [semithick, line cap = round] (3,0) -- ++(0,5);
		\draw [semithick, line cap = round] (4,0) -- ++(0,5);

		\draw [thick, line cap = round] (0,0) -- (1,1);
		\draw [thick, line cap = round] (1,3) -- ++(2,2);
		\draw [thick, line cap = round] (3,1) -- ++(2,2);

		\draw [draw=black, fill=black] (4,5) circle (5pt);
		\draw [draw=black, fill=black] (5,4) circle (5pt);

	\end{tikzpicture}
}

\newcommand{\CustomGridThree}[1][0.25]{
	\begin{tikzpicture}[scale=1, baseline = (current bounding box.center)]
		\pgftransformxscale{#1};
		\pgftransformyscale{#1};

		\draw [semithick, line cap = round] (0,0) rectangle ++(6,6);

		\foreach \x in {0.25, 0.50, 0.75} {
			\draw [draw=black, fill = black] ({0+\x},{0+\x}) circle (3pt);
			\draw [draw=black, fill = black] ({1+\x},{6-\x}) circle (3pt);
		}

		\draw [thick, line cap = round]     (2,1) --        ++(1,1);
		\draw [semithick, line cap = round] (2,1) rectangle ++(1,1);
		\draw [draw=black, fill=black] (2,2) circle (5pt);

		\draw [thick, line cap = round]     (3,5) --        ++(1,-1);
		\draw [semithick, line cap = round] (3,5) rectangle ++(1,-1);

		\draw [thick, line cap = round]     (4,2) --        ++(1,1);
		\draw [semithick, line cap = round] (4,2) rectangle ++(1,1);
		\draw [draw=black, fill=black] (4,3) circle (5pt);		

		\draw [thick, line cap = round]     (5,4) --        ++(1,-1);
		\draw [semithick, line cap = round] (5,4) rectangle ++(1,-1);

	\end{tikzpicture}
}

\newcommand{\CustomGridFour}[1][0.25]{
	\begin{tikzpicture}[scale=1, baseline = (current bounding box.center)]
		\pgftransformxscale{#1};
		\pgftransformyscale{#1};

		\draw [thick, line cap = round]     (1,1) --        ++(1,1);
		\draw [semithick, line cap = round] (1,1) rectangle ++(1,1);

		% \draw [semithick, line cap = round] ({2+2/3},{1/3}) -- ++( 0,2);
		% \draw [semithick, line cap = round] ({2+2/3},{1/3}) -- ++(-2,0);
		\draw [draw=black, fill=black] ({2+2/3},{1/3}) circle (5pt);

		\draw [thick, line cap = round]     (3,2) --        ++(1,1);
		\draw [semithick, line cap = round] (3,2) rectangle ++(1,1);

		% \draw [semithick, line cap = round] ({2+1/3},{3+2/3}) -- ++( 0,-2);
		% \draw [semithick, line cap = round] ({2+1/3},{3+2/3}) -- ++( 2, 0);
		\draw [draw=black, fill=black] ({2+1/3},{3+2/3}) circle (5pt);

		% \draw [semithick, line cap = round] ({5+2/3},{3+1/3}) -- ++( 0,2);
		% \draw [semithick, line cap = round] ({5+2/3},{3+1/3}) -- ++(-2,0);
		\draw [draw=black, fill=black] ({5+2/3},{3+1/3}) circle (5pt);

		% \draw [semithick, line cap = round] ({-1+1/3},{2/3}) -- ++( 0,-2);
		% \draw [semithick, line cap = round] ({-1+1/3},{2/3}) -- ++( 2, 0);
		\draw [draw=black, fill=black] ({-1+1/3},{2/3}) circle (5pt);

		\foreach \x in {0.25, 0.50, 0.75} {
			\draw [draw=black, fill = black] ({4.5+\x},{4+\x}) circle (3pt);
			\draw [draw=black, fill = black] ({0.5-\x},{0-\x}) circle (3pt);
		}

	\end{tikzpicture}
}

\newcommand{\CustomGridFive}[1][0.25]{
	\begin{tikzpicture}[scale=1, baseline = (current bounding box.center)]
		\pgftransformxscale{#1};
		\pgftransformyscale{#1};

		\draw [semithick, line cap = round] (0,0) grid ++(4,4);

		\draw [thick, line cap = round] (0,2) -- ++(1,1);
		\draw [thick, line cap = round] (1,0) -- ++(1,1);
		\draw [thick, line cap = round] (2,1) -- ++(1,1);
		\draw [thick, line cap = round] (3,3) -- ++(1,1);

		\draw [draw=black, fill=black] (2,4) circle (5pt);

	\end{tikzpicture}
}

\newcommand{\CustomGridSix}[1][0.25]{
	\begin{tikzpicture}[scale=1, baseline = (current bounding box.center)]
		\pgftransformxscale{#1};
		\pgftransformyscale{#1};

		% \draw [semithick, line cap = round] (0,0) grid ++(4,4);

		\foreach \x in {0,1,2} {
			\draw [draw=black, fill=black] (\x,{\x+2}) circle (4pt);
			\draw [thick, line cap = round] (\x, \x) -- ++(1,1);
			\draw (\x, {\x+2}) -- ++(0,-2);
		}

		\draw (0,2) -- ++(2,0);
		\draw (1,3) -- ++(2,0);

		

		\foreach \x in {0.25, 0.50, 0.75} {
			\draw [draw=black, fill = black] ({3+\x},{3+\x}) circle (2.5pt);
		}

	\end{tikzpicture}
}

\newcommand{\CustomGridSeven}[1][0.25]{
	\begin{tikzpicture}[scale=1, baseline = (current bounding box.center)]
		\pgftransformxscale{#1};
		\pgftransformyscale{#1};

		\plotpermborder{2,1}

	\end{tikzpicture}
}

%%%%%%%%%%%%%%%%%%%%%%%%%%%%%%%%%%%%%%%%%%%%%%%%%%%%%%%%%%%%%%%%%%%
%                          Column types                           %
%%%%%%%%%%%%%%%%%%%%%%%%%%%%%%%%%%%%%%%%%%%%%%%%%%%%%%%%%%%%%%%%%%%

\renewcommand{\arraystretch}{1.5}
\newcolumntype{L}{>{$}l<{$}}
\newcolumntype{C}{>{$}c<{$}}
\newcolumntype{R}{>{$}r<{$}}
\newcolumntype{t}{>{$}r<{$\hspace{-9pt}}} % A tight R

%%%%%%%%%%%%%%%%%%%%%%%%%%%%%%%%%%%%%%%%%%%%%%%%%%%%%%%%%%%%%%%%%%%
%                      Define \endfirstfoot                       %
%%%%%%%%%%%%%%%%%%%%%%%%%%%%%%%%%%%%%%%%%%%%%%%%%%%%%%%%%%%%%%%%%%%
% See:
% https://tex.stackexchange.com/questions/68439/caption-at-foot-of-long-tables-longtable-package

\makeatletter
\newbox\LT@firstfoot
\def\endfirstfoot{\LT@end@hd@ft\LT@firstfoot}
\newdimen\LT@footdiff
\def\LT@start{%
  \let\LT@start\endgraf
  \endgraf\penalty\z@
  \vskip\LTpre\endgraf
  \LT@footdiff-\ht\LT@foot
  \advance\LT@footdiff\ht\LT@firstfoot
  \dimen@\pagetotal
  \advance\dimen@ \ht\ifvoid\LT@firsthead\LT@head\else\LT@firsthead\fi
  \advance\dimen@ \dp\ifvoid\LT@firsthead\LT@head\else\LT@firsthead\fi
  \advance\dimen@ \ht\ifvoid\LT@firstfoot\LT@foot\else\LT@firstfoot\fi
  \dimen@ii\vfuzz
  \vfuzz\maxdimen
  \setbox\tw@\copy\z@
  \setbox\tw@\vsplit\tw@ to \ht\@arstrutbox
  \setbox\tw@\vbox{\unvbox\tw@}%
  \vfuzz\dimen@ii
  \advance\dimen@ \ht
      \ifdim\ht\@arstrutbox>\ht\tw@\@arstrutbox\else\tw@\fi
  \advance\dimen@\dp
      \ifdim\dp\@arstrutbox>\dp\tw@\@arstrutbox\else\tw@\fi
  \advance\dimen@ -\pagegoal
  \ifdim \dimen@>\z@\vfil\break\fi
  \global\@colroom\@colht
  \ifvoid\LT@firstfoot
    \ifvoid\LT@foot
    \else
      \advance\vsize-\ht\LT@foot
      \global\advance\@colroom-\ht\LT@foot
      \dimen@\pagegoal\advance\dimen@-\ht\LT@foot\pagegoal\dimen@
      \maxdepth\z@
    \fi
  \else
    \advance\vsize-\ht\LT@firstfoot
    \global\advance\@colroom-\ht\LT@firstfoot
    \dimen@\pagegoal\advance\dimen@-\ht\LT@firstfoot\pagegoal\dimen@
    \maxdepth\z@
  \fi
  \ifvoid\LT@firsthead\copy\LT@head\else\box\LT@firsthead\fi\nobreak
  \output{\LT@output}%
}
\def\LT@output{%
  \ifnum\outputpenalty <-\@Mi
    \ifnum\outputpenalty > -\LT@end@pen
      \LT@err{floats and marginpars not allowed in a longtable}\@ehc
    \else
      \setbox\z@\vbox{\unvbox\@cclv}%
      \ifdim \ht\LT@lastfoot>\ht\LT@foot
        \dimen@\pagegoal
        \advance\dimen@-\ht\LT@lastfoot
        \ifdim\dimen@<\ht\z@
          \setbox\@cclv\vbox{\unvbox\z@\copy\LT@foot\vss}%
          \@makecol
          \@outputpage
          \setbox\z@\vbox{\box\LT@head}%
        \fi
      \fi  
      \global\@colroom\@colht
      \global\vsize\@colht   
      \vbox
        {\unvbox\z@\box\ifvoid\LT@lastfoot\LT@foot\else\LT@lastfoot\fi}%
    \fi
  \else
    \ifvoid\LT@firstfoot
      \setbox\@cclv\vbox{\unvbox\@cclv\copy\LT@foot\vss}%
      \@makecol
      \@outputpage
      \global\vsize\@colroom
    \else
      \setbox\@cclv\vbox{\unvbox\@cclv\box\LT@firstfoot\vss}%
      \@makecol
      \@outputpage
      \global\advance\@colroom\LT@footdiff
      \global\vsize\@colroom
    \fi
    \copy\LT@head\nobreak
  \fi
}
\makeatother