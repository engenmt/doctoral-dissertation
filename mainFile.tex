\documentclass{ufdissertation}\sloppy

%%%%%%%%%%%%%%%%%%%%%%%%%%%%%%%%%%%%%%%%%%%%%%%%%%%%%%%%%%%%%%%%%%%%%%%%%%%%%%%%
%%%                 User Package and Style File loading.
%%%%%%%%%%%%%%%%%%%%%%%%%%%%%%%%%%%%%%%%%%%%%%%%%%%%%%%%%%%%%%%%%%%%%%%%%%%%%%%%

%%%%%%%%%%%%%%%%%%%%%%%%%%%%%%%%%%%%%%%%%%%%%%%%%%%%%%%%%%%%%%%%%%%
% Michael's macros                                                %
% Version of 2021 July 25                                         %
%%%%%%%%%%%%%%%%%%%%%%%%%%%%%%%%%%%%%%%%%%%%%%%%%%%%%%%%%%%%%%%%%%%

%%%%%%%%%%%%%%%%%%%%%%%%%%%%%%%%%%%%%%%%%%%%%%%%%%%%%%%%%%%%%%%%%%%
%                          Theorems etc.                          %
%%%%%%%%%%%%%%%%%%%%%%%%%%%%%%%%%%%%%%%%%%%%%%%%%%%%%%%%%%%%%%%%%%%

\theoremstyle{plain}
% \newtheorem{theorem}{Theorem}%[section]
\newtheorem{proposition}[theorem]{Proposition}
\newtheorem{observation}[theorem]{Observation}
%\newtheorem{claim}[theorem]{Claim}
%\newtheorem{lemma}[theorem]{Lemma}
\newtheorem{corollary}[theorem]{Corollary}
\newtheorem{conjecture}[theorem]{Conjecture}
\newtheorem{conjollary}[theorem]{Conjollary}

\theoremstyle{definition}
\newtheorem{definition}[theorem]{Definition}
\newtheorem{example}[theorem]{Example}
\newtheorem{question}[theorem]{Question}
\newtheorem{problem}[theorem]{Problem}
\newtheorem{remark}[theorem]{Remark}

%%%%%%%%%%%%%%%%%%%%%%%%%%%%%%%%%%%%%%%%%%%%%%%%%%%%%%%%%%%%%%%%%%%
%                         Commands, etc.                          %
%%%%%%%%%%%%%%%%%%%%%%%%%%%%%%%%%%%%%%%%%%%%%%%%%%%%%%%%%%%%%%%%%%%

\newcommand{\st}{\::\:}
\newcommand{\Age}{   \operatorname{Age} }
\newcommand{\Av}{    \operatorname{Av}  }
% \renewcommand{\c}{   \operatorname{C}   } % Don't do this! Need it for that fancy French ç.
\newcommand{\conj}{  \operatorname{conj}}
\newcommand{\cont}{  \operatorname{c}   }
\newcommand{\Cont}{  \operatorname{C}   }
\newcommand{\g}{     \operatorname{G}   }
\newcommand{\Geom}{  \operatorname{Geom}}
\newcommand{\gr}{    \operatorname{gr}  }
\newcommand{\Grid}[1]{\operatorname{Grid}\left(#1\right)}
\renewcommand{\hom}{ \operatorname{hom} }
\newcommand{\height}{\operatorname{height}}
% \newcommand{\length}{\operatorname{length}}
\newcommand{\len}{   \operatorname{len} }
\newcommand{\mmd}{   \operatorname{mmd} }
\newcommand{\rev}{   \operatorname{rev} }
\newcommand{\comp}{  \operatorname{comp}}
\renewcommand{\u}{   \operatorname{u}   }
\newcommand{\w}{     \operatorname{w}   }

\newcommand{\worda}{\textsf{a}}
\newcommand{\wordb}{\textsf{b}}

\newcommand{\rir}{\text{\textsc{rir}}}

\newcommand{\A}{  \mathcal{A}}
\newcommand{\C}{  \mathcal{C}}
\newcommand{\F}{  \mathcal{F}}
\newcommand{\G}{  \mathcal{G}}
\renewcommand{\P}{\mathcal{P}}
\newcommand{\Q}{  \mathcal{Q}}
\newcommand{\Lay}{\mathcal{L}}
\newcommand{\R}{  \mathcal{R}}
\renewcommand{\S}{\mathcal{S}}
\newcommand{\SF}{\mathcal{SF}}
\newcommand{\SR}{\mathcal{SR}}
\newcommand{\T}{  \mathcal{T}}
\newcommand{\U}{  \mathcal{U}}
\newcommand{\W}{  \mathcal{W}}


\newcommand{\V}{  \textsf{V}  }
\newcommand{\X}{  \textsf{X}  }
\newcommand{\Sep}{\textsf{Sep}}

\newcommand{\sumind}{\notdirectsum}
\newcommand{\skewind}{\notskewsum}

\newcommand{\ulemph}[1]{\underline{#1}}
\renewcommand{\emph}[1]{\ulemph{#1}}

\newcommand{\union}{+}
\newcommand{\join}{\times}
\newcommand{\conn}{\notdirectsum}

\newcommand{\size}[1]{\left\lvert{#1}\right\rvert}
\newcommand{\magnitude}[1]{\left\lvert\left\lvert{#1}\right\rvert\right\rvert}
\newcommand{\floor}[1]{{\left\lfloor{#1}\right\rfloor}}
\newcommand{\ceil}[1]{{\left\lceil{#1}\right\rceil}}

\newcommand{\oTheta}[1]{\Theta{\left(#1\right)}}
\newcommand{\oOmega}[1]{\Omega{\left(#1\right)}}
\newcommand{\oO}[1]{O{\left(#1\right)}}
\newcommand{\oo}[1]{o{\left(#1\right)}}

\newcommand{\rotle}{\rotatebox[origin=c]{-90}{$\leq$}}

\newcommand{\maybeeq}{\mathrel{\rlap{\. ?}{=}}}

% The OEIS links:
\newcommand{\OEISlink}[1]{\href{https://oeis.org/#1}{#1}}
%\newcommand{\OEISref}{\href{http://www.research.att.com/\~njas/sequences/}{OEIS}~\cite{sloane:the-on-line-enc:}}
%\newcommand{\OEIS}[1]{(Sequence \OEISlink{#1} in the \OEISref.)}

%
\newcommand{\first}{^{\mbox{\scriptsize st}}}
\newcommand{\nd}{^{\mbox{\scriptsize nd}}}
\renewcommand{\th}{^{\mbox{\scriptsize th}}}

%%%%%%%%%%%%%%%%%%%%%%%%%%%%%%%%%%%%%%%%%%%%%%%%%%%%%%%%%%%%%%%%%%%
%                           Bookkeeping                           %
%%%%%%%%%%%%%%%%%%%%%%%%%%%%%%%%%%%%%%%%%%%%%%%%%%%%%%%%%%%%%%%%%%%

\newcounter{todocounter}
\newcommand{\todo}[1]{
	\addtocounter{todocounter}{1}
	\bigskip
	\noindent{\bf $\ll$ To-do \#\arabic{todocounter}:\rule{10pt}{0pt}#1 $\gg$}\bigskip
}

%%%%%%%%%%%%%%%%%%%%%%%%%%%%%%%%%%%%%%%%%%%%%%%%%%%%%%%%%%%%%%%%%%%
%                             Symbols                             %
%%%%%%%%%%%%%%%%%%%%%%%%%%%%%%%%%%%%%%%%%%%%%%%%%%%%%%%%%%%%%%%%%%%

\newcommand{\directsum}{\oplus}
\newcommand{\skewsum}{\ominus}
\newcommand{\bigdirectsum}{\bigoplus}
% \newcommand{\bigskewsum}{\bigominus}
\newcommand{\bigskewsum}{\ominus}

\usepackage{cancel}
\newcommand{\notdirectsum}{\cancel{\oplus}}
\newcommand{\notskewsum}{\cancel{\ominus}}
\newcommand{\bignotdirectsum}{\cancel{\bigoplus}}
% \newcommand{\bignotskewsum}{\cancel{\bigominus}}
\newcommand{\bignotskewsum}{\cancel{\ominus}}

%%%%%%%%%%%%%%%%%%%%%%%%%%%%%%%%%%%%%%%%%%%%%%%%%%%%%%%%%%%%%%%%%%%
%                          Custom grids                           %
%%%%%%%%%%%%%%%%%%%%%%%%%%%%%%%%%%%%%%%%%%%%%%%%%%%%%%%%%%%%%%%%%%%

\newcommand{\CustomGridOne}[1][0.5]{
	\begin{tikzpicture}[scale=1]
		\pgftransformxscale{#1};
		\pgftransformyscale{#1};

		\draw [semithick, line cap = round] (0,0) rectangle (2,2);
		\draw [semithick, line cap = round] (0,1) -- ++(2,0);
		\draw [semithick, line cap = round] (1,0) -- ++(0,2);

		\draw [semithick, line cap = round] (0,0) -- (1,1);
		\draw [semithick, line cap = round] (1,1) -- (2,2);

		\draw [draw=black, fill=black] (0,1) circle (4pt);

	\end{tikzpicture}
}

\newcommand{\CustomGridTwo}[1][0.25]{
	\begin{tikzpicture}[scale=1, baseline = (current bounding box.center)]
		\pgftransformxscale{#1};
		\pgftransformyscale{#1};

		\draw [semithick, line cap = round] (0,0) rectangle (5,5);

		\draw [semithick, line cap = round] (0,1) -- ++(5,0);
		\draw [semithick, line cap = round] (0,2) -- ++(5,0);
		\draw [semithick, line cap = round] (0,3) -- ++(5,0);
		\draw [semithick, line cap = round] (0,4) -- ++(5,0);

		\draw [semithick, line cap = round] (1,0) -- ++(0,5);
		\draw [semithick, line cap = round] (2,0) -- ++(0,5);
		\draw [semithick, line cap = round] (3,0) -- ++(0,5);
		\draw [semithick, line cap = round] (4,0) -- ++(0,5);

		\draw [thick, line cap = round] (0,0) -- (1,1);
		\draw [thick, line cap = round] (1,3) -- ++(2,2);
		\draw [thick, line cap = round] (3,1) -- ++(2,2);

		\draw [draw=black, fill=black] (4,5) circle (5pt);
		\draw [draw=black, fill=black] (5,4) circle (5pt);

	\end{tikzpicture}
}

\newcommand{\CustomGridThree}[1][0.25]{
	\begin{tikzpicture}[scale=1, baseline = (current bounding box.center)]
		\pgftransformxscale{#1};
		\pgftransformyscale{#1};

		\draw [semithick, line cap = round] (0,0) rectangle ++(6,6);

		\foreach \x in {0.25, 0.50, 0.75} {
			\draw [draw=black, fill = black] ({0+\x},{0+\x}) circle (3pt);
			\draw [draw=black, fill = black] ({1+\x},{6-\x}) circle (3pt);
		}

		\draw [thick, line cap = round]     (2,1) --        ++(1,1);
		\draw [semithick, line cap = round] (2,1) rectangle ++(1,1);
		\draw [draw=black, fill=black] (2,2) circle (5pt);

		\draw [thick, line cap = round]     (3,5) --        ++(1,-1);
		\draw [semithick, line cap = round] (3,5) rectangle ++(1,-1);

		\draw [thick, line cap = round]     (4,2) --        ++(1,1);
		\draw [semithick, line cap = round] (4,2) rectangle ++(1,1);
		\draw [draw=black, fill=black] (4,3) circle (5pt);		

		\draw [thick, line cap = round]     (5,4) --        ++(1,-1);
		\draw [semithick, line cap = round] (5,4) rectangle ++(1,-1);

	\end{tikzpicture}
}

\newcommand{\CustomGridFour}[1][0.25]{
	\begin{tikzpicture}[scale=1, baseline = (current bounding box.center)]
		\pgftransformxscale{#1};
		\pgftransformyscale{#1};

		\draw [thick, line cap = round]     (1,1) --        ++(1,1);
		\draw [semithick, line cap = round] (1,1) rectangle ++(1,1);

		% \draw [semithick, line cap = round] ({2+2/3},{1/3}) -- ++( 0,2);
		% \draw [semithick, line cap = round] ({2+2/3},{1/3}) -- ++(-2,0);
		\draw [draw=black, fill=black] ({2+2/3},{1/3}) circle (5pt);

		\draw [thick, line cap = round]     (3,2) --        ++(1,1);
		\draw [semithick, line cap = round] (3,2) rectangle ++(1,1);

		% \draw [semithick, line cap = round] ({2+1/3},{3+2/3}) -- ++( 0,-2);
		% \draw [semithick, line cap = round] ({2+1/3},{3+2/3}) -- ++( 2, 0);
		\draw [draw=black, fill=black] ({2+1/3},{3+2/3}) circle (5pt);

		% \draw [semithick, line cap = round] ({5+2/3},{3+1/3}) -- ++( 0,2);
		% \draw [semithick, line cap = round] ({5+2/3},{3+1/3}) -- ++(-2,0);
		\draw [draw=black, fill=black] ({5+2/3},{3+1/3}) circle (5pt);

		% \draw [semithick, line cap = round] ({-1+1/3},{2/3}) -- ++( 0,-2);
		% \draw [semithick, line cap = round] ({-1+1/3},{2/3}) -- ++( 2, 0);
		\draw [draw=black, fill=black] ({-1+1/3},{2/3}) circle (5pt);

		\foreach \x in {0.25, 0.50, 0.75} {
			\draw [draw=black, fill = black] ({4.5+\x},{4+\x}) circle (3pt);
			\draw [draw=black, fill = black] ({0.5-\x},{0-\x}) circle (3pt);
		}

	\end{tikzpicture}
}

\newcommand{\CustomGridFive}[1][0.25]{
	\begin{tikzpicture}[scale=1, baseline = (current bounding box.center)]
		\pgftransformxscale{#1};
		\pgftransformyscale{#1};

		\draw [semithick, line cap = round] (0,0) grid ++(4,4);

		\draw [thick, line cap = round] (0,2) -- ++(1,1);
		\draw [thick, line cap = round] (1,0) -- ++(1,1);
		\draw [thick, line cap = round] (2,1) -- ++(1,1);
		\draw [thick, line cap = round] (3,3) -- ++(1,1);

		\draw [draw=black, fill=black] (2,4) circle (5pt);

	\end{tikzpicture}
}

\newcommand{\CustomGridSix}[1][0.25]{
	\begin{tikzpicture}[scale=1, baseline = (current bounding box.center)]
		\pgftransformxscale{#1};
		\pgftransformyscale{#1};

		% \draw [semithick, line cap = round] (0,0) grid ++(4,4);

		\foreach \x in {0,1,2} {
			\draw [draw=black, fill=black] (\x,{\x+2}) circle (4pt);
			\draw [thick, line cap = round] (\x, \x) -- ++(1,1);
			\draw (\x, {\x+2}) -- ++(0,-2);
		}

		\draw (0,2) -- ++(2,0);
		\draw (1,3) -- ++(2,0);

		

		\foreach \x in {0.25, 0.50, 0.75} {
			\draw [draw=black, fill = black] ({3+\x},{3+\x}) circle (2.5pt);
		}

	\end{tikzpicture}
}

\newcommand{\CustomGridSeven}[1][0.25]{
	\begin{tikzpicture}[scale=1, baseline = (current bounding box.center)]
		\pgftransformxscale{#1};
		\pgftransformyscale{#1};

		\plotpermborder{2,1}

	\end{tikzpicture}
}

%%%%%%%%%%%%%%%%%%%%%%%%%%%%%%%%%%%%%%%%%%%%%%%%%%%%%%%%%%%%%%%%%%%
%                          Column types                           %
%%%%%%%%%%%%%%%%%%%%%%%%%%%%%%%%%%%%%%%%%%%%%%%%%%%%%%%%%%%%%%%%%%%

\renewcommand{\arraystretch}{1.5}
\newcolumntype{L}{>{$}l<{$}}
\newcolumntype{C}{>{$}c<{$}}
\newcolumntype{R}{>{$}r<{$}}
\newcolumntype{t}{>{$}r<{$\hspace{-9pt}}} % A tight R

%%%%%%%%%%%%%%%%%%%%%%%%%%%%%%%%%%%%%%%%%%%%%%%%%%%%%%%%%%%%%%%%%%%
%                      Define \endfirstfoot                       %
%%%%%%%%%%%%%%%%%%%%%%%%%%%%%%%%%%%%%%%%%%%%%%%%%%%%%%%%%%%%%%%%%%%
% See:
% https://tex.stackexchange.com/questions/68439/caption-at-foot-of-long-tables-longtable-package

\makeatletter
\newbox\LT@firstfoot
\def\endfirstfoot{\LT@end@hd@ft\LT@firstfoot}
\newdimen\LT@footdiff
\def\LT@start{%
  \let\LT@start\endgraf
  \endgraf\penalty\z@
  \vskip\LTpre\endgraf
  \LT@footdiff-\ht\LT@foot
  \advance\LT@footdiff\ht\LT@firstfoot
  \dimen@\pagetotal
  \advance\dimen@ \ht\ifvoid\LT@firsthead\LT@head\else\LT@firsthead\fi
  \advance\dimen@ \dp\ifvoid\LT@firsthead\LT@head\else\LT@firsthead\fi
  \advance\dimen@ \ht\ifvoid\LT@firstfoot\LT@foot\else\LT@firstfoot\fi
  \dimen@ii\vfuzz
  \vfuzz\maxdimen
  \setbox\tw@\copy\z@
  \setbox\tw@\vsplit\tw@ to \ht\@arstrutbox
  \setbox\tw@\vbox{\unvbox\tw@}%
  \vfuzz\dimen@ii
  \advance\dimen@ \ht
      \ifdim\ht\@arstrutbox>\ht\tw@\@arstrutbox\else\tw@\fi
  \advance\dimen@\dp
      \ifdim\dp\@arstrutbox>\dp\tw@\@arstrutbox\else\tw@\fi
  \advance\dimen@ -\pagegoal
  \ifdim \dimen@>\z@\vfil\break\fi
  \global\@colroom\@colht
  \ifvoid\LT@firstfoot
    \ifvoid\LT@foot
    \else
      \advance\vsize-\ht\LT@foot
      \global\advance\@colroom-\ht\LT@foot
      \dimen@\pagegoal\advance\dimen@-\ht\LT@foot\pagegoal\dimen@
      \maxdepth\z@
    \fi
  \else
    \advance\vsize-\ht\LT@firstfoot
    \global\advance\@colroom-\ht\LT@firstfoot
    \dimen@\pagegoal\advance\dimen@-\ht\LT@firstfoot\pagegoal\dimen@
    \maxdepth\z@
  \fi
  \ifvoid\LT@firsthead\copy\LT@head\else\box\LT@firsthead\fi\nobreak
  \output{\LT@output}%
}
\def\LT@output{%
  \ifnum\outputpenalty <-\@Mi
    \ifnum\outputpenalty > -\LT@end@pen
      \LT@err{floats and marginpars not allowed in a longtable}\@ehc
    \else
      \setbox\z@\vbox{\unvbox\@cclv}%
      \ifdim \ht\LT@lastfoot>\ht\LT@foot
        \dimen@\pagegoal
        \advance\dimen@-\ht\LT@lastfoot
        \ifdim\dimen@<\ht\z@
          \setbox\@cclv\vbox{\unvbox\z@\copy\LT@foot\vss}%
          \@makecol
          \@outputpage
          \setbox\z@\vbox{\box\LT@head}%
        \fi
      \fi  
      \global\@colroom\@colht
      \global\vsize\@colht   
      \vbox
        {\unvbox\z@\box\ifvoid\LT@lastfoot\LT@foot\else\LT@lastfoot\fi}%
    \fi
  \else
    \ifvoid\LT@firstfoot
      \setbox\@cclv\vbox{\unvbox\@cclv\copy\LT@foot\vss}%
      \@makecol
      \@outputpage
      \global\vsize\@colroom
    \else
      \setbox\@cclv\vbox{\unvbox\@cclv\box\LT@firstfoot\vss}%
      \@makecol
      \@outputpage
      \global\advance\@colroom\LT@footdiff
      \global\vsize\@colroom
    \fi
    \copy\LT@head\nobreak
  \fi
}
\makeatother % This is a user macro/style file.

\usepackage{tikz}
\usepackage{pgfplots}
\pgfplotsset{compat=1.17}
\usetikzlibrary{calc}
\usetikzlibrary{shapes}
\usetikzlibrary{patterns}

\usepackage{ifthen}

\tikzstyle{grids}=[%
    every path/.style = {%
        thick,%
        line cap = round%
    }, %
    baseline = (%
        current bounding box.center%
    )%
]
\tikzstyle{graphs}=[%
    every node/.style = {%
        circle,%
        fill=black,%
        inner sep=0pt,%
        minimum size=4pt%
    }%
]

%\usepackage{mathrsfs} % Required for \mathscr
\DeclareMathAlphabet{\mathcal}{OMS}{cmsy}{m}{n} % Recover nice-looking \mathcal


\usepackage[T1]{fontenc} % For font encoding?
\usepackage{nicefrac}
\usepackage{pgfplots} % pgfplots is tikz but better.
% \usepackage{amsrefs} % amsrefs contains the .bibtex style content for mathematician papers.
\usepackage{changepage}
\usepackage{pdflscape}
%\usepackage{adjustbox}
\usepackage{multirow}

% \usepackage{mathabx} % Redefines \oplus, \bigoplus, \ominus; Defines \bigominus

\newcommand{\gridCell}[2][0.25]{
    % Draw one cell with direction (#2).
    \begin{tikzpicture}[scale=1, baseline = 1pt]
        \pgftransformxscale{#1};
        \pgftransformyscale{#1};

        \draw [semithick, line cap=round] (0,0) rectangle (1,1);
        \ifthenelse{#2 = 1 \OR #2 = -1}{
            \ifthenelse{#2 > 0}{
                \draw [semithick, line cap=round] (0,0) -- (1,1);
            }{
                \draw [semithick, line cap=round] (0,1) -- (1,0);
            };
        }{};
    \end{tikzpicture}
}

\newcommand{\gridCellDot}[3][0.25]{
    % Draw one cell with direction (#2) with a dot located in quadrant (#3).
    \begin{tikzpicture}[scale=1, baseline = 1pt]
        \pgftransformxscale{#1};
        \pgftransformyscale{#1};

        \draw [semithick, line cap=round] (0,0) rectangle (1,1);
        \ifthenelse{#2 = 1 \OR #2 = -1}{
            \ifthenelse{#2 > 0}{
                \draw [semithick, line cap=round] (0,0) -- (1,1);
            }{
                \draw [semithick, line cap=round] (0,1) -- (1,0);
            };
        }{};

        \ifthenelse{#3 = 1}{
            \draw [draw=black, fill=black] (1,1) circle (4pt);
        }{};
        \ifthenelse{#3 = 2}{
            \draw [draw=black, fill=black] (0,1) circle (4pt);
        }{};
        \ifthenelse{#3 = 3}{
            \draw [draw=black, fill=black] (0,0) circle (4pt);
        }{};
        \ifthenelse{#3 = 4}{
            \draw [draw=black, fill=black] (1,0) circle (4pt);
        }{};
    \end{tikzpicture}
}

\newcommand{\gridVertOne}[2][0.25]{
    % Draw one vertical column of grid cells.
    % Example: \gridVertOne[0.5]{1,1,1} draws a column at scale 0.5 with three increasing cells.
    \begin{tikzpicture}[scale=1, baseline = 1pt]
        \pgftransformxscale{#1};
        \pgftransformyscale{#1};

        \foreach \dir [count = \i] in {#2} {
            \draw [semithick, line cap=round] (0,{\i-1}) rectangle (1,{\i});
            \ifthenelse{\dir = 1 \OR \dir = -1}{
                \ifthenelse{\dir > 0}{
                    \draw [semithick, line cap=round] (0,{\i-1}) -- (1,{\i});
                }{
                    \draw [semithick, line cap=round] (0,{\i}) -- (1,{\i-1});
                };
            }{};
        }
    \end{tikzpicture}
}

\newcommand{\gridVertTwo}[3][0.25]{
    \begin{tikzpicture}[scale=1, baseline = 1pt]

    \pgftransformxscale{#1};
    \pgftransformyscale{#1};

     \foreach \dir [count = \i] in {#2} {
        \draw [semithick, line cap=round] (0,{\i-1}) rectangle (1,{\i});
        \ifthenelse{\dir = 1 \OR \dir = -1}{
            \ifthenelse{\dir > 0}{
                \draw [semithick, line cap=round] (0,{\i-1}) -- (1,{\i});
            }{
                \draw [semithick, line cap=round] (0,{\i}) -- (1,{\i-1});
            };
        }{};
    }

    \foreach \dir [count = \i] in {#3} {
        \draw [semithick, line cap=round] (1,{\i-1}) rectangle (2,{\i});
        \ifthenelse{\dir = 1 \OR \dir = -1}{
            \ifthenelse{\dir > 0}{
                \draw [semithick, line cap=round] (1,{\i-1}) -- (2,{\i});
            }{
                \draw [semithick, line cap=round] (1,{\i}) -- (2,{\i-1});
            };
        }{};
    }
    
    \end{tikzpicture}
}

\newcommand{\gridVertThree}[4][0.25]{
    \begin{tikzpicture}[scale=1, baseline = 1pt]

    \pgftransformxscale{#1};
    \pgftransformyscale{#1};

     \foreach \dir [count = \i] in {#2} {
        \draw [semithick, line cap=round] (0,{\i-1}) rectangle (1,{\i});
        \ifthenelse{\dir = 1 \OR \dir = -1}{
            \ifthenelse{\dir > 0}{
                \draw [semithick, line cap=round] (0,{\i-1}) -- (1,{\i});
            }{
                \draw [semithick, line cap=round] (0,{\i}) -- (1,{\i-1});
            };
        }{};
    }

    \foreach \dir [count = \i] in {#3} {
        \draw [semithick, line cap=round] (1,{\i-1}) rectangle (2,{\i});
        \ifthenelse{\dir = 1 \OR \dir = -1}{
            \ifthenelse{\dir > 0}{
                \draw [semithick, line cap=round] (1,{\i-1}) -- (2,{\i});
            }{
                \draw [semithick, line cap=round] (1,{\i}) -- (2,{\i-1});
            };
        }{};
    }

    \foreach \dir [count = \i] in {#4} {
        \draw [semithick, line cap=round] (2,{\i-1}) rectangle (3,{\i});
        \ifthenelse{\dir = 1 \OR \dir = -1}{
            \ifthenelse{\dir > 0}{
                \draw [semithick, line cap=round] (2,{\i-1}) -- (3,{\i});
            }{
                \draw [semithick, line cap=round] (2,{\i}) -- (3,{\i-1});
            };
        }{};
    }
    
    \end{tikzpicture}
}

\newcommand{\gridVertFour}[5][0.25]{
    \begin{tikzpicture}[scale=1, baseline = 1pt]

    \pgftransformxscale{#1};
    \pgftransformyscale{#1};

     \foreach \dir [count = \i] in {#2} {
        \draw [semithick, line cap=round] (0,{\i-1}) rectangle (1,{\i});
        \ifthenelse{\dir = 1 \OR \dir = -1}{
            \ifthenelse{\dir > 0}{
                \draw [semithick, line cap=round] (0,{\i-1}) -- (1,{\i});
            }{
                \draw [semithick, line cap=round] (0,{\i}) -- (1,{\i-1});
            };
        }{};
    }

    \foreach \dir [count = \i] in {#3} {
        \draw [semithick, line cap=round] (1,{\i-1}) rectangle (2,{\i});
        \ifthenelse{\dir = 1 \OR \dir = -1}{
            \ifthenelse{\dir > 0}{
                \draw [semithick, line cap=round] (1,{\i-1}) -- (2,{\i});
            }{
                \draw [semithick, line cap=round] (1,{\i}) -- (2,{\i-1});
            };
        }{};
    }

    \foreach \dir [count = \i] in {#4} {
        \draw [semithick, line cap=round] (2,{\i-1}) rectangle (3,{\i});
        \ifthenelse{\dir = 1 \OR \dir = -1}{
            \ifthenelse{\dir > 0}{
                \draw [semithick, line cap=round] (2,{\i-1}) -- (3,{\i});
            }{
                \draw [semithick, line cap=round] (2,{\i}) -- (3,{\i-1});
            };
        }{};
    }

    \foreach \dir [count = \i] in {#5} {
        \draw [semithick, line cap=round] (3,{\i-1}) rectangle (4,{\i});
        \ifthenelse{\dir = 1 \OR \dir = -1}{
            \ifthenelse{\dir > 0}{
                \draw [semithick, line cap=round] (3,{\i-1}) -- (4,{\i  });
            }{
                \draw [semithick, line cap=round] (3,{\i  }) -- (4,{\i-1});
            };
        }{};
    }
    
    \end{tikzpicture}
}

\newcommand{\gridHorizOne}[2][0.25]{
    \begin{tikzpicture}[scale=1, baseline = 1pt]

    \pgftransformxscale{#1};
    \pgftransformyscale{#1};

     \foreach \dir [count = \i] in {#2} {
        \draw [semithick, line cap=round] ({\i-1},0) rectangle ({\i},1);
        \ifthenelse{\dir = 1 \OR \dir = -1}{
            \ifthenelse{\dir > 0}{
                \draw [semithick, line cap=round] ({\i-1},0) -- ({\i},1);
            }{
                \draw [semithick, line cap=round] ({\i},0) -- ({\i-1},1);
            };
        }{};
    }
    
    \end{tikzpicture}
}

\newcommand{\gridHorizTwo}[3][0.25]{
    \begin{tikzpicture}[scale=1, baseline = 1pt]

    \pgftransformxscale{#1};
    \pgftransformyscale{#1};

     \foreach \dir [count = \i] in {#2} {
        \draw [semithick, line cap=round] ({\i-1},0) rectangle ({\i},1);
        \ifthenelse{\dir = 1 \OR \dir = -1}{
            \ifthenelse{\dir > 0}{
                \draw [semithick, line cap=round] ({\i-1},0) -- ({\i},1);
            }{
                \draw [semithick, line cap=round] ({\i},0) -- ({\i-1},1);
            };
        }{};
    }

    \foreach \dir [count = \i] in {#3} {
        \draw [semithick, line cap=round] ({\i-1},1) rectangle ({\i},2);
        \ifthenelse{\dir = 1 \OR \dir = -1}{
            \ifthenelse{\dir > 0}{
                \draw [semithick, line cap=round] ({\i-1},1) -- ({\i},2);
            }{
                \draw [semithick, line cap=round] ({\i},1) -- ({\i-1},2);
            };
        }{};
    }
    
    \end{tikzpicture}
}

\newcommand{\gridHorizThree}[4][0.25]{
    \begin{tikzpicture}[scale=1, baseline = 1pt]

        \pgftransformxscale{#1};
        \pgftransformyscale{#1};

         \foreach \dir [count = \i] in {#2} {
            \draw [semithick, line cap=round] ({\i-1},0) rectangle ({\i},1);
            \ifthenelse{\dir = 1 \OR \dir = -1}{
                \ifthenelse{\dir > 0}{
                    \draw [semithick, line cap=round] ({\i-1},0) -- ({\i},1);
                }{
                    \draw [semithick, line cap=round] ({\i},0) -- ({\i-1},1);
                };
            }{};
        }

        \foreach \dir [count = \i] in {#3} {
            \draw [semithick, line cap=round] ({\i-1},1) rectangle ({\i},2);
            \ifthenelse{\dir = 1 \OR \dir = -1}{
                \ifthenelse{\dir > 0}{
                    \draw [semithick, line cap=round] ({\i-1},1) -- ({\i},2);
                }{
                    \draw [semithick, line cap=round] ({\i},1) -- ({\i-1},2);
                };
            }{};
        }

        \foreach \dir [count = \i] in {#4} {
            \draw [semithick, line cap=round] ({\i-1},2) rectangle ({\i},3);
            \ifthenelse{\dir = 1 \OR \dir = -1}{
                \ifthenelse{\dir > 0}{
                    \draw [semithick, line cap=round] ({\i-1},2) -- ({\i},3);
                }{
                    \draw [semithick, line cap=round] ({\i},2) -- ({\i-1},3);
                };
            }{};
        }
    
    \end{tikzpicture}
}

\newcommand{\gridCellDisplay}[2][1.0]{
    % Draw one cell with direction (#2).
    \begin{tikzpicture}[scale=1, baseline = (current bounding box.center)]
        \pgftransformxscale{#1};
        \pgftransformyscale{#1};

        \draw [thick, line cap=round] (0,0) rectangle (1,1);
        \ifthenelse{#2 = 1 \OR #2 = -1}{
            \ifthenelse{#2 > 0}{
                \draw [thick, line cap=round] (0,0) -- (1,1);
            }{
                \draw [thick, line cap=round] (0,1) -- (1,0);
            };
        }{};
    \end{tikzpicture}
}

\newcommand{\gridCellDotDisplay}[3][1.0]{
    % Draw one cell with direction (#2) with a dot located in quadrant (#3).
    \begin{tikzpicture}[scale=1, baseline = (current bounding box.center)]
        \pgftransformxscale{#1};
        \pgftransformyscale{#1};

        \draw [thick, line cap=round] (0,0) rectangle (1,1);
        \ifthenelse{#2 = 1 \OR #2 = -1}{
            \ifthenelse{#2 > 0}{
                \draw [thick, line cap=round] (0,0) -- (1,1);
            }{
                \draw [thick, line cap=round] (0,1) -- (1,0);
            };
        }{};

        \ifthenelse{#3 = 1}{
            \draw [draw=black, fill=black] (1,1) circle (2pt);
        }{};
        \ifthenelse{#3 = 2}{
            \draw [draw=black, fill=black] (0,1) circle (2pt);
        }{};
        \ifthenelse{#3 = 3}{
            \draw [draw=black, fill=black] (0,0) circle (2pt);
        }{};
        \ifthenelse{#3 = 4}{
            \draw [draw=black, fill=black] (1,0) circle (2pt);
        }{};
    \end{tikzpicture}
}

\newcommand{\gridVertOneDisplay}[2][1.0]{
    % Draw one vertical column of grid cells.
    % Example: \gridVertOne[0.5]{1,1,1} draws a column at scale 0.5 with three increasing cells.
    \begin{tikzpicture}[scale=1, baseline = (current bounding box.center)]
        \pgftransformxscale{#1};
        \pgftransformyscale{#1};

        \foreach \dir [count = \i] in {#2} {
            \draw [thick, line cap=round] (0,{\i-1}) rectangle (1,{\i});
            \ifthenelse{\dir = 1 \OR \dir = -1}{
                \ifthenelse{\dir > 0}{
                    \draw [thick, line cap=round] (0,{\i-1}) -- (1,{\i});
                }{
                    \draw [thick, line cap=round] (0,{\i}) -- (1,{\i-1});
                };
            }{};
        }
    \end{tikzpicture}
}

\newcommand{\gridVertTwoDisplay}[3][1.0]{
    \begin{tikzpicture}[scale=1, baseline = (current bounding box.center)]

    \pgftransformxscale{#1};
    \pgftransformyscale{#1};

     \foreach \dir [count = \i] in {#2} {
        \draw [thick, line cap=round] (0,{\i-1}) rectangle (1,{\i});
        \ifthenelse{\dir = 1 \OR \dir = -1}{
            \ifthenelse{\dir > 0}{
                \draw [thick, line cap=round] (0,{\i-1}) -- (1,{\i});
            }{
                \draw [thick, line cap=round] (0,{\i}) -- (1,{\i-1});
            };
        }{};
    }

    \foreach \dir [count = \i] in {#3} {
        \draw [thick, line cap=round] (1,{\i-1}) rectangle (2,{\i});
        \ifthenelse{\dir = 1 \OR \dir = -1}{
            \ifthenelse{\dir > 0}{
                \draw [thick, line cap=round] (1,{\i-1}) -- (2,{\i});
            }{
                \draw [thick, line cap=round] (1,{\i}) -- (2,{\i-1});
            };
        }{};
    }
    
    \end{tikzpicture}
}

\newcommand{\gridVertThreeDisplay}[4][1.0]{
    \begin{tikzpicture}[scale=1, baseline = (current bounding box.center)]

    \pgftransformxscale{#1};
    \pgftransformyscale{#1};

     \foreach \dir [count = \i] in {#2} {
        \draw [thick, line cap=round] (0,{\i-1}) rectangle (1,{\i});
        \ifthenelse{\dir = 1 \OR \dir = -1}{
            \ifthenelse{\dir > 0}{
                \draw [thick, line cap=round] (0,{\i-1}) -- (1,{\i});
            }{
                \draw [thick, line cap=round] (0,{\i}) -- (1,{\i-1});
            };
        }{};
    }

    \foreach \dir [count = \i] in {#3} {
        \draw [thick, line cap=round] (1,{\i-1}) rectangle (2,{\i});
        \ifthenelse{\dir = 1 \OR \dir = -1}{
            \ifthenelse{\dir > 0}{
                \draw [thick, line cap=round] (1,{\i-1}) -- (2,{\i});
            }{
                \draw [thick, line cap=round] (1,{\i}) -- (2,{\i-1});
            };
        }{};
    }

    \foreach \dir [count = \i] in {#4} {
        \draw [thick, line cap=round] (2,{\i-1}) rectangle (3,{\i});
        \ifthenelse{\dir = 1 \OR \dir = -1}{
            \ifthenelse{\dir > 0}{
                \draw [thick, line cap=round] (2,{\i-1}) -- (3,{\i});
            }{
                \draw [thick, line cap=round] (2,{\i}) -- (3,{\i-1});
            };
        }{};
    }
    
    \end{tikzpicture}
}

\newcommand{\gridHorizOneDisplay}[2][1.0]{
    \begin{tikzpicture}[scale=1, baseline = (current bounding box.center)]

    \pgftransformxscale{#1};
    \pgftransformyscale{#1};

     \foreach \dir [count = \i] in {#2} {
        \draw [thick, line cap=round] ({\i-1},0) rectangle ({\i},1);
        \ifthenelse{\dir = 1 \OR \dir = -1}{
            \ifthenelse{\dir > 0}{
                \draw [thick, line cap=round] ({\i-1},0) -- ({\i},1);
            }{
                \draw [thick, line cap=round] ({\i},0) -- ({\i-1},1);
            };
        }{};
    }
    
    \end{tikzpicture}
}

\newcommand{\gridHorizTwoDisplay}[3][1.0]{
    \begin{tikzpicture}[scale=1, baseline = (current bounding box.center)]

    \pgftransformxscale{#1};
    \pgftransformyscale{#1};

     \foreach \dir [count = \i] in {#2} {
        \draw [thick, line cap=round] ({\i-1},0) rectangle ({\i},1);
        \ifthenelse{\dir = 1 \OR \dir = -1}{
            \ifthenelse{\dir > 0}{
                \draw [thick, line cap=round] ({\i-1},0) -- ({\i},1);
            }{
                \draw [thick, line cap=round] ({\i},0) -- ({\i-1},1);
            };
        }{};
    }

    \foreach \dir [count = \i] in {#3} {
        \draw [thick, line cap=round] ({\i-1},1) rectangle ({\i},2);
        \ifthenelse{\dir = 1 \OR \dir = -1}{
            \ifthenelse{\dir > 0}{
                \draw [thick, line cap=round] ({\i-1},1) -- ({\i},2);
            }{
                \draw [thick, line cap=round] ({\i},1) -- ({\i-1},2);
            };
        }{};
    }
    
    \end{tikzpicture}
}

\newcommand{\gridHorizThreeDisplay}[4][1.0]{
    \begin{tikzpicture}[scale=1, baseline = (current bounding box.center)]

    \pgftransformxscale{#1};
    \pgftransformyscale{#1};

     \foreach \dir [count = \i] in {#2} {
        \draw [thick, line cap=round] ({\i-1},0) rectangle ({\i},1);
        \ifthenelse{\dir = 1 \OR \dir = -1}{
            \ifthenelse{\dir > 0}{
                \draw [thick, line cap=round] ({\i-1},0) -- ({\i},1);
            }{
                \draw [thick, line cap=round] ({\i},0) -- ({\i-1},1);
            };
        }{};
    }

    \foreach \dir [count = \i] in {#3} {
        \draw [thick, line cap=round] ({\i-1},1) rectangle ({\i},2);
        \ifthenelse{\dir = 1 \OR \dir = -1}{
            \ifthenelse{\dir > 0}{
                \draw [thick, line cap=round] ({\i-1},1) -- ({\i},2);
            }{
                \draw [thick, line cap=round] ({\i},1) -- ({\i-1},2);
            };
        }{};
    }

    \foreach \dir [count = \i] in {#4} {
        \draw [thick, line cap=round] ({\i-1},2) rectangle ({\i},3);
        \ifthenelse{\dir = 1 \OR \dir = -1}{
            \ifthenelse{\dir > 0}{
                \draw [thick, line cap=round] ({\i-1},2) -- ({\i},3s);
            }{
                \draw [thick, line cap=round] ({\i},2) -- ({\i-1},3s);
            };
        }{};
    }
    
    \end{tikzpicture}
}
\newcommand{\skyline}[1]{
    \foreach [count=\col] \val in {#1} {
        \foreach \y in {1,...,\val} {
            \draw[thick, line cap = round] (\col, \y) rectangle ++(1,1);
        }
    }
}

\newcommand{\skylinelabeled}[1]{
    \foreach [count=\col] \val in {#1} {
        \foreach \y in {1,...,\val} {
            \draw[thick, line cap = round] (\col, \y) rectangle ++(1,1);
        }
        \node at ({\col+0.5}, 0) {$\val$};
    }
}

\newcommand{\skylinefilled}[1]{
    \foreach [count=\col] \val in {#1} {
        \ifthenelse{\equal{\val}{0}}{}{
            \foreach \y in {1,...,\val} {
                \draw[fill=gray!50, thick, line cap = round] (\col, \y) rectangle ++(1,1);
            }
        }
    }
}
\newcommand{\skylinefilledunderlines}[1]{
    \foreach [count=\col] \val in {#1} {
        \ifthenelse{\equal{\val}{0}}{}{
            \foreach \y in {1,...,\val} {
                \draw[fill=gray!50, thick, line cap = round] (\col, \y) rectangle ++(1,1);
            }
            \draw[shorten <= 2pt, shorten >= 2pt] (\col, -0.55) -- ++(1,0);
        }
    }
}

\newcommand{\ferrers}[1]{
    \foreach [count=\row] \val in {#1} {
        \foreach \x in {1,...,\val} {
            \draw[thick, line cap = round] (\x, -\row) rectangle ++(1,1);
        }
    }
}

\newcommand{\ferrersfilled}[1]{
    \foreach [count=\row] \val in {#1} {
        \foreach \x in {1,...,\val} {
            \draw[fill=gray!50, thick, line cap = round] (\x, -\row) rectangle ++(1,1);
        }
    }
}
%%%%%%%%%%%%%%%%%%%%%%%%%%%%%%%%%%%%%%%%%%%%%%%%%%%%%%%%%%%%%%%%%%%
% Vince's Tikz macros                                             %
% Version of 19 June 2016                                         %
%%%%%%%%%%%%%%%%%%%%%%%%%%%%%%%%%%%%%%%%%%%%%%%%%%%%%%%%%%%%%%%%%%%

\usepackage{pgfmath}

%
%
%
%

% Points of absolute size (to be used when drawing matchings and permutations).
% Call \absdot[label]{(2,3)} for label, otherwise omit optional argument.

\newcommand\mybullet{\raisebox{-5pt}{\normalsize \ensuremath{\bullet}}}
\newcommand\mycirc{\raisebox{-5pt}{\normalsize \ensuremath{\circ}}}
\newcommand\myscaledbullet[1]{\raisebox{-5pt}{\scalebox{#1}{\normalsize \ensuremath{\bullet}}}}

\makeatletter
\def\absdot{\@ifnextchar[{\@absdotlabel}{\@absdotnolabel}}
    \def\@absdotlabel[#1]#2{%
        \node at #2 {\normalsize \mybullet};
        \node at #2 [below=2pt] {\ensuremath{#1}};
    }
    \def\@absdotnolabel#1{%
        \node at #1 {\normalsize \mybullet};
    }
\def\absdothollow{\@ifnextchar[{\@absdothollowlabel}{\@absdothollownolabel}}
    % Make a dot of fixed absolute size.
    % Note that \absdothollow first overwrites what it is going on top of.
    \def\@absdothollowlabel[#1]#2{%
        \node at #2 {\normalsize \textcolor{white}{\mybullet}};
        \node at #2 {\normalsize \mycirc};
        \node at #2 [below=2pt] {\ensuremath{#1}};
    }
    \def\@absdothollownolabel#1{%
        \node at #1 {\normalsize \textcolor{white}{\mybullet}};
        \node at #1 {\normalsize \mycirc};
    }
\makeatother

%
%
%
%

% Plotting permutations. Recommended scale: 0.25 (and possibly up to 0.5).

\newcommand{\plotperm}[1]{
    \foreach \j [count=\i] in {#1} {
        \absdot{(\i,\j)};
    };
}

\newcommand{\plotpermprecise}[2]{
    \foreach \j [count=\i] in {#1} {
        \draw[black,fill=black] (\i,\j) circle (#2);
    };
}

\newcommand{\plotpermscaled}[2]{
    \foreach \j [count=\i] in {#1} {
        \node at (\i,\j) {\scalebox{#2}{\normalsize \ensuremath{\bullet}}};
    };
}

\newcommand{\plotpermhollow}[1]{
    \foreach \j [count=\i] in {#1} {
        \absdothollow{(\i,\j)};
    };
}

\newcommand{\plotpartialperm}[1]{
    \foreach \i/\j in {#1} {
        \absdot{(\i,\j)};
    };
}

\newcommand\myencircle{\raisebox{-8.75pt}{\LARGE \ensuremath{\circ}}}
\newcommand{\plotpartialpermencircle}[1]{
    \foreach \i/\j in {#1} {
        \node at (\i,\j) {\myencircle};
    };
}

\newcommand{\plotpartialpermhollow}[1]{
    \foreach \i/\j in {#1} {
        \absdothollow{(\i,\j)};
    };
}

\newcommand{\plotpermbox}[4]{
    \draw [darkgray, thick, line cap=round]
        ({#1-0.5}, {#2-0.5}) rectangle ({#3+0.5}, {#4+0.5});
}

\newcommand{\plotpermborder}[1]{
    \plotperm{#1};
    \pgfmathsetmacro\n{dim(#1)};
    % Now \n stores the number of entries of the permutation. Draw the border.
    \plotpermbox{1}{1}{\n}{\n};
}

\newcommand{\plotpermgraphedges}[1]{
    \foreach \j [count=\i] in {#1} {
        \foreach \b [count=\a] in {#1} {
            % Draw edge from (a,b) to (i,j) if they form an inversion.
            \ifthenelse{\a<\i \AND \b>\j}{\draw (\a,\b)--(\i,\j);}{}
        };
    };
}

\newcommand{\plotpermgraph}[1]{
    \foreach \j [count=\i] in {#1} {
        \foreach \b [count=\a] in {#1} {
            % Draw edge from (a,b) to (i,j) if they form an inversion.
            \ifthenelse{\a<\i \AND \b>\j}{\draw (\a,\b)--(\i,\j);}{}
        };
    };
    \plotperm{#1};
}

\newcommand{\plotpermgraphborder}[1]{
    \foreach \j [count=\i] in {#1} {
        \foreach \b [count=\a] in {#1} {
            % Draw edge from (a,b) to (i,j) if they form an inversion.
            \ifthenelse{\a<\i \AND \b>\j}{\draw (\a,\b)--(\i,\j);}{}
        };
    };
    \plotpermborder{#1};
}

\newcommand{\plotpermgrid}[1]{
    \pgfmathsetmacro\n{dim(#1)};
    % Now \n stores the number of entries of the permutation. Draw the border.
    \foreach \i in {0,1,2,...,\n}{
        \draw[color=darkgray] ({\i+0.5}, 0.5)--({\i+0.5}, {\n+0.5});
        \draw[color=darkgray] (0.5, {\i+0.5})--({\n+0.5}, {\i+0.5});
    }
    \plotperm{#1};
}

\newcommand{\plotpermdyckgrid}[1]{
    \pgfmathsetmacro\n{dim(#1)};
    % Now \n stores the number of entries of the permutation. Draw the border.
    \foreach \i in {0,1,2,...,\n}{
        \draw[color=lightgray] ({\i+0.5}, 0.5)--({\i+0.5}, {\n+0.5});
        \draw[color=lightgray] (0.5, {\i+0.5})--({\n+0.5}, {\i+0.5});
    }
    \draw[ultra thick, line cap=round, color=darkgray] (0.5, 0.5)--({\n+0.5}, {\n+0.5});
    \plotperm{#1};
}

\newcommand{\plotpermdyckpath}[1]{
    % Pass a list of coordinates. Everything is shifted, so the lower-left of the grid is the origin.
    \draw[ultra thick, line cap=round] (0.5,0.5)
    \foreach \step in {#1} {
        \ifnum\step=1
            -- ++(0,1)
        \else
            -- ++(1,0)
        \fi
    };
}

%
%
%
%

% Drawing Dyck paths. Pass the command a list of 1s and -1s.
% Note that this path will be a factor of sqrt(2) larger than \plotpermdyckpath
% Adapted from http://tex.stackexchange.com/questions/63540/how-to-draw-a-catalan-number-diagram-on-tikz

\newcommand{\plotdyckpath}[1]{
    \draw[ultra thick, line cap=round] (0.5,0)
    \foreach \step in {#1} {
        \ifnum\step=1
            -- ++(1,1)
        \else
            -- ++(1,-1)
        \fi
    };
}

%
%
%
%

% Drawing (large) matchings. Recommended scale is around 0.5.

\newcommand{\arcskinnyplain}[2]{
    \draw[thick] (#1,0) arc (180:0:{(#2-#1)/2});
}

\newcommand{\arc}[2]{
    \draw[thick] (#1,0) arc (180:0:{(#2-#1)/2});
    \absdot{(#1,0)};
    \absdot{(#2,0)};
}

\newcommand{\arcgray}[2]{
    \draw[thick, lightgray] (#1,0) arc (180:0:{(#2-#1)/2});
    \absdot{(#1,0)};
    \absdot{(#2,0)};
}

\newcommand{\arcdotted}[2]{
    \draw[thick, dotted, line cap=round] (#1,0) arc (180:0:{(#2-#1)/2});
    \absdot{(#1,0)};
    \absdot{(#2,0)};
}

\newcommand{\arclabel}[4]{
    \draw[thick] (#1,0) arc (180:0:{(#2-#1)/2});
    % The \mathstrut below gets the vertical alignment correct if you
    %   are using labels like a, b, c, d, e, ...
    \absdot[\ensuremath{\mathstrut #3}]{(#1,0)};
    \absdot[\ensuremath{\mathstrut #4}]{(#2,0)};
}

\newcommand{\arclabelgray}[4]{
    \draw[thick, darkgray] (#1,0) arc (180:0:{(#2-#1)/2});
    \absdot[\ensuremath{\mathstrut #3}]{(#1,0)};
    \absdot[\ensuremath{\mathstrut #4}]{(#2,0)};
}

\newcommand{\arclabeldotted}[4]{
    \draw[thick, dotted, line cap=round] (#1,0) arc (180:0:{(#2-#1)/2});
    \absdot[\ensuremath{\mathstrut #3}]{(#1,0)};
    \absdot[\ensuremath{\mathstrut #4}]{(#2,0)};
}

\newcommand{\arclabelgroup}[2]{
    \draw ({#1-0.25}, -1) rectangle ({#2+0.25}, -0.25);
}

\newcommand{\arclabelgroupgray}[2]{
    \draw[lightgray, fill=lightgray] ({#1-0.25}, -1) rectangle ({#2+0.25}, -0.3);
}

%
%
%
%

\newcommand{\matching}[1]{
    \foreach \i/\j in {#1} {
        \arc{\i}{\j};
    };
}

\newcommand{\matchingdotted}[1]{
    \foreach \i/\j in {#1} {
        \arcdotted{\i}{\j};
    };
}

\newcommand{\matchinglabels}[1]{
    \foreach \i/\j in {#1} {
        \arclabel{\i}{\j}{\i}{\j};
    };
}

\newcommand{\matchinglabelsdotted}[1]{
    \foreach \i/\j in {#1} {
        \arclabeldotted{\i}{\j}{\i}{\j};
    };
}

\newcommand{\matchinglabelsgray}[1]{
    \foreach \i/\j in {#1} {
        \arclabelgray{\i}{\j}{\i}{\j};
    };
}

\newcommand{\matchingperm}[1]{
    \pgfmathsetmacro\n{dim(#1)};
    % Now \n stores the number of entries of the permutation.
    \foreach \j [count=\i] in {#1} {
        \arclabel{\i}{{2*\n+1-\j}}{\i}{\j};
    };
}

\newcommand{\matchingshadeedge}[1]{
    % Just like drawing an edge, e.g., 3/6
    \foreach \i/\j in {#1} {
        \draw[color=lightgray, fill=lightgray] ({\i-0.25},0)
            arc (180:0:{(\j-\i)/2 + 0.25})
            arc (0:-180:0.25)
            arc (0:180:{(\j-\i)/2 - 0.25})
            arc (0:-180:0.25)--cycle;   
    };
}

\newcommand{\matchingshadeendpoints}[1]{
    % Multiple groups can be input, like 1/4, 5/8
    \foreach \i/\j in {#1} {
        \draw[color=lightgray, fill=lightgray] (\i,-0.25)
            arc (270:90:0.25)
            --(\j,0.25)
            arc (90:-90:0.25)
            --(\i,-0.25);
    };
}

\newcommand{\matchingshadegroup}[4]{
    % Give two groups, like {1}{4}{5}{8}. Will shade whole region of edges from vertices 1-4 to vertices 5-8.
    \draw[color=lightgray, fill=lightgray] ({#1-0.25},0)
        arc (180:0:{(#4-#1)/2 + 0.25}) % outer arc
        arc (0:-90:0.25)
        --(#3,-0.25) % right-hand bottom
        arc (270:180:0.25)
        arc (0:180:{(#3-#2)/2 - 0.25}) % inner arc
        arc (0:-90:0.25)
        --(#1,-0.25) % left-hand bottom
        arc (270:180:0.25)--cycle;
}

\newcommand{\matchinggroupdark}[4]{
    % Give two groups, like {1}{4}{5}{8}. Will shade whole region of edges from vertices 1-4 to vertices 5-8.
    \draw ({#1-0.25},0)
        arc (180:0:{(#4-#1)/2 + 0.25}) % outer arc
        arc (0:-90:0.25)
        --(#3,-0.25) % right-hand bottom
        arc (270:180:0.25)
        arc (0:180:{(#3-#2)/2 - 0.25}) % inner arc
        arc (0:-90:0.25)
        --(#1,-0.25) % left-hand bottom
        arc (270:180:0.25)--cycle;
}

\newcommand{\matchingshadegroupdark}[4]{
    % Give two groups, like {1}{4}{5}{8}. Will shade whole region of edges from vertices 1-4 to vertices 5-8.
    \draw[color=black, fill=lightgray] ({#1-0.25},0)
        arc (180:0:{(#4-#1)/2 + 0.25}) % outer arc
        arc (0:-90:0.25)
        --(#3,-0.25) % right-hand bottom
        arc (270:180:0.25)
        arc (0:180:{(#3-#2)/2 - 0.25}) % inner arc
        arc (0:-90:0.25)
        --(#1,-0.25) % left-hand bottom
        arc (270:180:0.25)--cycle;
}

%
%
%
%

% Drawing (small) matchings (all have width = 0.1*(verts-1) and height=0.15

\newcommand{\matchnest}{
    \begin{tikzpicture}[scale=.1, yscale=1.5, anchor=base]
        \arcskinnyplain{1}{4};
        \arcskinnyplain{2}{3};
    \end{tikzpicture}
}

\newcommand{\matchcross}{
    \begin{tikzpicture}[scale=.1, yscale=2.25, anchor=base]
        \arcskinnyplain{1}{3};
        \arcskinnyplain{2}{4};
    \end{tikzpicture}
}

\newcommand{\matchind}{
    \begin{tikzpicture}[scale=.1, yscale=4.5, anchor=base]
        \arcskinnyplain{1}{2};
        \arcskinnyplain{3}{4};
    \end{tikzpicture}
}

\newcommand{\matchsmall}[1]{
    \begin{tikzpicture}[scale=.1, anchor=base]
        % We need to read through so that we can rescale y so that the height is precisely 0.15
        \def\h{0};
        \def\maxh{0};
        \foreach \i/\j in {#1} {
            \pgfmathparse{\j-\i};
            \let\h\pgfmathresult;
            \pgfmathifthenelse{\h>\maxh}{\h}{\maxh};
            \global\let\maxh\pgfmathresult;
        };
        \pgftransformyscale{{4.5/\maxh}};
        \foreach \i/\j in {#1} {
            \arcskinnyplain{\i}{\j};
        };
    \end{tikzpicture}
}

\newcommand{\matchpermsmall}[1]{
    \begin{tikzpicture}[scale=.1, anchor=base]
        \pgfmathsetmacro\n{dim(#1)};
        % Now \n stores the length of the permutation.
        % We need to read through again so that we can rescale y so that the height is precisely 0.15
        \def\h{0};
        \def\maxh{0};
        \foreach \j [count=\i] in {#1} {
            \pgfmathparse{2*\n+1-\j-\i};
            \let\h\pgfmathresult;
            \pgfmathifthenelse{\h>\maxh}{\h}{\maxh};
            \global\let\maxh\pgfmathresult;
        };
        \pgftransformyscale{{4.5/\maxh}};
        \foreach \j [count=\i] in {#1} {
            \arcskinnyplain{\i}{{2*\n+1-\j}};
        };
    \end{tikzpicture}
}

%
%
%
%

% Plotting partitions.
\newcommand{\plotpartition}[1]{
    \foreach \i [count=\j] in {#1} {
        \draw [thick, line cap=round] (0,-\j)--(\i,-\j)--(\i,-\j-1)--(0,-\j-1)--cycle;
        \foreach \k in {1,2,...,\i}{
            \draw [thick, line cap=round] ({\k-1},-\j)--({\k-1},{-\j-1});
        };
    };
}
\newcommand{\plotpartitiongray}[1]{
    \foreach \i [count=\j] in {#1} {
        \draw [lightgray, thick, line cap=round] (0,-\j)--(\i,-\j)--(\i,-\j-1)--(0,-\j-1)--cycle;
        \foreach \k in {1,2,...,\i}{
            \draw [lightgray, thick, line cap=round] ({\k-1},-\j)--({\k-1},{-\j-1});
        };
    };
}

% Example:
%       \begin{center}
%           \begin{tikzpicture}[scale=0.25]
%               \plotpartitiongray{5,3,2,1,6,1,1};
%               \plotpartition{4,1,1,1,1};
%           \end{tikzpicture}
%       \end{center}

%
%
%
%

% Plotting compositions (skyline diagrams).
\newcommand{\plotskyline}[1]{
    \foreach \j [count=\i] in {#1} {
        \draw [thick, line cap=round] (\i-1,0)--(\i-1,\j)--(\i,\j)--(\i,0)--cycle;
        \foreach \k in {1,2,...,\j}{
            \draw [thick, line cap=round] (\i-1,\k-1)--(\i,\k-1);
        };
    };
}
\newcommand{\plotskylinegray}[1]{
    \foreach \j [count=\i] in {#1} {
        \draw [lightgray, thick, line cap=round] (\i-1,0)--(\i-1,\j)--(\i,\j)--(\i,0)--cycle;
        \foreach \k in {1,2,...,\j}{
            \draw [lightgray, thick, line cap=round] (\i-1,\k-1)--(\i,\k-1);
        };
    };
}
\newcommand{\plotskylineshaded}[1]{
    \foreach \j [count=\i] in {#1} {
        \draw [lightgray, fill=lightgray, thick, line cap=round] (\i-1,0)--(\i-1,\j)--(\i,\j)--(\i,0)--cycle;
        \foreach \k in {1,2,...,\j}{
            \draw [lightgray, fill=lightgray, thick, line cap=round] (\i-1,\k-1)--(\i,\k-1);
        };
    };
}

% Example:
%       \begin{center}
%           \begin{tikzpicture}[scale=0.25]
%               \plotskylinegray{5,3,2,4,6,1,2};
%               \plotskyline{4,1,1,1,5,1,2};
%           \end{tikzpicture}
%       \end{center}

%
%
%
%

%%%%%%%%%%%%%%%%%%%%%%%%%%%%%%%%%%%%%%%%%%%%%%%%%%%%%%%%%%%%%%%%
%
% Code for inline 1x2 (vert) and 2x1 (horiz) grids. June 2017.
%
%%%%%%%%%%%%%%%%%%%%%%%%%%%%%%%%%%%%%%%%%%%%%%%%%%%%%%%%%%%%%%%%

\newcommand{\onegrid}[1]{
\begin{tikzpicture}[scale=1, baseline = 1pt]
    \pgftransformxscale{0.25};
    \pgftransformyscale{0.25};
    \draw [semithick, line cap=round] (0,0) rectangle (1,1);
    \ifthenelse{#1>0}{
        \draw [semithick, line cap=round] (0,0)--(1,1);
    }{
        \draw [semithick, line cap=round] (0,1)--(1,0);
    };
\end{tikzpicture}
}

\newcommand{\onegriddotNW}{
\begin{tikzpicture}[scale=1, baseline = 1pt]
    \pgftransformxscale{0.25};
    \pgftransformyscale{0.25};
    \draw [draw=black, fill=black] (0,1) circle (4pt);
    \draw [semithick, line cap=round] (0,0) rectangle (1,1);    
    \draw [semithick, line cap=round] (0,0)--(1,1);
\end{tikzpicture}
}

\newcommand{\onegriddotSE}{
\begin{tikzpicture}[scale=1, baseline = 1pt]
    \pgftransformxscale{0.25};
    \pgftransformyscale{0.25};
    \draw [draw=black, fill=black] (1,0) circle (4pt);
    \draw [semithick, line cap=round] (0,0) rectangle (1,1);    
    \draw [semithick, line cap=round] (0,0)--(1,1);
\end{tikzpicture}
}

\newcommand{\gridsmallhoriz}[1]{
% Parameters should be +1 or -1, such as \gridsmallhoriz{1,-1,-1}.
    \begin{tikzpicture}[scale=1, baseline = 1pt]
        \def\gridheight{1};
        \def\gridwidth{1}
        % \foreach \dir [count=\gridwidth] in {#1} {
        %     % We read through first to determine how wide the grid is.
        % };
        \pgftransformxscale{{0.25/\gridwidth}};
        \pgftransformyscale{{0.25/\gridheight}};
        \foreach \dir [count=\i] in {#1} {
            % Now we have to draw the lines. We start at (0,0) or (0,1).
            \ifthenelse{\dir>0}{
                \draw [semithick, line cap=round] ({\i-1}, 0)--(\i, 1);
            }{
                \draw [semithick, line cap=round] ({\i-1}, 1)--(\i, 0);
            };
            \draw [semithick, line cap = round] ({\i-1}, 0) rectangle ({\i},1);
        };
    \end{tikzpicture}
}

\newcommand{\gridhoriz}[1]{
 % Parameters should be +1 or -1, such as \gridsmallhoriz{1,-1,-1}.
\begin{tikzpicture}[scale=1, anchor=base]
 \pgftransformxscale{0.225};
\pgftransformyscale{0.225};
   \foreach \dir [count=\i] in {#1} {
     % Now we have to draw the lines. We start at (0,0) or (0,1).
        \ifthenelse{\dir>0}{
            \draw [semithick, line cap=round] ({\i-1}, 0)--(\i, 1);
        }{
            \draw [semithick, line cap=round] ({\i-1}, 1)--(\i, 0);
        };
    \draw [semithick, line cap = round] ({\i-1}, 0) rectangle ({\i},1);
   };
 \end{tikzpicture}
}

\newcommand{\gridhorizframe}[1]{
 % Parameters should be +1 or -1, such as \gridsmallhoriz{1,-1,-1}.
\begin{tikzpicture}[scale=1, anchor=base]
 \pgftransformxscale{0.225};
\pgftransformyscale{0.225};
   \foreach \dir [count=\i] in {#1} {
     % Now we have to draw the lines. We start at (0,0) or (0,1).
     \ifthenelse{\dir>0}{
       \draw [semithick, line cap=round] ({\i-1}, 0)--(\i, 1);
     }{
       \draw [semithick, line cap=round] ({\i-1}, 1)--(\i, 0);
     };
     % Draw the frame.
     \draw [semithick, line cap = round] ({\i-1}, 0) rectangle ({\i},1);
   };
 \end{tikzpicture}
}

\newcommand{\gridsmallvert}[1]{
 % Parameters should be +1 or -1, such as \gridsmallhoriz{1,-1,-1}.
 % These are read from bottom to top.
\begin{tikzpicture}[scale=1, anchor=base]
   \def\gridwidth{1};
   \foreach \dir [count=\gridheight] in {#1} {
     % We read through first to determine how wide the grid is.
   };
 \pgftransformxscale{{0.225/\gridwidth}};
\pgftransformyscale{{0.225/\gridheight}};
   \foreach \dir [count=\i] in {#1} {
     % Now we have to draw the lines. We start at (0,0) or (0,1).
     \ifthenelse{\dir>0}{
   \draw [semithick, line cap=round] (0, {\i-1})--(1, \i);
     }{
       \draw [semithick, line cap=round] (0, \i)--(1, {\i-1});
     };
   };
 \end{tikzpicture}
}

\newcommand{\gridverysmallvert}[1]{
 % Parameters should be +1 or -1, such as \gridsmallhoriz{1,-1,-1}.
 % These are read from bottom to top.
\begin{tikzpicture}[scale=1, anchor=base]
   \def\gridwidth{1};
   \foreach \dir [count=\gridheight] in {#1} {
     % We read through first to determine how wide the grid is.
   };
 \pgftransformxscale{{0.175/\gridwidth}};
 \pgftransformyscale{{0.175/\gridheight}};
   \foreach \dir [count=\i] in {#1} {
     % Now we have to draw the lines. We start at (0,0) or (0,1).
     \ifthenelse{\dir>0}{
   \draw [semithick, line cap=round] (0, {\i-1})--(1, \i);
     }{
       \draw [semithick, line cap=round] (0, \i)--(1, {\i-1});
     };
   };
 \end{tikzpicture}
}

\newcommand{\gridsmallhorizfn}[1]{
    % For use in footnotes (where everything has to be smaller, about 80%)
    % Parameters should be +1 or -1, such as \gridsmallhoriz{1,-1,-1}.
    \begin{tikzpicture}[scale=1, anchor=base]
    \def\gridheight{1};
    \foreach \dir [count=\gridwidth] in {#1} {
        % We read through first to determine how wide the grid is.
    };
    \pgftransformxscale{{0.155/\gridwidth}};
    \pgftransformyscale{{0.155/\gridheight}};
    \foreach \dir [count=\i] in {#1} {
        % Now we have to draw the lines. We start at (0,0) or (0,1).
        \ifthenelse{\dir>0}{
            \draw [semithick, line cap=round] ({\i-1}, 0)--(\i, 1);
        }{
            \draw [semithick, line cap=round] ({\i-1}, 1)--(\i, 0);
        };
    };
  \end{tikzpicture}
}

\newcommand{\gridsmalltwobyvert}[2]{
    % Parameters should be +1 or -1, such as \gridsmalltwobyvert{1,-1,-1}.
    % These are read from bottom to top.
    % In this version, we have TWO COLUMNS.
    \begin{tikzpicture}[scale=1, anchor=base]
    \def\gridwidth{2};
    \foreach \dir [count=\gridheight] in {#1} {
        % We read through first to determine how wide the grid is.
    };
    \pgftransformxscale{{0.225/\gridwidth}};
    \pgftransformyscale{{0.225/\gridheight}};
    \foreach \dir [count=\i] in {#1} {
        % First column.
        \ifthenelse{\dir=1 \OR \dir=-1}{
            \ifthenelse{\dir>0}{
                \draw [semithick, line cap=round] (0, {\i-1})--(1, \i);
            }{
                \draw [semithick, line cap=round] (0, \i)--(1, {\i-1});
            };
        }{};
    };
    \foreach \dir [count=\i] in {#2} {
        % Second column.
        \ifthenelse{\dir=1 \OR \dir=-1}{
            \ifthenelse{\dir>0}{
                \draw [semithick, line cap=round] (1, {\i-1})--(2, {\i  });
            }{
                \draw [semithick, line cap=round] (1, {\i  })--(2, {\i-1});
            };
        }{};
    };
    \end{tikzpicture}
}

\newcommand{\gridsmalltwobyvertframe}[2]{
    % Parameters should be +1 or -1, such as \gridsmalltwobyvert{1,-1}{-1,0}.
    % These are read from bottom to top.
    % In this version, we have TWO COLUMNS.
    \begin{tikzpicture}[scale=1, anchor=base]
    \def\gridwidth{2};
    \def\gridheight{2};
    % \foreach \dir [count=\gridheight] in {#1} {
    %     % We read through first to determine how wide the grid is.
    % };
    \pgftransformxscale{{0.25/\gridwidth}};
    \pgftransformyscale{{0.25/\gridheight}};
    \foreach \dir [count=\i] in {#1} {
        % First column.
        \ifthenelse{\dir=1 \OR \dir=-1}{
            \ifthenelse{\dir>0}{
                \draw [semithick, line cap=round] (0, {\i-1})--(1, \i);
            }{
                \draw [semithick, line cap=round] (0, \i)--(1, {\i-1});
            };
        }{};
        \draw [line cap = round] (0, {\i-1}) rectangle (1, {\i});
    };
    \foreach \dir [count=\i] in {#2} {
        % Second column.
        \ifthenelse{\dir=1 \OR \dir=-1}{
            \ifthenelse{\dir>0}{
                \draw [semithick, line cap=round] (1, {\i-1})--(2, {\i  });
            }{
                \draw [semithick, line cap=round] (1, {\i  })--(2, {\i-1});
            };
        }{};
        \draw [line cap = round] (1, {\i-1}) rectangle (2, {\i});
    };
    \end{tikzpicture}
}

\newcommand{\gridsmallthreebyvertframe}[3]{
    % Parameters should be +1 or -1, such as \gridsmallthreebyvert{1,0,-1}{0,-1,1}{-1,1,0}.
    % These are read from bottom to top.
    % In this version, we have THREE COLUMNS.
    \begin{tikzpicture}[scale=1, baseline=1pt]
    \def\gridwidth{2};
    \def\gridheight{2};
    % \foreach \dir [count=\gridheight] in {#1} {
    %     % We read through first to determine how wide the grid is.
    % };
    \pgftransformxscale{{0.25/\gridwidth}};
    \pgftransformyscale{{0.25/\gridheight}};
    \foreach \dir [count=\i] in {#1} {
        % First column.
        \ifthenelse{\dir=1 \OR \dir=-1}{
            \ifthenelse{\dir>0}{
                \draw [semithick, line cap=round] (0, {\i-1})--(1, \i);
            }{
                \draw [semithick, line cap=round] (0, \i)--(1, {\i-1});
            };
        }{};
        \draw [line cap = round] (0, {\i-1}) rectangle (1, {\i});
    };
    \foreach \dir [count=\i] in {#2} {
        % Second column.
        \ifthenelse{\dir=1 \OR \dir=-1}{
            \ifthenelse{\dir>0}{
                \draw [semithick, line cap=round] (1, {\i-1})--(2, {\i  });
            }{
                \draw [semithick, line cap=round] (1, {\i  })--(2, {\i-1});
            };
        }{};
        \draw [line cap = round] (1, {\i-1}) rectangle (2, {\i});
    };
    \foreach \dir [count=\i] in {#3} {
        % Second column.
        \ifthenelse{\dir=1 \OR \dir=-1}{
            \ifthenelse{\dir>0}{
                \draw [semithick, line cap=round] (2, {\i-1})--(3, {\i  });
            }{
                \draw [semithick, line cap=round] (2, {\i  })--(3, {\i-1});
            };
        }{};
        \draw [line cap = round] (2, {\i-1}) rectangle (3, {\i});
    };
    \end{tikzpicture}
}


% \usepackage{tikz}

%%%%%%%%%%%%%%%%%%%%%%%%%%%%%%%%%%%%%%%%%
%%%%%%%%%%%%%%%%% Notes %%%%%%%%%%%%%%%%%
%%%%%%%%%%%%%%%%%%%%%%%%%%%%%%%%%%%%%%%%%

% \oplus=\mathchar"2208
% \bigoplus=\mathchar"134C

%%%%%%%%%%%%%%%%%%%%%%%%%%%%%%%%%%%%%%%%%
%%%%%%%%%%%%%%%% Plusses %%%%%%%%%%%%%%%%
%%%%%%%%%%%%%%%%%%%%%%%%%%%%%%%%%%%%%%%%%

\DeclareRobustCommand{\yesplus}[3]{
	\begin{tikzpicture}[every path/.style={line width = #2mm, line cap = round}, baseline=#3]
		\pgftransformxscale{#1};
		\pgftransformyscale{#1};
		
		\draw (-1, 0) -- (1,0);
		\draw ( 0,-1) -- (0,1);
		
		\draw (0,0) circle (1);
	\end{tikzpicture}
}
\newcommand{\directsum}{{\mathchoice%
	{\hspace{2.75pt}\mathbin{\yesplus{0.112}{0.115}{-2.51}}\hspace{2.75pt}}%
	{\hspace{2.75pt}\mathbin{\yesplus{0.112}{0.115}{-2.51}}\hspace{2.75pt}}%
	{\hspace{0.55pt}\mathbin{\yesplus{0.085}{0.100}{-1.75}}\hspace{0.55pt}}%
	{\hspace{0.55pt}\mathbin{\yesplus{0.070}{0.085}{-1.25}}\hspace{0.55pt}}%
}}
% \DeclareMathOperator{\bigdirectsum}{\scalerel*{\directsum}{\sum}}
\newcommand{\bigdirectsum}{{\mathchoice%
	{\hspace{2.20pt}\mathbin{\yesplus{0.230}{0.300}{-2.50}}\hspace{2.20pt}}%
	{\hspace{2.20pt}\mathbin{\yesplus{0.164}{0.250}{-2.51}}\hspace{2.20pt}}%
	{\hspace{0.70pt}\mathbin{\yesplus{0.120}{0.210}{-1.75}}\hspace{0.70pt}}%
	{\hspace{0.50pt}\mathbin{\yesplus{0.085}{0.170}{-1.25}}\hspace{0.55pt}}%
}}

\DeclareRobustCommand{\noplus}[3]{
	\begin{tikzpicture}[every path/.style={line width = #2mm, line cap = round}, baseline=#3]
		\pgftransformxscale{#1};
		\pgftransformyscale{#1};
		
		\draw (-1, 0) -- (1,0);
		\draw ( 0,-1) -- (0,1);
		\draw (-1,-1) -- (1,1);
		
		\draw (0,0) circle (1);
	\end{tikzpicture}
}
\newcommand{\notdirectsum}{{\mathchoice%
	{\hspace{2.75pt}\mathbin{\noplus{0.112}{0.115}{-2.51}}\hspace{2.75pt}}%
	{\hspace{2.75pt}\mathbin{\noplus{0.112}{0.115}{-2.51}}\hspace{2.75pt}}%
	{\hspace{0.55pt}\mathbin{\noplus{0.085}{0.100}{-1.75}}\hspace{0.55pt}}%
	{\hspace{0.55pt}\mathbin{\noplus{0.070}{0.085}{-1.25}}\hspace{0.55pt}}%
}}

% \DeclareMathOperator{\bignotdirectsum}{\scalerel*{\notdirectsum}{\sum}}
\newcommand{\bignotdirectsum}{{\mathchoice%
	{\hspace{2.20pt}\mathbin{\noplus{0.230}{0.300}{-2.50}}\hspace{2.20pt}}%
	{\hspace{2.20pt}\mathbin{\noplus{0.164}{0.250}{-2.51}}\hspace{2.20pt}}%
	{\hspace{0.70pt}\mathbin{\noplus{0.120}{0.210}{-1.75}}\hspace{0.70pt}}%
	{\hspace{0.50pt}\mathbin{\noplus{0.085}{0.170}{-1.25}}\hspace{0.55pt}}%
}}

%%%%%%%%%%%%%%%%%%%%%%%%%%%%%%%%%%%%%%%%%
%%%%%%%%%%%%%%%% Minuses %%%%%%%%%%%%%%%%
%%%%%%%%%%%%%%%%%%%%%%%%%%%%%%%%%%%%%%%%%
\DeclareRobustCommand{\yesminus}[3]{
	\begin{tikzpicture}[every path/.style={line width = #2mm, line cap = round}, baseline=#3]
		\pgftransformxscale{#1};
		\pgftransformyscale{#1};
		
		\draw (-1, 0) -- (1,0);
		
		\draw (0,0) circle (1);
	\end{tikzpicture}
}
\newcommand{\skewsum}{{\mathchoice%
	{\hspace{2.75pt}\mathbin{\yesminus{0.112}{0.115}{-2.51}}\hspace{2.75pt}}%
	{\hspace{2.75pt}\mathbin{\yesminus{0.112}{0.115}{-2.51}}\hspace{2.75pt}}%
	{\hspace{0.55pt}\mathbin{\yesminus{0.085}{0.100}{-1.75}}\hspace{0.55pt}}%
	{\hspace{0.55pt}\mathbin{\yesminus{0.070}{0.085}{-1.25}}\hspace{0.55pt}}%
}}
% \DeclareMathOperator{\bigskewsum}{\scalerel*{\skewsum}{\sum}}
\newcommand{\bigskewsum}{{\mathchoice%
	{\hspace{2.20pt}\mathbin{\yesminus{0.230}{0.300}{-2.50}}\hspace{2.20pt}}%
	{\hspace{2.20pt}\mathbin{\yesminus{0.164}{0.250}{-2.51}}\hspace{2.20pt}}%
	{\hspace{0.70pt}\mathbin{\yesminus{0.120}{0.210}{-1.75}}\hspace{0.70pt}}%
	{\hspace{0.50pt}\mathbin{\yesminus{0.085}{0.170}{-1.25}}\hspace{0.55pt}}%
}}

%%%%%%%%%%%%%%% No Minuses %%%%%%%%%%%%%%
\DeclareRobustCommand{\nominus}[3]{
	\begin{tikzpicture}[every path/.style={line width = #2mm, line cap = round}, baseline=#3]
		\pgftransformxscale{#1};
		\pgftransformyscale{#1};
		
		\draw (-1, 0) -- (1,0);
		\draw (-1,-1) -- (1,1);
		
		\draw (0,0) circle (1);
	\end{tikzpicture}
}
\newcommand{\notskewsum}{\mathchoice
	{\hspace{0.55pt}\mathbin{\nominus{0.112}{0.115}{-2.51}}\hspace{0.55pt}}%
	{\hspace{0.55pt}\mathbin{\nominus{0.112}{0.115}{-2.51}}\hspace{0.55pt}}%
	{\hspace{0.55pt}\mathbin{\nominus{0.085}{0.100}{-1.75}}\hspace{0.55pt}}%
	{\hspace{0.55pt}\mathbin{\nominus{0.070}{0.085}{-1.25}}\hspace{0.55pt}}
}
% \DeclareMathOperator{\bignotskewsum}{\scalerel*{\notskewsum}{\sum}}
\newcommand{\bignotskewsum}{{\mathchoice%
	{\hspace{2.20pt}\mathbin{\nominus{0.230}{0.300}{-2.50}}\hspace{2.20pt}}%
	{\hspace{2.20pt}\mathbin{\nominus{0.164}{0.250}{-2.51}}\hspace{2.20pt}}%
	{\hspace{0.70pt}\mathbin{\nominus{0.120}{0.210}{-1.75}}\hspace{0.70pt}}%
	{\hspace{0.50pt}\mathbin{\nominus{0.085}{0.170}{-1.25}}\hspace{0.55pt}}%
}}

%%%%%%%%%%%%%%%%%%%%%%%%%%%%%%%%%%%%%%%%%%%%%%%%%%%%%%%%%%%%%%%%%%%%%%%%%%%%%%%%
%%%                       This centers the figures.                          %%%
%%%%%%%%%%%%%%%%%%%%%%%%%%%%%%%%%%%%%%%%%%%%%%%%%%%%%%%%%%%%%%%%%%%%%%%%%%%%%%%%
\makeatletter
\g@addto@macro\@floatboxreset\centering
\makeatother

%%%%%%%%%%%%%%%%%%%%%%%%%%%%%%%%%%%%%%%%%%%%%%%%%%%%%%%%%%%%%%%%%%%%%%%%%%%%%%%%
%%%                     User Configuration commands                          %%%
%%%%%%%%%%%%%%%%%%%%%%%%%%%%%%%%%%%%%%%%%%%%%%%%%%%%%%%%%%%%%%%%%%%%%%%%%%%%%%%%

%% Uncomment the relevant line below if you have tables or figures.
\haveTablestrue   % Uncomment this if you have tables in your thesis.
\haveFigurestrue  % Uncomment this if you have figures in your thesis.
%\haveObjectstrue % Uncomment this if you have Objects in your thesis. This is almost certainly not the case however.

%%%%%%%%%%%%%%%%%%%%%%%%%%%%%%%%%%%%%%%%%%%%%%%%%%%%%%%%%%%%%%%%%%%%%%%%%%%%%%%%
%%% Below are the commands to set the degree type, department, graduation time, and chair. 
%     Most of these are self explanatory. 
%     Note: The \chair command takes an optional argument for a cochair. 
%       So if John was your chair and Jacob was a cochair, you would use \chair[Jacob]{John}.
%       If John was your chair and you had no cochair, you can simply use \chair{John}.
%%%%%%%%%%%%%%%%%%%%%%%%%%%%%%%%%%%%%%%%%%%%%%%%%%%%%%%%%%%%%%%%%%%%%%%%%%%%%%%%

\title{Universal Combinatorial Structures}%  Put your title here.

\degreeType{Doctor of Philosophy} % Official name of your degree; e.g. "Doctor of Philosophy".
\major{Mathematics}                 % Your official Department
\author{Michael Engen}              % Your Name
\thesisType{Dissertation}           % Dissertation (PhD) or Thesis (Masters)
\degreeYear{2021}                   % Intended graduation year (not the year you submit the thesis)
\degreeMonth{August}                % Month of graduation should be May, August, or December.
\chair{Vincent Vatter}              % Chair and Cochair (see comment block above).

%%%%%%%%%%%%%%%%%%%%%%%%%%%%%%%%%%%%%%%%%%%%%%%%%%%%%%%%%%%%%%%%%%%%%%%%%%%%%%%%
%%% For each of the following, type in the name of the file that contains each section. 
%       They are assumed to be tex files, but if they aren't the command takes an optional argument for the extension.
%       So, you could load dedication.tex as your dedication file using \setDedicationFile{dedication}
%       You could load dedication.txt instead with \setDedicationFile[txt]{dedication}.
%       NOTE: For some compilers they may or may not add a .tex to the end of the file automatically.
%           If you get a "couldn't find dedication.tex.tex" type error, try the command with an empty optional argument,
%           e.g. \setDedicationFile[]{dedication}
%%%
%%%%%%%%%%%%%%%%%%%%%%%%%%%%%%%%%%%%%%%%%%%%%%%%%%%%%%%%%%%%%%%%%%%%%%%%%%%%%%%%

%%% These are REQUIRED sections; easiest to do via these commands.

\setDedicationFile{dedicationFile}            % Dedication Page
\setAcknowledgementsFile{acknowledgementsFile}% Acknowledgements Page
\setAbstractFile{abstractFile}                % Abstract Page (This should only include the abstract itself)
\setReferenceFile{referenceFile}{apalike}     % References. 
                                              % The first argument is your bibtex source file
                                              % The second argument is your bibtex style file.
\setBiographicalFile{biographyFile}           % Biography file of the Author (you).

%%% These are NOT required, so only use them if you actually need/have them.

%\setAbbreviationsFile{abbreviations} % Abbreviations Page
\setAppendixFile{appendix-tables}     % Appendix Content; hyperlinking might be weird.
\multipleAppendixtrue                 % Uncomment this if you have more than one appendix, 
%                                     % Comment it if you have only one appendix.


%%%%%%%                     End of File Assignment
%%%%%%%%%%%%%%%%%%%%%%%%%%%%%%%%%%%%%%%%%%%%%%%%%%%%%%%%%%%%%%%%%%%%%%%%%%%%%%%%

\begin{document}
% Here you just need to include/input your actual work. 
%%% The above files (dedication, acknowledgement, titlepage, etc etc) will all be added for you 
%%% using the files you assigned above. 
%%% If you want to input the above files manually you can comment out the \setFILE command above 
%%% and use \input or \include here. Generally you want to use \include to get your pagebreak.
%%% NOTE: If you input manually you will have to do some/all the formatting manually.

\chapter{Introduction}
\label{chap-introduction}

In 1964, Rado~\cite{rado:universal-graph:} introduced the concept of a \emph{universal} graph, which he defined as a countably infinite graph that contained all finite graphs as induced subgraphs. Later that same year, Moon~\cite{moon:on-minimal-n-un:} studied finite graphs that contain all finite graphs on $m$ vertices, a property that he called $m$-universality, initiating a line of research that continues to enjoy scholarly attention. In this work, we provide a brief survey of universality for various combinatorial structures---words, integer partitions, compositions, graphs, and permutations---under induced containment orders.

We say that the structure $\sigma$ is \emph{$m$-universal} if it contains all structures of size $m$ of the same kind. More generally, if $S$ is a family of structures, we say that $\sigma$ is \emph{$S_m$-universal} if it contains every structure in $S$ of size $m$.

%%%%%%%%%%%%%%%%%%%%%%%%%%%%%%%%%%%%%%%%%%%%%%%%%%%%%%%%%%%%%%%%
\section{Words over a Finite Alphabet}
\label{sec-words}
%%%%%%%%%%%%%%%%%%%%%%%%%%%%%%%%%%%%%%%%%%%%%%%%%%%%%%%%%%%%%%%%

As a simple example that will be prove to be a useful reference, consider the collection of words over a $k$-letter alphabet, where we say a word $u$ is contained in the word $v$ if $u$ is a subsequence of of $v$, and the \emph{size} of a word is its number of letters. 
\begin{proposition}
\label{prop-word-universal}
	Let $\W$ denote the collection of words over a $k$-letter alphabet under the subword order. The smallest $\W_m$-universal words have size $km$.
\end{proposition}
\begin{proof}
	To begin, observe that any $\W_m$-universal word must have at least $m$ occurrences of each letter, as for each letter $\ell$, it must contain the word $\ell^m$, ie. the letter $\ell$ repeated $m$ times. Thus any $\W_m$-universal word must have at least $km$ letters.

	To show that this lower bound is tight, we recursively construct a $\W_m$-universal word of size $km$. Let our alphabet be $\{1, 2, \dots, k\}$. If $v$ is a $\W_{m-1}$-universal word, then the concatenation $v 12\cdots k$ is $\W_m$-universal, as the first $m-1$ letters of an abritrary word $u$ of size $m$ embed into $v$, and the final letter embeds into $12\cdots k$. This fact, together with the $\W_0$-universal empty word, shows that the word $(12 \cdots k)^m$ of size $km$ is $\W_m$-universal, completing the proof.
\end{proof}

%%%%%%%%%%%%%%%%%%%%%%%%%%%%%%%%%%%%%%%%%%%%%%%%%%%%%%%%%%%%%%%%
\section{Keeping it in the Family: Proper Universal Structures}
%%%%%%%%%%%%%%%%%%%%%%%%%%%%%%%%%%%%%%%%%%%%%%%%%%%%%%%%%%%%%%%%

Not all kinds of structures admit $m$-universal structures for all $m$. Consider, for example, the family $\C = \{\varepsilon, \worda, \wordb, \worda\worda, \wordb\wordb, \cdots\}$ of constant words over the two-letter alphabet $\{\worda, \wordb\}$. No word in $\C$ contains both $\worda$ and $\wordb$, so no word in $\C$ is $\C_m$-universal for any $m \ge 1$. Nevertheless, one may still seek universal words for $\C$ in the collection of all words over $\{\worda, \wordb\}$.

A set of structures forms a \emph{class} if it is closed downward under the induced-substructure order, and a class is called \emph{proper} if it does not contain every structure of the given kind. Given a family of structures $\F$, we say that a structure is \emph{$\F_m$-universal} or \emph{$m$-universal for $\F$} if it contains every structure in $\F_m$. If an $\F_m$-universal structure itself lies in $\F$, we say it is a \emph{proper} $\F_m$-universal structure. The classes that admit proper universal objects of all sizes are precisely those that satisfy the joint-embedding property, where we say that a set $\S$ of structures satisfies the \emph{joint-embedding property} if for any $\rho, \sigma \in \S$, there is some $\tau \in \S$ so that $\rho \le \tau$ and $\sigma \le \tau$.

\chapter{Integer Partitions}
\label{chap-partitions}

An (integer) partition $p = p(1) \cdots p(n)$ is a weakly decreasing sequence of positive integers, which are called the \emph{parts} of $p$. The \emph{size} of a partition, denoted $\size{p}$, is the sum of its parts. For convenience, we say that for $i > n$ the $i\th$ part of $p$ is $0$ and write $p(i) = 0$. We say that a partition $p = p(1) \cdots p(m)$ is contained in another partition $q = q(1) \cdots q(n)$ if $p(i) \le q(i)$ for all $i$. This partial order on partitions is simply the one of Young's lattice, namely containment of Ferrers diagrams. The \emph{Ferrers diagram} of a partition $p = p(1) \cdots p(m)$ is a visual representation of $p$ consisting of an arrangement of $p(1)$ cells in the first (topmost) row, $p(2)$ cells in the second row, and so on. An example of partition containment displayed through Ferrers diagrams is presented in Figure~\ref{fig-ptn-ferrers}.

\begin{figure}[ht]
\captionsetup{justification=centering}
	\begin{tikzpicture}[scale={1/3}]
		\ferrers{3,1}
		\node at (5.5,-1) {$\le$};

		\begin{scope}[shift={(6,1)}]
			\ferrers{4,2,1,1}
			\ferrersfilled{3,1}
		\end{scope}

		\begin{scope}[shift={(5.5,-4)}]
			\node at (-3,0) {$31$};
			\node at ( 0,0) {$\le$};
			\node at (3.5,0) {$4211$};
		\end{scope}
	\end{tikzpicture}
\caption{An example partition containment presented by way of Ferrers diagrams.}
\label{fig-ptn-ferrers}
\end{figure}

The \emph{conjugate} of a partition $p$ is the partition whose Ferrers diagram is the reflection of the Ferrers diagram of $p$ reflected along the anti-diagonal $y = -x$. An example of a partition and its conjugate is displayed in Figure~\ref{fig-ptn-conjugate}.
\begin{figure}[ht]
\captionsetup{justification=centering,margin=1in}
	\begin{tikzpicture}[scale={1/3}]
		\ferrers{6,3,3,2,1}
		\draw[dashed, gray] (1,0) -- ++(3.5,-3.5);

		\begin{scope}[shift={(9,0.5)}]
			\ferrers{5,4,3,1,1,1}
			\draw[dashed, gray] (1,0) -- ++(3.5,-3.5);
		\end{scope}
	\end{tikzpicture}
\caption{The Ferrers diagram of a partition and its conjugate, both displayed with the anti-diagonal.}
\label{fig-ptn-conjugate}
\end{figure}

Given a finite set of partitions $S$, the unique smallest partition that contains each partition in $S$ is the partition whose $i\th$ part is equal to the largest $i\th$ part among all members of $S$. 
\begin{observation}
\label{obs-ptn-join}
	If $S$ is a finite set of partitions, then the smallest partition $p$ that is $S$-universal has $i\th$ part given by
	\[
		p(i) 
		= 
		\max\{q(i) \st q \in S\}.
	\]
\end{observation}
Thus, to construct the smallest $m$-universal partition for the set of all partitions, one must only observe that the largest $i\th$ part among all partitions of size $m$ is $\floor{m/i}$, meaning the unique smallest $m$-universal partition has $i\th$ part equal to $\floor{m/i}$ for all $i$.

\begin{theorem}
\label{thm-ptn-universal}
	The unique smallest $m$-universal partition has size
	\[
		\phi(m)
		=
		\floor{m/1} + \floor{m/2} + \cdots + \floor{m/m}
		\footnote{The sequence $\phi$ appears as sequence \OEISlink{A006218} in the OEIS~\cite{sloane:the-on-line-enc:}}.
	\]
\end{theorem}
Figure~\ref{fig-ptn-universal} displays the $24$-universal partition of size $\phi(24) = 84$. In~\cite{dirichlet:uber-die:}, Dirichlet shows that $\phi(m) = m(\log(m) + 2\gamma-1) + \Delta(m)$, where $\gamma \approx 0.5772$ is the Euler--Mascheroni constant and $\Delta(m) = \oO{\sqrt{m}}$, and thus $\phi(m) \sim m \log m$. The asymptotics of $\Delta(m)$ are the subject of the \emph{Dirichlet Divisor Problem}, which aims to find the smallest value $\theta$ so that $\Delta(m) = \oO{m^\theta}$. The best known bound is due to Huxley~\cite{huxley:exponential-sums-and:}, who showed that $\inf \theta \le 131/416 \approx 0.3149$. 

\begin{figure}[ht]
\captionsetup{justification=centering}
	\begin{tikzpicture}[scale={24/120}]
		\ferrers{24,12,8,6,4,4,3,3,2,2,2,2,1,1,1,1,1,1,1,1,1,1,1,1}
	\end{tikzpicture}
\caption{The unique $24$-universal partition of size $\phi(24) = 84$.}
\label{fig-ptn-universal}
\end{figure}

%%%%%%%%%%%%%%%%%%%%%%%%%%%%%%%%%%%%%%%%%%%%%%%%%%%%%%%%%%%%%%%%
\section{Proper Classes of Partitions}
\label{sec-ptn-classes}
%%%%%%%%%%%%%%%%%%%%%%%%%%%%%%%%%%%%%%%%%%%%%%%%%%%%%%%%%%%%%%%%

In this section, we show that any proper class of partitions admits universal partitions of linear size. For reference, some known or computed values of minimum sizes for universal partitions for various classes are presented in Appendix~\ref{appendix-partitions}.

We require one more definition to discuss proper classes of partitions. We say that an \emph{potentially infinite partition} $p$ is an infinite weakly decreasing sequence over the infinite ordered alphabet $\{0 < 1 < \cdots < \omega\}$, where $\omega$ represents an infinite part. If $p(n) = \ell > 0$ for each $n > N$, then we abbreviate $p$ as $p(1) \cdots p(N) \ell^\omega$, where $\ell^\omega$ representing an infinite number of parts equal to $\ell$. If $p(n) = 0$ for each $n > N$, then we abbreviate $p$ as $p(1) \cdots p(N)$. Infinite partitions give way to infinite Ferrers diagrams, and we say that the \emph{age} of a potentially infinite partition $p$, denoted $\Age(p)$, is the set of finite partitions whose Ferrers diagrams embed into the Ferrers diagram of $p$ (the term \emph{age} dates to Fra{\"i}ss{\'e}~\cite{fraisse:sur-lextension-:}). An example of containment of a finite Ferrers diagram into an infinite one, together with the corresponding set containment, is presented in Figure~\ref{fig-ptn-age}.

\begin{figure}[ht]
\captionsetup{justification=centering}
	\begin{tikzpicture}[scale={1/3}]
		\ferrers{4,2,1,1}
		
		\node at (6.5,-2) {$\le$};
		\begin{scope}[shift={(7.5,0.5)}]
			\foreach \d in {0.25, 0.50, 0.75} {
				\filldraw[black] ({6+\d},-0.5) circle [radius=0.05cm];
				\filldraw[black] ({6+\d},-1.5) circle [radius=0.05cm];
				\filldraw[black] (1.5,-{6+\d}) circle [radius=0.05cm];
				% \filldraw[black] (2.5,-{6+\d}) circle [radius=0.05cm];
			}
			\ferrers{5,5,4,1,1}
			\ferrersfilled{4,2,1,1}
		\end{scope}
		
		\begin{scope}[shift={(6.5,-6.5)}]
			\node at (-3.5,0) {$4211$};
			\node at ( 0  ,0) {$\in$};
			\node at ( 5  ,0) {$\Age(\omega \omega 4 1^\omega)$};
		\end{scope}
	\end{tikzpicture}
\caption{An example of containment of a finite Ferrers digram into an infinite one.}
\label{fig-ptn-age}
\end{figure}

We show that every proper class of partitions admits linear-size universal partitions in two parts. First, we show that every proper class of partitions is contained in an age of the form $\Age(\omega^k \ell^\omega)$ in Proposition~\ref{prop-ptn-age-containment}, and then that every such age admits linear-size universal partitions in Theorem~\ref{thm-ptn-universal-proper}.
\begin{proposition}
\label{prop-ptn-age-containment}
	For any proper partition class $\P$, there are nonnegative integers $k, \ell$ such that $\P \subseteq \Age(\omega^k \ell^\omega)$.
\end{proposition}
\begin{proof}
	Let $\ell$ be the largest integer such that $\P$ contains the partition $\ell^n$ for all $n$ and let $k$ be the largest integer such that $\P$ contains $(\ell+1)^k$. We claim that $\P \subseteq \Age(\omega^k \ell^\omega)$. For any partition $p \in \P$, we must have $p(n) \le \ell$ for each $n > k$, as otherwise $p$ contains $(\ell+1)^n$, and as $\P$ is closed downward, we would have $(\ell+1)^n \in \P$, a contradiction. The condition that $p(n) \le \ell$ for each $n > k$ is precisely the condition that defines containment in $\Age(\omega^k \ell^\omega)$, and thus $p \in \Age(\omega^k \ell^\omega)$, completing the proof.
\end{proof}

\begin{theorem}
\label{thm-ptn-universal-proper}
	Let $\P = \Age(\omega^k \ell^\omega)$ be a class of partitions. Then for $m \ge k\ell$, the smallest $\P_m$-universal partitions have size
	\[
		\u_{\P}(m)
		= 
		\sum_{i=1}^{k} \floor{\frac{m}{i}} + \sum_{i=1}^{\ell} \floor{\frac{m}{i}} - k\ell
	\]
	and thus $\u_{\P}(m) = \oTheta{m}$.
\end{theorem}

Before proving Theorem~\ref{thm-ptn-universal-proper}, observe that if $P \subseteq Q$ are sets of partitions and $q$ is a $Q$-universal partition, then $q$ is $P$-universal as well, as $q$ necessarily contains each partition in $P$. Thus, by proving that each age of the form $\Age(\omega^k \ell^\omega)$ admits universal partitions of linear size, Proposition~\ref{prop-ptn-age-containment} implies that every proper partition class does as well.

\newenvironment{proof-of-thm-ptn-universal-proper}{%
	\medskip\noindent {\it Proof of Theorem~\ref{thm-ptn-universal-proper}.\/}%
}{%
	\qed\bigskip%
}
\begin{proof-of-thm-ptn-universal-proper}
	By Observation~\ref{obs-ptn-join}, the unique smallest $\P_m$-universal partition $P$ has its $i\th$ part equal to the largest $i\th$ part among all partitions in $\P_m$. The only restriction on partitions in $\P$ is that their $i\th$ part is at most $\ell$ for $i > k$, so we have
	\[
		P(i)
		=
		\begin{cases}
			\floor{\dfrac{m}{i}}                          & \text{if $1 \le i \le k$,} \\
			\min\left\{\floor{\dfrac{m}{i}}, \ell\right\} & \text{if $k  <  i \le m$.}
		\end{cases}
	\]
	As $\floor{\dfrac{m}{i}} \ge \ell$ if and only if $\floor{\dfrac{m}{\ell}} \ge i$, this may be simplified to
	\[
		P(i)
		=
		\begin{cases}
			\floor{\dfrac{m}{i}} & \text{if $1 \le i \le k$,} \\
			\ell                 & \text{if $k  <  i \le \floor{\dfrac{m}{\ell}}$,} \\
			\floor{\dfrac{m}{i}} & \text{if $\floor{\dfrac{m}{\ell}} < i \le m$.}
		\end{cases}
	\]
	Thus, for $m \ge k\ell$, the size of $P$ is
	\[
		\size{P} 
		= 
		\underbrace{\floor{\dfrac{m}{1}} + \floor{\dfrac{m}{2}} + \cdots + \floor{\dfrac{m}{k}} + \ell + \ell + \cdots + \ell}_{\text{$\floor{\dfrac{m}{\ell}}$ terms}} + \floor{\dfrac{m}{\floor{\tfrac{m}{\ell}}+1}} + \cdots + \floor{\dfrac{m}{m}}.
	\]

	The sum of the parts $P(k+1), \cdots, P(m)$ is equal to $\sum_{i = 1}^{\ell} \floor{\frac{m}{i}} - k\ell$, as these correspond to the first $\ell$ parts of the conjugate of $P$, which are $\floor{\dfrac{m}{i}}$ for $1 \le i \le \ell$, and we need to subtract those cells that overlap with the first $k$ parts of $P$. As $m \ge k\ell$, we have both $\floor{\frac{m}{k}} \ge \ell$ and $\floor{\frac{m}{\ell}} \ge k$, and thus
	\[
		\size{P}
		= 
		\sum_{i=1}^{k} \floor{\frac{m}{i}} + \sum_{i=1}^{\ell} \floor{\frac{m}{i}} - k\ell.
	\]
	For all $n$ and $j$, the inequality
	\[
		\log(j+1)n-j
		\le
		\sum_{i = 1}^{j} \floor{\frac{n}{j}} 
		\le 
		\left(\log(j)+1\right)n,
	\]
	holds, and thus $\size{P} = \oTheta{m}$, as desired.
\end{proof-of-thm-ptn-universal-proper}

%%%%%%%%%%%%%%%%%%%%%%%%%%%%%%%%%%%%%%%%%%%%%%%%%%%%%%%%%%%%%%%%
\section{Concluding Remarks}
\label{sec-ptn-conclusion}
%%%%%%%%%%%%%%%%%%%%%%%%%%%%%%%%%%%%%%%%%%%%%%%%%%%%%%%%%%%%%%%%

Observation~\ref{obs-ptn-join} establishes that the smallest $m$-universal partition is unique, and Theorem~\ref{thm-ptn-universal} establishes its precise size for all $m$. For any proper class of partitions $\P$, there are proper $m$-universal partitions for $\P$ for all $m$ if and only if $\P$ satisfies the joint-embedding property, which is equivalent to $\P$ being the age of some potentially infinite partition $u$, as the following result of Fra{\"i}ss{\'e} (which we have specialized to our context here) shows:
\begin{theorem}[Fra{\"i}ss{\'e}~\cite{fraisse:sur-lextension-:}; see also Hodges~{\cite[Section 7.1]{hodges:model-theory:}}] 
	The following are equivalent for a class $\P$ of integer partitions or compositions:
	\begin{enumerate}
		\item $\P$ cannot be expressed as the union of two proper subclasses,
		\item $\P$ satisfies the \emph{joint embedding property}, meaning that for every $a, b \in \P$ there is some $c \in \P$ such that $a, b \le c$, and 
		\item $\P = \Age(u)$ for some word $u \in (\mathbb{P} \cup \{n^\omega \st n \in \mathbb{P}\} \cup \{\omega, \omega^\omega\})^\ast$.
	\end{enumerate}
\end{theorem}


\chapter{Compositions}
\label{chap-compositions}

A composition $c = c(1) \cdots c(n)$ is a finite sequence of positive integers, which are called the \emph{parts} of $c$. The \emph{length} of $c$, denoted $\len(c)$ is the length of the sequence, the \emph{size} of a composition, denoted $\size{c}$, is the sum of its parts. We say a composition $c = c(1) c(2) \cdots c(m)$ is contained in another composition $d = d(1) d(2) \cdots d(n)$ if there is a subsequence $d(i_1) d(i_2) \cdots d(i_m)$ of $d$ such that $c(j) \le d(i_j)$ for all $j$, and such a sequence of indices is called an \emph{embedding} of $c$ into $d$. Note that with this definition, when determining whether $c$ is contained in $d$, it suffices to take a greedy approach: embed the first part of $c$ as far left as possible in $d$, then embed the second part of $c$ as far left as possible after that, and so on. 

This partial order may be exhibited visually by way of skyline diagrams, which are simply a symmetry of the Ferrers diagrams of Chapter~\ref{chap-partitions}. The \emph{skyline diagram} of the composition $c = c(1) \cdots c(n)$ consists of $n$ columns of cells, with the $i\th$ column (from the left) having $c(i)$ cells. For compositions $c$ and $d$, we have $c \le d$ if the skyline diagram of $c$ can be embedded into that of $d$. An example of composition containment displayed through skyline diagrams is shown in Figure~\ref{fig-comp-skyline}.

\begin{figure}[ht]
\captionsetup{justification=centering}
	\begin{tikzpicture}[scale={1/3}]
		\skylinelabeled{3,4,1,3}

		\node at (6.5,3) {$\le$};
		\node at (6.5,0) {$\le$};

		\begin{scope}[shift={(7,0)}]
			\skylinelabeled{1,4,1,4,2,1,1,4,3}
			\skylinefilledunderlines{0,3,0,4,1,0,0,3,0}
		\end{scope}
	\end{tikzpicture}
\caption{An example of composition containment presented by way of skyline diagrams.}
\label{fig-comp-skyline}
\end{figure}

This partial order on compositions is called the \emph{generalized subword order} for words over the positive integers $\mathbb{P}$. It was first considered by Bergeron, Bousquet-M{\'e}lou, and Dulucq~\cite{bergeron:standard-paths-:}, who studied the saturated chains in this poset. Later, Sagan and Vatter~\cite{sagan:the-mobius-func:} determined the M{\"o}bius function of this poset, and Bj{\"o}rner and Sagan~\cite{bjorner:rationality-of-:comp} showed that this M{\"o}bius function has a rational generating function. Finally, Vatter~\cite{vatter:reconstructing-:} considered the analogue of the reconstruction conjecture in this poset, and Engen and Vatter~\cite{engen:on-the-dimensio:} characterized the finite-dimensional classes in this poset with respect to Dushnik--Miller dimension.

The following result shows that smallest universal compositions can be constructed recursively, leading to an explicit formula on their size, and has been adapted from~\cite{albert:universal-layer:}.

\begin{theorem}[Albert, Engen, Pantone, and Vatter~\cite{albert:universal-layer:}]
\label{thm-comp-universal}
	Let $\ell(m)$ be the size of a smallest $m$-universal composition. Then $\ell(0) = 0$ and for $m \ge 1$,
	\begin{equation}
	\label{eqn-composition}
		\ell(m)
		=
		m + \min\{\ell(k) + \ell(m-k-1) \st 0 \le k \le m-1\}
		\footnote{Up to shifting indices by $1$, the sequence $\ell(m)$ in Theorem~\ref{thm-comp-universal} is sequence \OEISlink{A001855} in the OEIS~\cite{sloane:the-on-line-enc:}.}
		.
	\end{equation}
\end{theorem}

\begin{proof}
	We prove the theorem by induction on $m$. As the base case of $m = 0$ is trivial, suppose that the statement is true for all values less than $m$.

	We begin by showing that
	\[
		\ell(m)
		\le
		m + \min\{\ell(k) + \ell(m-k-1) \st 0 \le k \le m-1\}
	\]
	by constructing an $m$-universal composition of this size. To this end, choose $k$ so that $\ell(k) + \ell(m-k-1)$ is minimized and then choose a composition $c$ of size $\ell(k)$ that is $k$-universal and a composition $d$ of size $\ell(m-k-1)$ that is $(m-k-1)$-universal. We claim that the composition $c m d$ is $m$-universal. 

	Consider any composition $e = e(1) e(2) \cdots e(n)$ of size $m$, and choose $j$ such that
	\[
		\begin{array}{lccclclcl}
			e(1) &+& \cdots &+& e(j-1) & &      &\le& k,\\
			e(1) &+& \cdots &+& e(j-1) &+& e(j) & > & k.
		\end{array}
	\]
	Since $c$ is $k$-universal and $\size{e(1) \cdots e(j-1)} \le k$, we have that $c$ contains the composition $e(1) \cdots e(j-1)$. Moreover, the part $e(j)$ embeds into the part $m$. Finally, since $d$ is $(m-k-1)$-universal, it necessarily contains $e(j+1) \cdots e(n)$ as $\size{e(j+1) \cdots e(n)} \le m - k -1$. Thus $e(1) \cdots e(n)$ is contained in the composition $c m d$, and as $e$ was an arbitrary composition of size $m$, we have that $c m d$ is $m$-universal.

	To establish the reverse inequality 
	\[
		\ell(m)
		\ge
		m + \min\{\ell(k) + \ell(m-k-1) \st 0 \le k \le m-1\},
	\]
	we prove that every $m$-universal composition must have size at least $m + \ell(k) + \ell(m-k-1)$ for some $0 \le k \le m-1$. Let $c$ be an $m$-universal composition. As $c$ contains the composition $m$, we may choose some index $j$ so that $c(j) \ge m$. Choose $k$ so that
	\[
		\ell(k) \le c(1) + \cdots + c(j-1) < \ell(k+1),
	\]
	meaning that there is some composition $d$ of size $k+1$ that does not embed into $c(1) \cdots c(j-1)$, and thus the earliest that $d$ may embed into $c$ is into $c(1) \cdots c(j)$. 

	We claim that $c(j+1) \cdots c(n)$ must be $(m-k-1)$-universal. Let $e$ be an arbitrary composition of size $m-k-1$. As $c$ is $m$-universal, it must contain the composition $de$. Since $d$ can embed no earlier than into $c(1) \cdots c(j)$, $e$ must embed entirely within $c(j+1) \cdots c(n)$. As $e$ was an arbitrary composition of size $m-k-1$, the composition $c(j+1) \cdots c(n)$ is therefore $(m-k-1)$-universal, and thus $c(j+1) + \cdots + c(n) \ge \ell(m-k-1)$. Together, this shows
	\begin{align*}
		\size{c}
			&= \size{c(1) \cdots c(j-1)} + \size{c(j)} + \size{c(j+1) \cdots c(n)} \\
			&\ge \ell(k) + m + \ell(m-k-1),
	\end{align*}
	completing the proof.
\end{proof}

By showing that $\ell$ is a convex sequence, we may obtain an explicit formula for $\ell(m)$. In~\cite{morris:some-theorems-on:}, Morris proves that the sequence defined by $G(1) = 0$ and, for $m > 1$ 
\[
	G(m) 
	= 
	n + \min\{G(k) + G(m-k) \st 1 \le k \le m-1\}
\]
is convex, and it is elementary to show that $G(m+1) = \ell(m) + m$. For completeness, we present a proof of the convexity of $\ell$, adapted from Morris's proof of the convexity of $G$.

\begin{proposition}
\label{prop-ell-convex}
	The sequence $\ell(m)$ is convex, ie. $2\ell(m) \le \ell(m-1) + \ell(m+1)$ for all $m \ge 1$.
\end{proposition}
\begin{proof}
	We prove that $\ell$ is convex at $m$, ie. $2\ell(m) \le \ell(m-1) + \ell(m+1)$ by induction on $m$. Our base case of $m = 1$ follows from computing $\ell(0) = 0$, $\ell(1) = 1$, and $\ell(2) = 3$, so assume that the statement holds for all values less than $m$. We proceed in two cases based on the parity of $m$.

	First, assume that $m$ is even, with $m = 2n$. We may compute the formulas for $\ell(m-1)$, $\ell(m)$, and $\ell(m+1)$ using that convexity of $\ell$ at values $m' < m$ implies that the minimum of $\ell(k) + \ell(m'-k)$ over all values of $k$ occurs when $k = \floor{m'/2}$.
	\begin{align*}
		\ell(2n-1)
			&= 2n-1 + \min\{\ell(k) + \ell(2n-2-k) \st 0 \le k \le 2n-2\} \\
			&= 2n-1 + \ell(n-1) + \ell(n-1) \\
		\ell(2n)
			&= 2n   + \min\{\ell(k) + \ell(2n-1-k) \st 0 \le k \le 2n-1\} \\
			&= 2n   + \ell(n-1) + \ell(n) \\
		\ell(2n+1)
			&= 2n+1 + \min\{\ell(k) + \ell(2n  -k) \st 0 \le k \le 2n  \} \\
			&= 2n+1 + \ell(n) + \ell(n)
	\end{align*}
	These formulas show that in this case, we in fact have equality: $2\ell(2n) = \ell(2n-1) + \ell(2n+1)$.

	On the other hand, assume that $m$ is odd, with $m = 2n+1$. As before, the convexity of $\ell$ at values less than $m$ allows us to write explicit recursive formulas for $\ell(2n)$, $\ell(2n+1)$, and $\ell(2n+2)$.
	\begin{align*}
		\ell(2n)
			&= 2n   + \min\{\ell(k) + \ell(2n-1-k) \st 0 \le k \le 2n-1\} \\
			&= 2n   + \ell(n-1) + \ell(n) \\
		\ell(2n+1)
			&= 2n+1 + \min\{\ell(k) + \ell(2n  -k) \st 0 \le k \le 2n  \} \\
			&= 2n+1 + \ell(n) + \ell(n) \\
		\ell(2n+2)
			&= 2n+2 + \min\{\ell(k) + \ell(2n+1-k) \st 0 \le k \le 2n+1\} \\
			&= 2n+2 + \ell(n) + \ell(n+1)
	\end{align*}
	Some arithmetic reveals that the inequality $2 \ell(2n+1) \le \ell(2n) + \ell(2n+2)$ is equivalent to the inequality $2\ell(n) \le \ell(n-1) + \ell(n+1)$, which is true by induction, completing the proof.
\end{proof}

Proposition~\ref{prop-ell-convex} allows us to recursively define an explicit sequence of smallest $m$-universal compositions. Let $w_0$ be the empty composition, and for $m \ge 1$, define $w_m$ as the concatenation $w_{\floor{(m-1)/2}} m w_{\ceil{(m-1)/2}}$. Then, as $\ell(k) + \ell(m-1-k)$ is minimized at $k = \floor{(m-1)/2}$, $w_m$ is $m$-universal by the proof of Theorem~\ref{thm-comp-universal} and has length $m$ and size $\ell(m)$. Figure~\ref{fig-comp-universal} shows the skyline diagram of $w_{63}$.

Knuth gave an explicit formula for $\ell(m)$ in \emph{The Art of Computer Programming, Volume 3}~\cite[Section 5.3.1, Eq. (3)]{knuth:the-art-of-comp:3}, where it is related to sorting by binary insertion. Knuth shows there that,
\[
	\ell(m) 
	= 
	(m+1)\ceil{\log_2 (m+1)} - 2^{\ceil{\log_2 (m+1)}} + 1,
\]
so $\ell(m) \sim m \log_2(m)$.

\begin{figure}[ht]
\captionsetup{justification=centering}
	\begin{tikzpicture}[scale=0.1]
		% from functools import lru_cache
		%
		% @lru_cache
		% def w(n):
		%     if n == 0: return []
		%     return w((n-1)//2) + [n] + w(n//2)
		%
		% inner = ','.join(str(val) for val in w(63))
		%
		% print(r'\skyline{' + inner + r'}')
		
		% \skyline{1,3,1,7,1,3,1,15,1,3,1,7,1,3,1,31,1,3,1,7,1,3,1,15,1,3,1,7,1,3,1} % 31
		\skyline{1,3,1,7,1,3,1,15,1,3,1,7,1,3,1,31,1,3,1,7,1,3,1,15,1,3,1,7,1,3,1,63,1,3,1,7,1,3,1,15,1,3,1,7,1,3,1,31,1,3,1,7,1,3,1,15,1,3,1,7,1,3,1} % 63
	\end{tikzpicture}
\caption{The $63$-universal composition $w_{63}$ of length $63$ and size $\ell(63) = 321$.}
\label{fig-comp-universal}
\end{figure}

%%%%%%%%%%%%%%%%%%%%%%%%%%%%%%%%%%%%%%%%%%%%%%%%%%%%%%%%%%%%%%%%
\section{Proper Classes of Compositions}
\label{sec-comp-classes}
%%%%%%%%%%%%%%%%%%%%%%%%%%%%%%%%%%%%%%%%%%%%%%%%%%%%%%%%%%%%%%%%

In this section, we examine the problem of universality for proper classes of compositions. As we will see, every proper composition class admits universal compositions of linear size. Moreover, we study universality of two natural families of compositions as well as for the class of compositions with at most one large part. For reference, some known or computed values of minimum sizes for universal compositions for various classes are presented in Appendix~\ref{appendix-compositions}. 

As in Chapter~\ref{chap-partitions}, we define infinite structures to aid discussion of proper classes. We define a \emph{potentially infinite} composition to be a word over the alphabet $\mathbb{P} \cup \{n^\omega \st n \in \mathbb{P}\} \cup \{\omega, \omega^\omega\}$. As in the case of partitions, the symbol $\omega$ stands for an infinite part, the symbol $n^\omega$ stands for an infinite number of parts equal to $n$, and the symbol $\omega^\omega$ stands for an infinite number of infinite parts. Given a potentially infinite composition $c$, the \emph{age} of $c$, denoted $\Age(c)$, is the set of all compositions whose skyline diagrams embed into the skyline diagram of $c$ (the term \emph{age} dates to Fra{\"i}ss{\'e}~\cite{fraisse:sur-lextension-:}.) An example of containment of a finite skyline diagram in an infinite one, together with the corresponding set containment, is presented in Figure~\ref{fig-comp-age}.

\begin{figure}[ht]
\captionsetup{justification=centering}
	\begin{tikzpicture}[scale={1/2}]
		\skylinelabeled{1,3,2,3,1}

		\node at ( 7,3) {$\le$};
		\node at ( 7,0) {$\in$};
		\node at (14,0) {$\Age(1^\omega \omega 2 1 3 1^\omega)$};

		\begin{scope}[shift={(8,0)}]
			\skyline{1,1,1,4,2,1,3,1,1,1}
			\foreach \d in {0.25, 0.50, 0.75} {
				\filldraw[black] (     \d,   1.5) circle [radius=0.05cm];
				\filldraw[black] ({11+\d},   1.5) circle [radius=0.05cm];
				\filldraw[black] (    4.5,{5+\d}) circle [radius=0.05cm];
			}
			\skylinefilled{0,0,1,3,2,0,3,1}
		\end{scope}
	\end{tikzpicture}
\caption{Containment of a finite skyline diagram into an infinite one.}
\label{fig-comp-age}
\end{figure}

The following proposition shows that any proper composition class admits universal compositions of linear size.

\begin{theorem}
\label{thm-comp-universal-proper}
	For any proper class of compositions $\C$, there is a positive constant $c$ so that there are $\C_m$-universal compositions of size $cm$ for all positive integers $m$.
\end{theorem}
\begin{proof}
	We show that $\C$ admits linear-size universal compositions by first showing that $\C$ is contained in an age of the form $\Age(\ell^\omega (\omega \ell^\omega)^k)$, which in turn admits linear-size universal compositions.

	Let $\ell$ be the largest integer such that $\C$ contains $\ell^n$ for all $n$, and let $k$ be the largest integer such that $\C$ contains $(\ell + 1)^k$. We claim that $\C \subseteq \Age(\ell^\omega (\omega \ell^\omega)^k)$. If $c \in \C$, then $c$ does not contain $(\ell + 1)^k$, and thus $c$ has at most $k$ parts of size larger than $\ell$. The class $\Age(\ell^\omega (\omega \ell^\omega)^k)$ is precisely the class of compositions that have at most $k$ parts larger than $\ell$, and thus we have $c \in \Age(\ell^\omega (\omega \ell^\omega)^k)$.

	To construct an $m$-universal composition for $\Age(\ell^\omega (\omega \ell^\omega)^k)$, note that if $c$ and $d$ are $m$-universal compositions for $\Age(u)$ and $\Age(v)$ respectively, then $cd$ is $m$-universal for $\Age(uv)$. Thus, as the composition $\ell^m$ is $m$-universal for $\Age(\ell^\omega)$ and the composition $m$ is $m$-universal for $\Age(\omega)$, we may conclude that $\ell^m (m \ell^m)^k$ is $m$-universal for $\Age(\ell^\omega (\omega \ell^\omega)^k)$.
\end{proof}

%%%%%%%%%%%%%%%%%%%%%%%%%%%%%%%%
\subsection{Compositions with Bounded Part Sizes}
\label{subsec-comp-bdd-pt-size}
%%%%%%%%%%%%%%%%%%%%%%%%%%%%%%%%

Fix a positive integer $t$, and consider the class $\Age(t^\omega)$ of compositions whose parts are all of size at most $t$. In a composition $c$, each part of size larger than $t$ may be replaced one of size $t$ without disturbing any of the embeddings of compositions in $\Age(t^\omega)$. 
\begin{observation}
	Let $c$ be a composition of size $n$. For each positive integer $t$, there is composition $c'$ of size at most $n$ in $\Age(t^\omega)$ that contains every composition in $\Age(t^\omega)$ that $c$ does.
\end{observation}

In particular, if $c$ is any $\Age(t^\omega)_m$-universal composition, then there is a weakly smaller proper $\Age(t^\omega)_m$-universal composition, and thus the smallest $\Age(t^\omega)_m$-universal compositions and the smallest proper $\Age(t^\omega)_m$-universal compositions have the same size.

Smallest proper universal compositions for $\Age(t^\omega)$ can be constructed in a manner similar that of the proof of Theorem~\ref{thm-comp-universal}. The construction only differs in that the ``large'' part of size $m$ used in the proof of Theorem~\ref{thm-comp-universal} needs to only have size $\min\{t, m\}$ in this case.
\begin{theorem}
\label{thm-comp-bdd-part-size}
	Let $\ell_t(m)$ denote the size of the smallest (proper) $\Age(t^\omega)_m$-universal compositions. Then $\ell_t(0) = 0$ and for $m \ge 1$,
	\[
		\ell_t(m)
		=
		\min\{t,m\} + \min\{\ell_t(k) + \ell_t(m-k-1) \st 0 \le k \le m-1\}.
	\]
\end{theorem}
\begin{proof}
	We prove the theorem by induction on $m$. As the base case of $m = 0$ is trivial, suppose that the statement is true for all values less than $m$. 
	
	We begin by showing that
	\[
		\ell_t(m) 
		\le
		\min\{t, m\} + \min\{\ell_t(k) + \ell_t(m-k-1) \st 0 \le k \le m-1\}.
	\]
	by constructing an $\Age(t^\omega)_m$-universal composition of this size. To this end, choose $k$ so that $\ell_t(k) + \ell_t(m-k-1)$ is minimized and then choose a proper $\Age(t^\omega)_k$-universal composition $c$ of size $\ell_t(k)$ and a proper $\Age(t^\omega)_{m-k-1}$-universal composition $d$ of size $\ell_t(m-k-1)$. We claim that the composition $c (\min\{t, m\}) d \in \Age(t^\omega)$ is $\Age(t^\omega)_m$-universal.

	Consider any composition $e = e(1) e(2) \cdots e(n) \in \Age(t^\omega)$ of size $m$, and choose $j$ such that
	\[
		\begin{array}{lccclclcl}
			e(1) &+& \cdots &+& e(j-1) & &      &\le& k,\\
			e(1) &+& \cdots &+& e(j-1) &+& e(j) & > & k.
		\end{array}
	\]
	Since $c$ is $\Age(t^\omega)_k$-universal and $\size{e(1) \cdots e(j-1)} \le k$, we have that $c$ contains the composition $e(1) \cdots e(j-1)$. Moreover, the part $e(j)$ embeds in the part $\min\{t, m\}$. Finally, since $d$ is $\Age(t^\omega)_{m-k-1}$-universal, it contains the composition $e(j+1) \cdots e(n)$ as $\size{e(j+1)\cdots e(n)} \le m-k-1$. Thus $e(1) \cdots e(n)$ is contained in the composition $c (\min\{t, m\}) d$, and as $e$ was an arbitrary composition of size $m$, we have that $c (\min\{t, m\}) d$ is $\Age(t^\omega)_m$-universal.

	To establish the reverse inequality
	\[
		\ell_t(m)
		\ge
		\min\{t, m\} + \min\{\ell_t(k) + \ell_t(m-k-1) \st 0 \le k \le m-1\},
	\]
	we prove that every proper $\Age(t^\omega)_m$-universal composition must have size at least $\min\{t, m\} + \ell_t(k) + \ell_t(m-k-1)$ for some $k$. Let $c$ be an $\Age(t^\omega)_m$-universal composition. As $c$ contains the composition $\min\{t, m\}$, so there must be some index $j$ so that $c(j) \ge \min\{t, m\}$. Choose $k$ so that
	\[
		\ell_t(k) 
		\le 
		c(1) + \cdots + c(j-1) 
		< 
		\ell_t(k+1),
	\]
	meaning that there is some composition $d$ of size $k+1$ that does not embed into $c(1) \cdots c(j)$, and thus the earliest $d$ may embed into $c$ is into $c(1) \cdots c(j)$.

	We claim that $c(j+1) \cdots c(n)$ must be $(m-k-1)$-universal for $\Age(t^\omega)$. Let $e$ be an arbitrary composition in $\Age(t^\omega)$ of size $m-k-1$. As $c$ is $\Age(t^\omega)_m$-universal, it must contain the composition $de$. Since $d$ can embed no earlier than into $c(1) \cdots c(j)$, $e$ must embed entirely within $c(j+1) \cdots c(n)$. As $e$ was an abitrary composition of size $m-k-1$, the composition $c(j+1) \cdots c(n)$ is therefore $(m-k-1)$-universal for $\Age(t^\omega)$, and thus $c(j+1) + \cdots + c(n) \ge \ell_t(m-k-1)$. Finally, we have
	\begin{align*}
		\size{c}
			&= \size{c(1) \cdots c(j-1)} + \size{c(j)} + \size{c(j+1) \cdots c(n)} \\
			&\ge \ell_t(k) + \min\{t, m\} + \ell_t(m-k-1),
	\end{align*}
	completing the proof.
\end{proof}

\begin{figure}[ht]
	\captionsetup{justification=centering}
	\begin{tikzpicture}[scale=0.2]
		\skyline{1,3,1,7,1,3,1,7,1,3,1,7,1,3,1,7,1,3,1,7,1,3,1,7,1,3,1,7,1,3,1,7,1,3,1,7,1,3,1,7,1,3,1,7,1,3,1,7,1,3,1,7,1,3,1,7,1,3,1,7,1,3,1}
	\end{tikzpicture}
	\caption{A proper $\Age(7^\omega)_{63}$-universal composition of length $63$ and size $\ell_7(63) = 185$.}
	\label{fig-bdd-part-size}
\end{figure}

Figure~\ref{fig-bdd-part-size} shows a smallest (proper) $\Age(7^\omega)_{63}$-universal composition of the above construction. Unlike the sequence $\ell(m)$, the sequences $\ell_t(m)$ do not appear to be convex. Through tedious casework, the explicit formulas for the sequences $\ell_t(m)$ may be determined for small values of $t$, and computation supports the following conjecture:

\begin{conjecture}
\label{conj-comp-bdd-part-size}
	Let $t = 2^k + r$ with $0 \le r < 2^k$. Then for $m \ge k$ we have
	\[
		\ell_t(m)
		=
		km + k - 2^k + 1 + \sum_{i = 0}^{r-1} \floor{\dfrac{m-i}{2^k}}.
	\]
\end{conjecture}

In particular, if $t = 2^k - 1$, then for $m \ge k$ we conjecture
\[
	\ell_t(m)
	=
	km + k - 2^k + 1.
\]

%%%%%%%%%%%%%%%%%%%%%%%%%%%%%%%%
\subsection{Compositions with a Bounded Number of Parts}
\label{subsec-comp-bdd-num-pts}
%%%%%%%%%%%%%%%%%%%%%%%%%%%%%%%%

Fix a positive integer $t$, and consider the class $\Age(\omega^t)$ of compositions that have at most $t$ parts. For $m \le t$, $\Age(\omega^t)$ contains all compositions of size $m$, and the composition $w_m$ is in $\Age(\omega^t)$, so the smallest proper $\Age(\omega^t)_m$-universal compositions are smallest $m$-universal compositions for $m \le t$. 

\begin{proposition}
\label{thm-comp-bdd-num-parts-small}
	For all $m \le t$, the smallest proper $\Age(\omega^t)_m$-universal compositions have size $\ell(m)$. 
\end{proposition}

Given an $(m-1)$-universal composition $c$ for $\Age(\omega^t)$, we may form an $m$-universal composition for $\Age(\omega^t)$ of the same length by incrementing each of its parts, as the next result shows.
\begin{proposition}
\label{prop-comp-bdd-num-parts-proper}
	For all $m > t$, if $c$ is a $\Age(\omega^t)_{m-1}$-universal composition, then the composition $c^{+1}$ formed by incrementing each part of $c$ is $m$-universal for $\Age(\omega^t)$.
\end{proposition}
Before presenting the proof of Proposition~\ref{prop-comp-bdd-num-parts-proper}, we note that it implies the existence of proper $\Age(\omega^t)_m$-universal compositions of size $t(m-t) + \ell(t)$ by simply incrementing each part of $w_t$ a total of $m-t$ times. For any composition $c$, let $c^{+n}$ denote the composition formed by adding $n$ to each part of $c$. A proper universal composition for $\Age(\omega^7)$ formed in this way is drawn in Figure~\ref{fig-comp-bdd-num-parts}.
\begin{corollary}
	For all $m > t$, $w_t^{+(m-t)}$ is a proper $\Age(\omega^t)_m$-universal of size $tm - t^2 + \ell(t)$.
\label{cor-comp-bdd-num-parts-proper}
\end{corollary}
%
\newenvironment{proof-of-prop-comp-bdd-num-parts-proper}{%
	\medskip\noindent {\it Proof of Proposition~\ref{prop-comp-bdd-num-parts-proper}.\/}%
}{%
	\qed\bigskip%
}
\begin{proof-of-prop-comp-bdd-num-parts-proper}
	Suppose that $m > t$ and that $c$ is a proper $\Age(\omega^t)_{m-1}$-universal composition. Let $c^{+1}$ be the composition formed by incrementing each part of $c$ and let $d = d(1) \cdots d(n)$ be an arbitrary composition in $\Age(\omega^t)$ of size $m$. As $m > t$, there is some part $d(j)$ of $d$ such that $d(j) > 1$. Thus, the composition $d_{j}^{-1}$ formed from $d$ by decrementing its $j\th$ part is a composition of size $m-1$ in $\Age(\omega^t)$, and thus embeds into $c$, ie. there is a subsequence $c(i_1) c(i_2) \cdots c(i_n)$ of $c$ such that $d_{j}^{-1}(k) \le c(i_k)$ for all $k$. Then, as $d(k) \le d_{j}^{-1}(k) + 1$ for all $k$, we have that $d(k) \le c(i_k) + 1$ for all $k$, so $d$ embeds into $c^{+1}$, completing the proof.
\end{proof-of-prop-comp-bdd-num-parts-proper}

\begin{figure}[ht]
\captionsetup{justification=centering}
	\begin{tikzpicture}[scale=0.2]
		\skyline{7, 9, 7, 13, 7, 9, 7}
	\end{tikzpicture}
\caption{The proper $\Age(\omega^7)_{13}$-universal composition $w_7^{+6}$ of length $7$ and size $59$.}
\label{fig-comp-bdd-num-parts}
\end{figure}

In the case that $t = 2$, Corollary~\ref{cor-comp-bdd-num-parts-proper} shows that there are proper $\Age(\omega^2)_m$-universal compositions of size $2m-1$. In fact, the smallest $\Age(\omega^2)_m$-universal compositions of arbitrary length have this size as well:

\begin{proposition}
\label{prop-comp-length-2-improper}
	The smallest (proper) $\Age(\omega^2)_m$-universal compositions have size $2m-1$.
\end{proposition}
\begin{proof}
	Let $c = c(1) \cdots c(\ell)$ be an $\Age(\omega^2)_m$-universal composition for $m \ge 2$. As $c$ must contain the composition $m$, there must be some $j$ such that $c(j) \ge m$. Let $a$ be the size of the largest part before $c(i)$ and $0$ if $i = 1$. Similarly, let $b$ be the size of the largest part after $c(i)$ and $0$ if $i = \ell$. There are now two cases to consider: if $a < m-1$ and if $a \ge m-1$. If $a < m-1$, then $c$ contains the composition $(a+1)(m-a-1)$. As $a$ is the largest part to the left of $c(i)$, we must have $b \ge m-a-1$. Thus
	\begin{align*}
		\size{c}
			&\ge a + m + b \\
			&\ge a + m + (m-a-1) \\
			&= 2m-1.
	\end{align*}
	On the other hand, if $a \ge m-1$, then we have
	\begin{align*}
		\size{c}
			&\ge   a   + m + b \\
			&\ge (m-1) + m + 0 \\
			&= 2m-1,
	\end{align*}
	as well, completing the proof.
\end{proof}

% Not every smallest $\Age(\omega^2)_m$-universal composition lies in $\Age(\omega^2)$. The composition $\floor{\frac{m-1}{2}} m \ceil{\frac{m-1}{2}}$ of size $2m-1$ may be shown to be $\Age(\omega^2)_m$-universal but does not lie in $\Age(\omega^2)$.

Computational evidence indicates that each smallest $m$-universal composition for $\Age(\omega^t)$ for $m > t$ can be constructed with Proposition~\ref{prop-comp-bdd-num-parts-proper}, and thus we make the following conjecture.
\begin{conjecture}
\label{conj-comp-bdd-num-parts-proper}
	For $m > t$, the smallest proper $m$-universal compositions for $\Age(\omega^t)$ have size $tm - t^2 + \ell(t)$.
\end{conjecture}

In the case that $t = 3$, Corollary~\ref{cor-comp-bdd-num-parts-proper} shows that there are proper $\Age(\omega^3)_m$-universal compositions of size $3m-4$, and empirical evidence suggests---and we conjecture---that the smallest $\Age(\omega^3)_m$-universal compositions of arbitrary length have this size as well. However, for $t \ge 4$, computations reveal that the smallest $\Age(\omega^t)_m$-universal compositions are strictly smaller than the proper $\Age(\omega^t)_m$-universal composition presented in Corollary~\ref{cor-comp-bdd-num-parts-proper}, and thus the size of smallest $\Age(\omega^t)_m$-universal compositions for $t > 4$ is unknown.

\begin{conjecture}
\label{conj-comp-length-3-improper}
	For all $m \ge 3$, there are smallest $\Age(\omega^3)_m$-universal compositions of length $3$.
\end{conjecture}

%%%%%%%%%%%%%%%%%%%%%%%%%%%%%%%%
\subsection{Compositions with at Most One Large Part}
\label{subsec-comp-one-large-part}
%%%%%%%%%%%%%%%%%%%%%%%%%%%%%%%%

As one final case study, consider the class $\C = \Age(1^\omega \omega 1^\omega)$ of compositions with at most one part of size greater than $1$.

\begin{theorem}
	For $m \ge 3$, the smallest (proper) $\C_m$-universal compositions have size $3m-4$.
\end{theorem}
\begin{proof}
	To begin, we claim that the composition $c = 1^{m-2} m 1^{m-2}$ is a proper $\C_m$-universal composition. Let $d$ be an arbitrary composition in $\C_m$. If $d$ has a part that is at least $2$, then it must decompose as $1^i j 1^k$, where $2 \le j \le m$ and both $i$ and $k$ are at most $m-2$. Thus $d$ embeds into $1^{m-2} m 1^{m-2}$. Otherwise, $d$ must be the composition $1^m$, and as the $c$ has length $2m-3 \ge m$, we have that $d$ embeds into $c$, and thus $c$ is $\C_m$-universal.

	Now, suppose $c$ is any $\C_m$-universal composition. We claim that $c$ must have size at least $3m-4$. As $c$ contains the composition $m$, there must some index $j$ so that $c(j) \ge m$. Let $a$ denote the prefix of $c$ before the part $c(j)$ and let $b$ denote the suffix of $c$ after the part $c(j)$ so that $c = a c(j) b$. Let $\operatorname{h}(a)$ and $\operatorname{h}(b)$ denote the largest parts of $a$ and $b$, respectively. If $\operatorname{h}(a) < m$, then $c$ must contain the composition $(\operatorname{h}(a)+1)1^{m-\operatorname{h}(a)-1}$, and as the earliest that the part $\operatorname{h}(a)+1$ may embed into $c$ is into $c(i)$, the composition $1^{m-\operatorname{h}(a)-1}$ must embed into $b$, and thus $\len(b) \ge m-\operatorname{h}(a)-1$. If $\operatorname{h}(a) \ge m$, then $\len(b) \ge 0 > m - \operatorname{h}(a) - 1$ anyways. Similarly, we must have $\len(a) \ge m - \operatorname{h}(b) - 1$ in any case. For any composition $d$, we have $\size{d} \ge \len(d) + \operatorname{h}(d) - 1$, and thus
	\begin{align*}
		\size{c}
			&= c(i) + \size{a} + \size{b} \\
			&\ge m +  \len(a) + \operatorname{h}(a) - 1 + \len(b) + \operatorname{h}(b)  - 1 \\
			&=   m + (\len(a) + \operatorname{h}(b)) +   (\len(b) + \operatorname{h}(a)) - 2 \\
			&\ge m + (m-1) + (m-1) - 2 \\
			&= 3m-4,
	\end{align*}
	as desired.
\end{proof}

%%%%%%%%%%%%%%%%%%%%%%%%%%%%%%%%%%%%%%%%%%%%%%%%%%%%%%%%%%%%%%%%
\section{Concluding Remarks}
\label{sec-comp-conclusion}
%%%%%%%%%%%%%%%%%%%%%%%%%%%%%%%%%%%%%%%%%%%%%%%%%%%%%%%%%%%%%%%%

In its proof, Theorem~\ref{thm-comp-universal} provides a recursive strategy for constructing smallest $m$-universal compositions, leading to a precise formula for their size asymptotic to $m \log m$. As in the case of integer partitions discussed in Chapter~\ref{chap-partitions}, any proper class of compositions $\C$ admits proper $\C_m$-universal compositions if and only if $\C$ satisfies the joint-embedding property. This property is equivalent to $\C$ being the age of some potentially infinite composition $u$, as the following result of Fra{\"i}ss{\'e} shows:

\begin{theorem}[Fra{\"i}ss{\'e}~\cite{fraisse:sur-lextension-:}; see also Hodges~{\cite[Section 7.1]{hodges:model-theory:}}] 
	The following are equivalent for a class $\P$ of integer partitions or compositions:
	\begin{enumerate}
		\item $\P$ cannot be expressed as the union of two proper subclasses,
		\item $\P$ satisfies the \emph{joint embedding property}, meaning that for every $a, b \in \P$ there is some $c \in \P$ such that $a, b \le c$, and 
		\item $\P = \Age(u)$ for some word $u \in (\mathbb{P} \cup \{n^\omega \st n \in \mathbb{P}\} \cup \{\omega, \omega^\omega\})^\ast$.
	\end{enumerate}
\end{theorem}

While Theorem~\ref{thm-comp-bdd-part-size} provides a recursive formula for the size of the smallest (proper) universal compositions for the classes of compositions whose part sizes are bounded, an explicit formula exists only in Conjecture~\ref{conj-comp-bdd-part-size}. For the classes of compositions of bounded size, the both the sizes of the smallest proper and not-necessarily-proper universal compositions remain unknown, but Conjecture~\ref{conj-comp-bdd-num-parts-proper} provides a guess in the former case.
\chapter{Graphs} 
\label{chap-graphs}

In this chapter, we consider the problem of universality for simple undirected graphs without loops. Given a graph $G$, we denote its vertex set by $V(G)$ and its edge set by $E(G)$. We call the number of vertices of $G$ its \emph{size}, denoted $\size{G}$, and denote its number of edges by $e(G)$. For a set $S$ of vertices of $G$, the \emph{subgraph of $G$ induced by $S$} is the graph on vertex set $S$ containing each edge in $E(G)$ in which both endpoints lie in $S$. We consider graphs up to isomorphism, where two graphs $G$ and $H$ are isomorphic if there is a bijection $\varphi: V(G) \to V(H)$ such that $u$ is adjacent to $v$ in $G$ if and only if $\varphi(u)$ is adjacent to $\varphi(v)$ in $H$ for all $u, v \in V(G)$. Finally, we say that a graph $G$ \emph{contains} another graph $H$ if there is some set $S$ of vertices of $G$ that induces the graph $H$ in $G$, and otherwise we say that $G$ is $H$-free.

In~\cite{rado:universal-graph:}, Rado introduced the concept of a \emph{universal} graph, which he defined as a countably infinite graph that contained all finite graphs as induced subgraphs. Building on this idea, Moon~\cite{moon:on-minimal-n-un:} defined a graph to be $m$-universal if it contained all graps on $m$ finite graphs that contain all finite graphs on $m$ vertices. Moon observed that there are at least $2^{\binom{m}{2}}/m!$ graphs on $m$ vertices, as there are $2^{\binom{m}{2}}$ labeled graphs on $m$ vertices and at most $m!$ of these correspond to each unlabeled graph. Thus, any $m$-universal graph of size $n$ must have at least $2^{\binom{m}{2}}/m!$ subsets of size $m$, so
\[
	\binom{n}{m}
	\ge
	\dfrac{2^{\binom{m}{2}}}{m!}
	.
\]
Using the inequality $\binom{n}{m} \le n^m/m!$, it follows that $n \ge 2^{(m-1)/2}$. Moreover, Moon constructed an $m$-universal graph $\gamma(m)$, where the size of $\gamma(m)$ is given by
\[
	\size{\gamma(m)}
	\le
	\begin{cases}
		m \cdot 2^{(m-1)/2}                       & \text{if $m$ is odd,} \\
		\dfrac{3}{2\sqrt{2}}\ m \cdot 2^{(m-1)/2} & \text{if $m$ is even}
	\end{cases}
\]

Building on decades of intermediate results, Alon~\cite{alon:asymptotically-:} proved that the smallest $m$-universal graphs have size asymptotic to $2^{(m-1)/2}$.

%%%%%%%%%%%%%%%%%%%%%%%%%%%%%%%%%%%%%%%%%%%%%%%%%%%%%%%%%%%%%%%%
\section{Proper Classes of Graphs}
\label{sec-graphs-proper}
%%%%%%%%%%%%%%%%%%%%%%%%%%%%%%%%%%%%%%%%%%%%%%%%%%%%%%%%%%%%%%%%

In the remainder of this chapter, we examine the problem of universality for proper classes of graphs. Before proceeding, we introduce some definitions that will prove necessary. A \emph{clique} is a graph with all possible edges. Given graphs $G$ and $H$, the \emph{complement} of $G$, denoted co-$G$, is the graph on the same vertex set that contains precisely those edges that are not present in $G$. The \emph{union} of $G$ and $H$, denoted $G \union H$, is the graph whose vertex set is the disjoint union of the vertex sets of $G$ and $H$ and whose edge set consists of all the edges from both $G$ and $H$. Similarly, the \emph{join} of $G$ and $H$, denoted $G \join H$, is the graph whose vertex set is the disjoint union of the vertex sets of $G$ and $H$ and whose edge set consists of all the edges from both $G$ and $H$ as well as every edge connecting a vertex from $G$ to a vertex from $H$.

In~\cite{alstrup:small-induced-universal:}, Alstrup and Rauhe present a simple  argument lower bounding the size of a universal graph for any set of graphs.
\begin{observation}[Alstrup and Rauhe~\cite{alstrup:small-induced-universal:}]
\label{obs-alstrup-rauhe}
	For any set of graphs $\G$, the size of a $\G_m$-universal graph must be at least $\dfrac{m}{e} \size{\G_m}^{\nicefrac{1}{m}}$.
\end{observation}
\begin{proof}
	Let $H$ be a $\G_m$-universal graph, with our aim being to show that $\size{H} \ge \frac{m}{e} \size{\G_m}^{\nicefrac{1}{m}}$. The number of induced subgraphs of size $m$ in $G$ is bounded below by $\binom{\size{H}}{m}$, as each subgraph of $H$ of size $m$ corresponds to at least one subset of $m$ vertices of $H$, and thus $\binom{\size{H}}{m} \ge \size{\G_m}$. As $\left(\frac{e \cdot \size{H}}{m}\right)^m \ge \binom{\size{H}}{m}$, we have $\size{H} \ge \frac{m}{e} \size{\G_m}^{\nicefrac{1}{m}}$, as desired.
\end{proof}

In~\cite{lozin:minimal-univers:}, Lozin and Rudolf define a sense of optimality for universal graphs, which we now present. To this end, they begin with the same observation of Alstrup and Rauhe above, noting that if $\G$ is a class of graphs and $H$ is a $\G_m$-universal graph, then we must have 
\[	
	\size{\G_m}
	\le
    \binom{\size{H}}{m}   
\]
as each graph in $\G_m$ must correspond to at least one subset of $m$ vertices of $G$. As $\binom{\size{H}}{m} \le \size{H}^m$, we have
\[
	\size{\G_m}
	\le
	\size{H}^m.
\]
This inequality, together with $\size{H} \ge m$, which is true if $\G$ is infinite and thus each $\G_m$ is nonempty, gives the pair of inequalities
\begin{align*}
	\log\size{\G_m} 
		&\le m \log \size{H} \\[5pt]
	m \log m 
		&\le m \log \size{H}.
\end{align*}
Finally, we say that a sequence of graphs $H_m$ that are $\G_m$-universal is \emph{asymptotically optimal} if
\[
	\lim_{m \to \infty}
	\dfrac{m \log \size{H_m}}{\max\{\log \size{\G_m}, m \log m\}}
	=
	1
\]
and \emph{optimal in order} (\emph{order-optimal}) if there is a constant $c$ such that for any $m \ge 1$,
\[
	\dfrac{m \log \size{H_m}}{\max\{\log \size{\G_m}, m \log m\}}
	\le 
	c.
\]

In~\cite{kannan:implicit-representation-1992:}, Kannan, Naor, and Rudich ask if every graph class $\G$ with $\log\size{\G_m} = \oO{m \log m}$ admits universal graphs of polynomial size. Spinrad~\cite{spinrad:efficient-graph:} called the (unproven) affirmation to this question the \emph{implicit graph conjecture} and provided a generalization to classes of all sizes:
%
\newtheorem*{gen-implicit}{Generalized implicit graph conjecture}%
\begin{gen-implicit}
	For every graph class $\G$, there are $\G_m$-universal graphs $H_m$ with $\log\size{H_m} = \oO{\log\size{\G_m}/m}$.
\end{gen-implicit}

In the language of Lozin and Rudolf above, the generalized implicit graph conjecture may be restated:
\newtheorem*{gen-implicit-equiv}{Generalized implicit graph conjecture (equivalent)}
\begin{gen-implicit-equiv}
	Every graph class admits an order-optimal sequence of universal graphs.
\end{gen-implicit-equiv}

%%%%%%%%%%%%%%%%%%%%%%%%%%%%%%%%%%%%%%%%%%%%%%%%%%%%%%%%%%%%%%%%
\subsection{Cluster Graphs}
\label{subsec-graphs-cluster}
%%%%%%%%%%%%%%%%%%%%%%%%%%%%%%%%%%%%%%%%%%%%%%%%%%%%%%%%%%%%%%%%

Let $\C$ be the class of \emph{cluster} graphs, which are precisely those graphs that are disjoint unions of cliques. Equivalently, $\C$ is the class of $P_3$-free graphs. We begin by noting that $\C$ is isomorphic to the poset of integer partitions. This means that the unique smallest universal partition constructed in Theorem~\ref{thm-ptn-universal} translates to a smallest proper universal graph for $\C$ of the same size:
\begin{proposition}
\label{prop-graph-cluster-proper}
For all integers $m \ge 0$, the smallest proper $\C_m$-universal graphs have size 
\[
	\u_{\C}^p(m)
	=
	\phi(m)
	=
	\floor{\dfrac{m}{1}} + \floor{\dfrac{m}{2}} + \cdots + \floor{\dfrac{m}{m}}.
\]
\end{proposition}

We will show that no smaller $\C_m$-universal graph may be found by searching outside the class by showing that any graph may be transformed into a cluster graph of the same size without losing any of its contained cluster graphs.

\begin{proposition}
\label{prop-clusterize}
	Given any graph $G$ of size $n$, there is a cluster graph of size $n$ that contains every cluster graph contained in $G$.
\end{proposition}
\begin{proof}
	Let $S \subseteq V(G)$ be a set of vertices that induce a clique of maximum size in $G$, and let $H$ be the graph formed from $G$ by deleting each edge in which precisely one endpoint lies in $S$. We claim that $H$ contains every cluster graph contained in $G$. Let $g$ be a cluster graph contained in $G$, and fix some embedding $\iota$ of $g$ into $G$. Note that at most one clique of $g$ uses vertices of $S$ in $G$. If (the fixed embedding of) $g$ uses no vertices of $S$, then the same $\iota$ may be used to embed $g$ into $H$. Otherwise, whichever clique of $g$ uses vertices of $S$ may be embedded \emph{entirely} within $S$ in $G$. Let $\iota'$ be this embedding. Then $\iota'$ may be used to embed $g$ into $H$ as well, and thus $H$ contains each cluster graph contained in $G$. 
	
	By induction, we can transform the graph $H_1$ induced by remaining vertices $V(H) \setminus S$ into a cluster graph $C_1$ while retaining each of the cluster graphs contained in $H_1$. The union of $C_1$ with the clique $S$ is thus a cluster graph that contains each cluster graph that $G$ does, completing the proof.
\end{proof}

In particular, Proposition~\ref{prop-clusterize} implies that any $\C_m$-universal graph may be transformed into a proper $\C_m$-universal graph of the same size.
\begin{corollary}
\label{cor-graph-cluster-improper}
For all integers $m \ge 0$, the smallest $\C_m$-universal graphs have size
\[
	\u_{\C}(m)
	=
	\phi(m)
	=
	\floor{\dfrac{m}{1}} + \floor{\dfrac{m}{2}} + \cdots + \floor{\dfrac{m}{m}}.
\]
\end{corollary}

The number of cluster graphs of size $n$ is equal to the number of partitions of size $n$, and thus $\size{\C_n} \sim \dfrac{1}{4n\sqrt{3}} \exp(\pi \sqrt{2n/3})$ by a result of Hardy and Ramanujan~\cite{hardy:asymptotic-formulae:}. From this, it follows that the unique sequence of smallest proper $\C_m$-universal graphs of size $\phi(m) \sim m \log m$ is asymptotically optimal. 

Proposition~\ref{prop-clusterize} also implies two linear lower bounds on the size of universal graphs for classes that contain certain disjoint unions of cliques:
\begin{corollary}
\label{cor-graph-lower-big-cliques}
	If the graph $G$ contains $k K_{\floor{m/k}}$ for $1 \le k \le \ell$, then
	\[
		\size{G}
		\ge
		\floor{\dfrac{m}{1}} + \floor{\dfrac{m}{2}} + \cdots + \floor{\dfrac{m}{\ell}}.
	\]
\end{corollary}
\begin{proof}
	Suppose that the graph $G$ contains the graph $\floor{\frac{m}{k}} K_{k}$ for $1 \le k \le \ell$. By Proposition~\ref{prop-clusterize}, we may assume that $G$ is a cluster graph. As the poset of cluster graphs is isomorphic to the poset of integer partitions, it suffices to show that if $p = p(1) \cdots p(t)$ is a partition that contains $\floor{\frac{m}{k}}^k$ for all $1 \le k \le \ell$, then $\size{p} \ge \floor{m/1} + \floor{m/2} + \cdots + \floor{m/\ell}$. For each $k$, the fact that $p$ contains $\floor{\frac{m}{k}}^k$ is equivalent to $p(k) \ge \floor{\frac{m}{k}}$, and thus
	\begin{align*}
		\size{p}
			&= p(1) + p(2) + \cdots + p(n) \\
			&\ge p(1) + p(2) + \cdots + p(\ell) \\
			&\ge \floor{\dfrac{m}{1}} + \floor{\dfrac{m}{2}} + \cdots + \floor{\dfrac{m}{\ell}},
	\end{align*}
	as desired.
\end{proof}

\begin{corollary}
	\label{cor-graph-lower-many-cliques}
	If the graph $G$ contains $\floor{\frac{m}{k}} K_{k}$ for $1 \le k \le \ell$, then
	\[
		\size{G}
		\ge
		\floor{\dfrac{m}{1}} + \floor{\dfrac{m}{2}} + \cdots + \floor{\dfrac{m}{\ell}}.
	\]
\end{corollary}
\begin{proof}
	Suppose that the graph $G$ contains the graph $\floor{\frac{m}{k}} K_{k}$ for $1 \le k \le \ell$. By Proposition~\ref{prop-clusterize}, we may assume that $G$ is a cluster graph. As the poset of cluster graphs is isomorphic to the poset of integer partitions, it suffices to show that if $p = p(1) \cdots p(t)$ is a partition that contains $k^{\floor{m/k}}$ for all $1 \le k \le \ell$, then $\size{p} \ge \floor{m/1} + \floor{m/2} + \cdots + \floor{m/\ell}$. 

	For any partitions $q, r$, we have $q \le r$ if and only if $q^{\conj} \le r^{\conj}$. The conjugate of $k^{\floor{m/k}}$ is $\floor{\frac{m}{k}}^k$, so if $p^{\conj} = p'(1) p'(2) \cdots p'(s)$, then $p^{\conj}$ contains $\floor{\frac{m}{k}}^k$ for each $1 \le k \le \ell$, and thus $p'(k) \ge \floor{\frac{m}{k}}$ for each $1 \le k \le \ell$. Finally, we have
	\begin{align*}
		\size{p}
			&= \size{p^{\conj}} \\
			&= p'(1) + p'(2) + \cdots + p'(n) \\
			&\ge p'(1) + p'(2) + \cdots + p'(\ell) \\
			&\ge \floor{\dfrac{m}{1}} + \floor{\dfrac{m}{2}} + \cdots + \floor{\dfrac{m}{\ell}},
	\end{align*}
	as desired.
\end{proof}

In the case that $\ell = 2$, Corollary~\ref{cor-graph-lower-many-cliques} shows that any graph that contains $m K_1$, $\floor{\frac{m}{2}} K_2$, and $\floor{\frac{m}{3}} K_3$ has size at least $m + \floor{\frac{m}{2}} + \floor{\frac{m}{3}}$, improving upon and generalizing the following result, which appears in~\cite{esperet:on-induced-universal:}.

\begin{claim}[{Esperet, Labourel, and Ochem~\cite[Claim 1]{esperet:on-induced-universal:}}]
\label{claim-max-degree-2}
	Let $\F$ denote the class of graphs with maximum degree $2$. Every $\F_m$-universal graph has size at least $11\floor{m/6}$.
\end{claim}


%%%%%%%%%%%%%%%%%%%%%%%%%%%%%%%%%%%%%%%%%%%%%%%%%%%%%%%%%%%%%%%%
\subsection{Bipartite Permutation Graphs}
%%%%%%%%%%%%%%%%%%%%%%%%%%%%%%%%%%%%%%%%%%%%%%%%%%%%%%%%%%%%%%%%

Let $\G$ be the class of bipartite permutation graphs. Then $\G$ is the collection of $\{S_{2,2,2}, \text{Sun}_{3}, \Phi, C_3, C_5, C_6, \cdots\}$-free graphs, where the graphs $S_{2,2,2}$, $\text{Sun}_{3}$, and $\Phi$ are drawn in Figure~\ref{fig-graphs-bipartite-perm}.
\begin{figure}[ht]
\captionsetup{justification=centering}
\setlength{\tabcolsep}{12pt}
	\begin{tabular}{ccc}
		\begin{tikzpicture}[graphs, scale = {2/3}]
			\node (vCenter) at (0, 0) {};
			\foreach \theta in {90, 210, 330} {
				\node (vInner\theta) at (\theta:1) {};
				\node (vOuter\theta) at (\theta:2) {};
				\draw (vCenter) -- (vInner\theta) -- (vOuter\theta);
			}
		\end{tikzpicture}
		&
		\begin{tikzpicture}[graphs, scale = {2/3}]
			\foreach \theta in {0, 90, 180} {
				\node (vInner\theta) at (\theta:1) {};
				\node (vOuter\theta) at (\theta:2) {};
				\draw (vInner\theta) -- (vOuter\theta);
			}
			\node (vInner270) at (270:1) {};
			\draw (vInner0) -- (vInner90) -- (vInner180) -- (vInner270) -- (vInner0);
		\end{tikzpicture}
		&
		\begin{tikzpicture}[graphs]
			\node (SE) at ( 1,-1) {};
			\node (S)  at ( 0,-1) {};
			\node (SW) at (-1,-1) {};

			\node (W)  at (-1, 0) {};
			\node (O)  at ( 0, 0) {};
			\node (E)  at ( 1, 0) {};

			\node (N)  at ( 0, 1) {};
			
			\draw (N) -- (O) -- (E) -- (SE) -- (S) -- (SW) -- (W) -- (O) -- (S);
		\end{tikzpicture}
		\\
		$S_{2,2,2}$ & $\text{Sun}_{3}$ & $\Phi$
	\end{tabular}
\caption{The graphs that, in addition to the cycles $\{C_3, C_5, C_6, \dots\}$, comprise the set of minimal forbidden subgraphs of the class of bipartite permutation graphs.}
\label{fig-graphs-bipartite-perm}
\end{figure}
In~\cite{lozin:minimal-univers:}, Lozin and Rudolf explicitly construct a proper $\G_m$-universal of size $m^2$. It is well known that $\log\size{\G_m} \sim \frac{m}{2} \log m$\footnote{The number of bipartite permutation graphs of size $n$ is sequence \OEISlink{A000085} in the OEIS~\cite{sloane:the-on-line-enc:}.}, so their construction is order-optimal. The proper $\G_5$-universal graph of their construction is drawn in Figure~\ref{fig-graphs-bipartite-perm-univ}.

\begin{figure}[ht]
\captionsetup{justification=centering}
	\begin{tikzpicture}[
		scale=1.5, 
		every node/.style={circle,fill=black,inner sep=0pt,minimum size=4pt}
	]
		\foreach \y in {0,1,...,4} {
			\pgfmathtruncatemacro\yminus{\y - 1};
			\foreach \x in {0,1,...,4} {
				\node (v\x\y) at (\x, \y) {};
				\ifthenelse{\equal{\y}{0}}{}{
					\foreach \z in {0,...,\x} {
						\draw (v\z\yminus.center) -- (v\x\y.center);
					}
				}
			}
		}
	\end{tikzpicture}
\caption{The $\G_5$-universal graph constructued of Lozin and Rudolf's construction~\cite{lozin:minimal-univers:}.}
\label{fig-graphs-bipartite-perm-univ}
\end{figure}

In~\cite{alecu:critical-properties:}, Alecu, Lozin, and Malyshev prove that every proper $\G_m$-universal graph has $\oOmega{m^\alpha}$ vertices for all $\alpha < 2$. Moreover, they conjecture that the minimum size of a proper $\G_m$-universal graph is $\oOmega{m^2}$.

%%%%%%%%%%%%%%%%%%%%%%%%%%%%%%%%%%%%%%%%%%%%%%%%%%%%%%%%%%%%%%%%
\subsection{Caterpillar Forests}
%%%%%%%%%%%%%%%%%%%%%%%%%%%%%%%%%%%%%%%%%%%%%%%%%%%%%%%%%%%%%%%%

A \emph{tree} is a connected acyclic graph, and a \emph{forest} is a disjoint union of trees, ie. an acyclic graph. A \emph{leaf} is a vertex of degree $1$. A connected graph is called a \emph{caterpillar} if its set of non-leaf vertices vertices induces a path. Every caterpillar is a tree, and a disjoint union of caterpillar graphs is called a \emph{cateprillar forest}. Let $\G$ be the class of caterpillar forests, or equivalently, the class of acyclic $S_{2,2,2}$-free graphs, where the graph $S_{2,2,2}$ is drawn in Figure~\ref{fig-graphs-bipartite-perm}.

In~\cite{bonichon:short-labels:}, Bonichon, Gavoille, and Labourel construct graphs on at most $64m$ vertices that contain all caterpillars of size $m$. There are precisely $2^{m-4} + 2^{\floor{(m-4)/2}}$ caterpillars of size $m$, so their construction is asymptotically optimal.

%%%%%%%%%%%%%%%%%%%%%%%%%%%%%%%%%%%%%%%%%%%%%%%%%%%%%%%%%%%%%%%%
\subsection{Linear Forests}
%%%%%%%%%%%%%%%%%%%%%%%%%%%%%%%%%%%%%%%%%%%%%%%%%%%%%%%%%%%%%%%%

Let $\G$ be the class of graphs that are disjoint unions of paths. Equivalently, $\G$ is the the set of acyclic $K_{1,3}$-free graphs. In~\cite{abrahamsen:near-optimal-induced-1:}, Abrahamsen, Alstrup, Holm, Knudsen, and St\"ockel construct a $\G_m$-universal graph of size $\floor{3m/2}$, which they note is best-possible by extracting a lower bound of $\floor{3m/2}$ from the proof of Claim~\ref{claim-max-degree-2} in \cite{esperet:on-induced-universal:}. Alternatively, this same lower bound is implied by Corollary~\ref{cor-graph-lower-many-cliques}, as any $\G_m$-universal graphs contains both $m K_1$ and $\floor{\frac{m}{2}} K_2$.

\begin{figure}[ht]
\captionsetup{justification=centering, margin=1in}
\begin{tikzpicture}[scale = 1, graphs]
	\foreach \x in {0, 1, ..., 9} {
		\node (v\x1) at (\x, 1) {};
		\ifthenelse{\equal{\x}{0}}{}{
			\pgfmathtruncatemacro{\xminus}{\x - 1}
			\draw (v\xminus1) -- (v\x1);
		}
	}
	\foreach \x in {0, 1, ..., 4} {
		\pgfmathtruncatemacro{\twox}{2 * \x}
		\node (v\x0) at (\twox, 0) {};
		\draw (v\x0) -- (v\twox1);
	}
\end{tikzpicture}
\caption{The $\G_{10}$-universal graph of size $15$ of Abrahamsen, Alstrup, Holm, Knudsen, and St\"ockel's construction~\cite{abrahamsen:near-optimal-induced-1:}.}
\end{figure}

Linear forests of size $n$ are in bijection with partitions of size $n$, and thus $\size{\G_n} \sim \dfrac{1}{4n\sqrt{3}} \exp(\pi \sqrt{2n/3})$ by a result of Hardy and Ramanujan~\cite{hardy:asymptotic-formulae:}. From this, it follows that the sequence of $\G_m$-universal graphs constructed by Abrahamsen, Alstrup, Holm, Knudsen, and St\"ockel is asymptotically optimal. 

%%%%%%%%%%%%%%%%%%%%%%%%%%%%%%%%%%%%%%%%%%%%%%%%%%%%%%%%%%%%%%%%
\subsection{Star Forests}
%%%%%%%%%%%%%%%%%%%%%%%%%%%%%%%%%%%%%%%%%%%%%%%%%%%%%%%%%%%%%%%%

We call complete bipartite graphs of the form $K_{1,n}$ \emph{star} graphs or just \emph{stars}. Disjoint unions of star graphs are called \emph{star forests}. Let $\SF$ be the class of star forests. Equivalently, $\SF$ is the collection of acyclic $P_4$-free graphs. 

In~\cite{alecu:critical-properties:}, Alecu, Lozin, and Malyshev show that the minimum size of an $\SF_m$-universal bipartite permutation graph is $\oTheta{m \log m}$. This implies that a proper $\SF_m$-universal graph must have size $\oOmega{m \log m}$ and that any $\SF_m$-universal graph must have size $\oO{m \log m}$. As the set of star forests of size $n$ is in bijection with the set of partitions of size $n$, we have that $\size{\SF_n} \sim \dfrac{1}{4n\sqrt{3}} \exp(\pi \sqrt{2n/3})$, and thus the graphs constructed by Alecu, Lozin, and Malyshev are asymptotically optimal.

%%%%%%%%%%%%%%%%%%%%%%%%%%%%%%%%%%%%%%%%%%%%%%%%%%%%%%%%%%%%%%%%
\subsection{Threshold Graphs}
%%%%%%%%%%%%%%%%%%%%%%%%%%%%%%%%%%%%%%%%%%%%%%%%%%%%%%%%%%%%%%%%

A \emph{threshold graph} is a graph that may be constructed from the empty graph through a sequence of unions and joins with a single vertex. Threshold graphs have been widely studied, so much so that they serve as the subject of a 2003 book~\cite{mahadev:threshold-graph:} by Spinrad. Let $\T$ denote the class of non-empty threshold graphs. Below, we show that $\T$ is isomorphic to the poset of words over a two-letter alphabet, discussed in generality in Section~\ref{sec-words}.

\begin{proposition}
\label{prop-graph-threshold-isomorphism}
	Define the map $\gamma: \{\textsf{u}, \textsf{j}\}^\ast \to \T$ recusively as follows:
	\begin{enumerate}
		\item $G(\varepsilon) = g$ is the graph on one vertex.
		\item $G(v\textsf{u}) = G(v) \union g$ is the union of $G(v)$ with a vertex.
		\item $G(v\textsf{j}) = G(v) \join  g$ is the  join of $G(v)$ with a vertex.
	\end{enumerate}
	Then $\gamma$ is a poset isomorphism. 
\end{proposition}
The map $\gamma$ sends words of size $m$ to graphs of size $m+1$, so we omit the $0$-vertex graph from $\T$. Before proving Proposition~\ref{prop-graph-threshold-isomorphism}, we present a useful lemma.
\begin{lemma}
\label{lemma-graph-union}
	If $G$ and $H$ are graphs and $g$ is the one-vertex graph, then 
	\begin{enumerate}
		\item $G \union g \le H \union g$ if and only if $G \le H$.
		\item $G \union g \le H \join  g$ if and only if $G \union g \le H$.
		\item $G \join  g \le H \union g$ if and only if $G \join  g \le H$.
		\item $G \join  g \le H \join  g$ if and only if $G \le H$.
	\end{enumerate}
\end{lemma}

We now present the proof of Proposition~\ref{prop-graph-threshold-isomorphism}.

\newenvironment{proof-of-prop-graph-threshold-isomorphism}{%
	\medskip\noindent {\it Proof of Proposition~\ref{prop-graph-threshold-isomorphism}.\/}%
}{%
	\qed\bigskip%
}
\begin{proof-of-prop-graph-threshold-isomorphism}
	We begin by proving that if $v, w \in \{\textsf{u}, \textsf{j}\}^\ast$ with $v \le w$, then $\gamma(v) \le \gamma(w)$.

	Leveraging the recursive nature of the map $\gamma$, we proceed by induction on the size $v$. As a base case, observe that $\gamma(\varepsilon)$ is the one-vertex graph, which is contained in every non-empty graph. Thus, assume that $v$ is a non-empty word, let the final letter of $v$ be $\ell$, and write $v = v_0 \ell$. Let $w \in \{u, j\}^\ast$ be any word with $v \le w$. As $v$ is a subsequence of $w$, we may write $w = w_0 \ell w_1$, where $v_0 \le w_0$. By induction, we have $\gamma(v_0) \le \gamma(w_0)$. By Lemma~\ref{lemma-graph-union}, we have $\gamma(v_0\ell) \le \gamma(w_0\ell)$ no matter the value of $\ell$. Finally, as $\gamma(w_0\ell) \le \gamma(w_0\ell w_1)$, so
	\[
		\gamma(v_0\ell) 
		\le 
		\gamma(w_0 \ell w_1),
	\]
	as desired.

	For the converse, we aim to show that if $\gamma(v)$ and $\gamma(w)$ are graphs with $\gamma(v) \le \gamma(w)$, then $v \le w$. Again, we proceed by induction on the size of $\gamma(v)$ and begin by noting that the one-vertex graph $\gamma(\varepsilon)$ is contained in every graph in $\T$, and the empty word $\varepsilon$ is contained in every word in $\{u, j\}^\ast$, so the base case is satisfied. 

	Let $\gamma(v)$ and $\gamma(w)$ be graphs with $\gamma(v) \le \gamma(w)$ and $\size{\gamma(v)} \ge 2$, or equivalently, $\size{v} \ge 1$. Without loss of generality, assume that $v = v_0 u$ and $w = w_0 u j^k$ for some $k \ge 0$. By repeated applications of Lemma~\ref{lemma-graph-union}, the containment $\gamma(v_0 u) \le \gamma(w_0 u j^k)$ is equivalent to $\gamma(v_0) \le \gamma(w_0)$, and by induction we have $v_0 \le w_0$, and thus $v_0 u \le w_0 u j^k$, completing the proof.
\end{proof-of-prop-graph-threshold-isomorphism}

One consequence of this isomorphism is that the smallest universal words constructed in Proposition~\ref{prop-word-universal} translate into smallest proper universal threshold graphs of size $2m-1$. Proper $\T_m$-universal graphs of this size were first constructed by Hammer and Kelmans in~\cite{hammer:on-universal-th:}.
\begin{corollary}[Hammer \& Kelmans~\cite{hammer:on-universal-th:}]
\label{cor-graphs-threshold-proper}
	The smallest proper $\T_m$-universal graphs have size $2m-1$.
\end{corollary}

Any $\T_m$-universal graph contains both a clique and a co-clique of size $m$, and these subgraphs must not intersect in more than one vertex. Thus any $\T_m$-universal graph must contain at least $2m-1$ vertices, meaning the smallest proper $\T_m$-universal graphs derived from Proposition~\ref{prop-word-universal} are smallest $\T_m$-universal graphs as well:
\begin{corollary}[Hammer \& Kelmans~\cite{hammer:on-universal-th:}]
\label{cor-graphs-threshold-improper}
	The smallest (proper) $\T_m$-universal graphs have size $2m-1$.
\end{corollary}

%%%%%%%%%%%%%%%%%%%%%%%%%%%%%%%%%%%%%%%%%%%%%%%%%%%%%%%%%%%%%%%%
\subsection{Split Permutation Graphs}
%%%%%%%%%%%%%%%%%%%%%%%%%%%%%%%%%%%%%%%%%%%%%%%%%%%%%%%%%%%%%%%%

A \emph{split graph} is a graph whose vertex set may be partitioned into a clique and a co-clique. Let $\G$ be the class of split permutation graphs. Equivalently, $\G$ is also the class of $\{2K_2, C_4, C_5, \text{net}, \text{co-net}, \text{rising sun}, \text{co-rising sun}\}$-free graphs, where the net and rising sun graphs as well as their complements, the co-net and co-rising sun graphs, are drawn in Figure~\ref{fig-graphs-split}.
\begin{figure}[ht]
\captionsetup{justification=centering}
\setlength{\tabcolsep}{12pt}
	\begin{tabular}{cccc}
		\begin{tikzpicture}[graphs, scale = 0.577350] % 1 / sqrt(3)
			\foreach \theta in {90, 210, 330} {
				\node (vInner\theta) at (\theta:1) {};
				\node (vOuter\theta) at (\theta:2) {};
				\draw (vInner\theta) -- (vOuter\theta);
			}
			\draw (vInner90) -- (vInner210) -- (vInner330) -- (vInner90);
			
			% \node[fill=none] at (0, -1) {net};
		\end{tikzpicture}
		&
		\begin{tikzpicture}[graphs, scale = 0.577350]
			\foreach \theta in {30, 150, 270} {
				\node (vInner\theta) at (\theta:1) {}; % 1/sqrt(3)
			}
			\foreach \theta in {90, 210, 330} {
				\node (vOuter\theta) at (\theta:2) {}; % 2/sqrt(3)
			}

			\draw (vInner30) -- (vInner150) -- (vInner270) -- (vInner30);
			\draw (vOuter90) 
				-- (vInner150) -- (vOuter210) 
				-- (vInner270) -- (vOuter330) 
				-- (vInner30)  -- (vOuter90);
			
			% \node[fill=none] at (0, -1) {co-net};
		\end{tikzpicture}
		&
		\begin{tikzpicture}[graphs, scale = 0.816497] % sqrt(2/3)
			\foreach \theta in {45, 135, 225, 315} {
				\node (vInner\theta) at (\theta:1) {};
			}
			\foreach \theta in {0, 90, 180} {
				\node (vOuter\theta) at (\theta:1.414214) {}; % sqrt(2)
			}

			% "Outer" edges
			\draw  (vInner315) -- (vOuter0)   
				-- (vInner45)  -- (vOuter90)  
				-- (vInner135) -- (vOuter180) 
				-- (vInner225);
			\draw (vInner225) -- (vInner315);

			% "Inner" edges
			\draw (vInner45) -- (vInner135) -- (vInner225) -- (vInner315) -- (vInner45) -- (vInner225);
			\draw (vInner135) -- (vInner315);
			
			% \node[fill=none] at (0, -1) {rising sun};
		\end{tikzpicture}
		&
		\begin{tikzpicture}[graphs]
			\node (vCenter) at (0,0) {};

			\foreach \theta in {0, 60, 120, 180} {
				\node (vInner\theta) at (\theta:1) {};
			}

			\foreach \theta in {60, 120} {
				\node (vOuter\theta) at (\theta:2) {};
			}

			% Left triangle
			\draw (vCenter) -- (vInner0)   -- (vInner60)  -- (vCenter);
			% Right triangle
			\draw (vCenter) -- (vInner120) -- (vInner180) -- (vCenter);

			% Uppers
			\draw (vOuter120)  -- (vInner120)  -- (vInner60) -- (vOuter60);
			
			% \node[fill=none] at (0, -1) {co-risinsg sun};
		\end{tikzpicture}
		\\
		net & co-net & rising sun & co-rising sun
	\end{tabular}
\caption{The forbidden subgraphs for the class of split permutation graphs.}
\label{fig-graphs-split}
\end{figure}
In~\cite{Atminas:Universal-graph:}, Atminas, Kitaev, Lozin, and Valyuzhenich construct a proper $\G_m$-universal graph of size $4m^3$, and in~\cite{brignall:bichain-graphs:}, Brignall, Lozin, and Stacho construct a $\G_m$-universal graph of size $m^2$.

%%%%%%%%%%%%%%%%%%%%%%%%%%%%%%%%%%%%%%%%%%%%%%%%%%%%%%%%%%%%%%%%
\section{Concluding Remarks}
%%%%%%%%%%%%%%%%%%%%%%%%%%%%%%%%%%%%%%%%%%%%%%%%%%%%%%%%%%%%%%%%

Just two years after Moon introduced $m$-universal graphs, Breur~\cite{breuer:coding-the:} introduced adjacency labeling schemes, which aim to efficiently encode the adjacencies of graphs. Informally, an adjacency labeling scheme for a graph $G$ is a labeling of the vertices so that the adjacency of any two vertices may be recovered by knowing only their labels. 

More formally, a \emph{$b$-bit adjacency labeling scheme} for a family of graphs $\F$ is (1) an assignment of a length-$b$ bitstring to each vertex in each graph in $F$ and (2) some method so that, given two vertices $u, v$ of a graph $G$ in $F$, we can determine the adjacency between $u$ and $v$ in $G$ given only their labels\footnote{Some scholars insist that the algorithm that recovers the adjacency of $u$ and $v$ be computable in polynomial time, but we do not.}. More than three decades after universal graphs and adjacency labelings schemes were introduced, Kannan, Naor, and Rudich~\cite{kannan:implicit-representation-1988:} proved their equivalence: a family of graphs $F$ admits a $b$-bit labeling scheme if an only if there is an $F$-universal graph with $2^b$ vertices.

To conclude, we would be remiss if we did not remark upon the substantial efforts to characterize $m$-universal graphs. We say that a set of vertices in a graph $G$ is \emph{homogenous} if it induces either a clique or a co-clique in $G$. The following sequence of results show that, with various interpretations, graphs that contain only small homogenous sets are universal.
Let $\hom(G)$ denote the size of the largest homogenous subset of $G$, and define
\[
	\hom(n,m)
	=
	\min\{\hom(G) \st \text{$G$ of size $n$ not $m$-universal}\}
\]
Equivalently, $\hom(n,m)$ is the largest integer so that for all graphs $G$ on $n$ vertices, the inequality $\hom(G) < \hom(n,m)$ implies that $G$ is $m$-universal.

In~\cite{erdos:on-spanned-subgraphs:}, Erd\H{o}s and Hajnal show that for all integers $m$, for all positive constants $c$, and for all $n$ sufficiently large, $\hom(n,m) \ge c \log(n)$. In~\cite{erdos:ramsey-type-the:}, they go on to improve this result, proving the following theorem.
\begin{theorem}[Erd\H{o}s and Hajnal~\cite{erdos:ramsey-type-the:}]
\label{thm-erdos-hajnal}
	Let $m$ be a positive integer, and let $0 < c < 1/m$. There there is some $n_0 = n_0(m,c)$ such that for all $n > n_0$, we have 
	\[
		\hom(n,m) \ge e^{c\sqrt{\log n}/2}.	
	\]
\end{theorem}

In~\cite{promel:non-ramsey-graphs:}, Pr\"{o}mel and R\"{o}dl allowed the ``level'' of universality to vary with the size of the graph, proving the following.
\begin{theorem}[Pr\"{o}mel and R\"{o}dl~\cite{promel:non-ramsey-graphs:}]
\label{thm-promel-rodl}
For any $c_1 > 0$, there is some $c_2 > 0$ such that for all $n$, we have
\[
	\hom(n,c_2 \log n) \ge c_1 \log n.
\]
\end{theorem}
In~\cite{fox:induced-Ramsey-type:}, Fox and Sudakov generalize Theorem~\ref{thm-promel-rodl}:
\begin{theorem}[Fox and Sudakov~\cite{fox:induced-Ramsey-type:}]
\label{thm-fox-sudakov}
There are constants $c_1, c_2 > 1$ such that for all $n$, $m$, we have 
\[
	\hom(n,m) \ge c_1 2^{c_2 \sqrt{\frac{\log n}{m}}} \log n.
\]
\end{theorem}

Another structural description of $m$-universal graphs came in~\cite{chung:on-graphs-not:}, wherein Chung and Graham give alternative sufficient conditions for a graph to be $m$-universal:
\begin{theorem}[Chung and Graham~~\cite{chung:on-graphs-not:}]
\label{thm-chung-universal}
Let $m$ be a positive integer. If the graph $G$ of size $n$ is not $m$-universal, then there is some induced subgraph $H \subseteq G$ of size $\floor{n/2}$ so that
\[
	\size{e(H) - \frac{n^2}{16}}
	> 
	2^{-2(m^2+27)}n^2.
\]
\end{theorem}
Restated, Theorem~\ref{thm-chung-universal} says that graphs that are not universal must contain a subgraph that has an edge density that strongly deviates from what is ``typical''. Let $D(n, m)$ denote the largest integer such that every graph $G$ on $n$ vertices that is not $m$-universal contains a subgraph $H$ of size $\floor{n/2}$ with $\size{e(H)-\frac{1}{16}n^2} > D(n, m)$. The above result shows that $D(n, m) \ge 2^{-2(m^2+27)}n^2$ for sufficiently large $n$. In \cite{fox:induced-Ramsey-type:}, Fox and Sudakov use Theorem~\ref{thm-fox-sudakov} to show the following result.
\begin{proposition}
\label{prop-fox-sudakov-d}
There are constants $c_3, c_4 > 1$ such that $c_3^{-m}n^2 < D(n,m) < c_4^{-m}n^2$ for all positive integers $n, m$.
\end{proposition}

\chapter{Permutations}
\label{chap-Permutations}

From our perspective, a \emph{permutation of size $n$} is a sequence of length $n$ taken from the alphabet $\{1, 2, \dots, n\}$ in which each letter occurs precisely once. To define containment, we say that two sequences $u$ and $v$ of the same length are \emph{order-isomorphic} if for all indices $i$ and $j$ we have
\[
	u(i) > u(j)
	\iff
	v(i) > v(j).
\]
Finally, the permutation $\pi$ contains another permutation $\sigma = \sigma(1) \sigma(2) \cdots \sigma(m)$ \emph{as a pattern} if there is a subsequence $\pi(i_1) \pi(i_2) \cdots \pi(i_m)$ of $\pi$ that is order-isomorphic to $\sigma$. If $\pi$ does not contain $\sigma$, then we say that $\pi$ \emph{avoids} $\sigma$.

It is helpful to identify a permutation $\pi = \pi(1) \cdots \pi(n)$ with its plot. A \emph{plot} of $\pi$ is a set of points in the plane, no two with the same $x$- or $y$-value, so that the sequence of $y$-values read from left to right are order-isomorphic to $\pi$, and \emph{the (canonical)} plot of $\pi$ is the set of points $\{(i,\pi(i))\}_{i = 1}^{n}$. This graphical representation of $\pi$ helps to visualize the pattern-containment order: $\pi$ contains $\sigma$ if a subset of the plot of $\pi$ constitutes a plot of $\sigma$. An example of containment displayed through plots is exhibited in Figure~\ref{fig-perm-plots}.

\begin{figure}[ht]
\captionsetup{justification=centering}
	\begin{tikzpicture}[scale={1/3}, baseline=(current bounding box.center)]
		% The permutation:
		\draw (0,0) rectangle (6,6);
		\plotpartialperm{1/2,2/5,4/1};
		% Labels:
		\node at (1,0) [below] {$2$};
		\node at (2,0) [below] {$3$};
		\node at (4,0) [below] {$1$};
		
		\begin{scope}[shift={(8.5,0)}]
			\node at (0,3) {$\le$};
			\node at (0,0) [below] {$\le$};
	%			\node at (0,0) [below] {\phantom{$1$}};
		\end{scope}

		\begin{scope}[shift={(11,0)}]
			% The permutation:
			\draw (0,0) rectangle (6,6);
			\plotperm{2,5,3,1,4};
			% The containment of 231---can't use \plotpermencircle because of strange font sizes.
			\begin{scope}[xshift=-0.15pt, yshift=+1.45pt]
				\draw (1,2) circle (16pt);
				\draw (2,5) circle (16pt);
				\draw (4,1) circle (16pt);
			\end{scope}
			% Labels:
			\node at (1,0) [below] {$\underline{2}$};
			\node at (2,0) [below] {$\underline{5}$};
			\node at (3,0) [below] {$3$};
			\node at (4,0) [below] {$\underline{1}$};
			\node at (5,0) [below] {$4$};
		\end{scope}
	\end{tikzpicture}
\caption{An example of permutation containment displayed through their plots.}
\label{fig-perm-plots}
\end{figure}

Before introducing the universality problem for permutations, we give a few definitions that will be helpful throughout this chapter. Given permutations $\pi$ and $\sigma$ of respective sizes $m$ and $n$, their \emph{direct sum} is the permutation $\pi\directsum\sigma$ of size $m+n$ defined by
\[
	(\pi\directsum\sigma)(i)
	=
	\left\{\begin{array}{ll}
		\pi(i)         &\text{if $  1 \le i \le m$,}\\
		\sigma(j-m)+m  &\text{if $m+1 \le i \le m+n$.}
	\end{array}\right.
\]
Pictorially, the plot of $\pi\directsum\sigma$ consists of the plot of $\sigma$ placed above and to the right of the plot of $\pi$, as shown on the left of Figure~\ref{fig-sums}. Similarly, we define the \emph{skew sum} of $\pi$ and $\sigma$ by
\[
	(\pi\skewsum\sigma)(i)
	=
	\left\{\begin{array}{ll}
		\pi(i) + n  &\text{if $  1 \le i \le m  $,}\\
		\sigma(j-m) &\text{if $m+1 \le i \le m+n$.}
	\end{array}\right.
\]
Pictorially, the plot of $\pi\skewsum\sigma$ consists of the plot of $\sigma$ placed below and to the right of the plot of $\pi$, as shown on the right of Figure~\ref{fig-sums}. A permutation that cannot be written as a direct sum (resp. skew sum) is said to be \emph{sum-indecomposable} (resp. \emph{skew-indecomposable}.)

\begin{figure}
	\begin{tikzpicture}[scale=0.2, baseline=(current bounding box.center)]
		\node at (-4, 4.75) {$\pi\directsum\sigma = $};
		\plotpermbox{1}{1}{4}{4};
		\plotpermbox{5}{5}{8}{8};
		\node at (2.5,2.5) {$\pi$};
		\node at (6.5,6.5) {$\sigma$};

		\begin{scope}[shift={(20,0)}]
			\node at (-4, 4.75) {$\pi\skewsum\sigma = $};
			\plotpermbox{1}{5}{4}{8};
			\plotpermbox{5}{1}{8}{4};
			\node at (2.5,6.5) {$\pi$};
			\node at (6.5,2.5) {$\sigma$};
		\end{scope}
	\end{tikzpicture}
\caption{The plot of the direct sum of $\pi$ and $\sigma$ on the left and the plot of the skew sum of $\pi$ and $\sigma$ on the right.}
\label{fig-sums}
\end{figure}

As usual, we say that a permutation is $m$-universal if it contains all permutations of size $m$ as patterns. Such a permutation is sometimes called an \emph{$m$-superpattern} (for example, by B\'ona~\cite[Chapter 5, Exercises 19--22 and Problems Plus 9--12]{bona:combinatorics-o:}.) The first result about universal permutations was obtained by Simion and Schmidt in 1985~\cite[Section~5]{simion:restricted-perm:}, who computed the number of $3$-universal permutations of size $n \ge 5$ to be
\[
	n!
	- 6 C_n
	+ 5 \cdot 2^n
	+ 4 \binom{n}{2}
	- 2F_n
	- 14n
	+ 20.
\]
(Here $C_n$ denotes the $n$th Catalan number and $F_n$ denotes the $n$th \emph{combinatorial} Fibonacci number, so $F_0=F_1=1$ and $F_n=F_{n-1}+F_{n-2}$ for $n\ge 2$.) Simion and Schmidt did not use the term ``universal'' and received their formula as a corollary to a sequence of results enumerating those permutation classes where membership is defined by avoiding sets of $3$-patterns. The first to study the universal permutation problem for general $m$ was Arratia~\cite{arratia:on-the-stanley-:} in 1999.

Let $\u(m)$ denote the size of the smallest $m$-universal permutation. Arratia observed that for any permutation of size $n$ to be $m$-universal, it must have at least $m!$ length-$m$ subsequences, and thus
\[
	\binom{n}{m}
	\ge
	m!.
\]
This simple inequality, together with the inequalities $\left(\frac{ne}{m}\right)^m \ge \binom{n}{m}$ and $m! \ge \left(\frac{m}{e}\right)^m$, implies that $n \ge m^2/e^2$. This also follows from Observation~\ref{obs-alstrup-rauhe}, again using that $m! \ge \left(\frac{m}{e}\right)^m$.

Arratia also provided a simple construction of an $m$-universal permutation of size $m^2$, proving that $\u(m) = \oTheta{m^2}$. Consider the permutation whose plot consists of an $m \times m$ grid of points, with sides initially parallel to the axes, then rotated slightly clockwise. More precisely, partition the integers $1$ to $m^2$ into congruence classes modulo $m$, write each congruence class in increasing order, and then concatenate these sequences in decreasing order according to their first element. The left panel of Figure~\ref{fig-tilted-square} displays the plot of the $4$-universal permutation constructed in this manner. Arratia observed that such a permutation is $m$-universal, as the point $(i, \pi(i))$ of the pattern $\pi$ of size $m$ can simply embed into the point $i$th column (from the left) and $\pi(i)$th row (from the bottom). This kind of embedding actually allows for much more freedom in the universal permutations formed: dividing the square $(0, m^2] \times (0, m^2]$ into $m^2$ many $m \times m$ squares, this construction demonstrates that any permutation whose canonical plot places precisely $1$ point in each sqaure is $m$-universal. We exhibit another $4$-universal permutation of size $16$ that meets this criteria in the right panel of Figure~\ref{fig-tilted-square}.

\begin{figure}[ht]
\captionsetup{justification=centering}
	\begin{tikzpicture}[scale=0.25]
		\plotpermborder{4,8,12,16,3,7,11,15,2,6,10,14,1,5,9,13}

		\draw ( 4.5,0.5) -- ++(0,16);
		\draw ( 8.5,0.5) -- ++(0,16);
		\draw (12.5,0.5) -- ++(0,16);

		\draw (0.5, 4.5) -- ++(16,0);
		\draw (0.5, 8.5) -- ++(16,0);
		\draw (0.5,12.5) -- ++(16,0);

		\begin{scope}[shift={(20,0)}]
			\plotpermborder{6, 13, 9, 1, 11, 2, 5, 15, 10, 8, 4, 16, 14, 3, 12, 7}

			\draw ( 4.5,0.5) -- ++(0,16);
			\draw ( 8.5,0.5) -- ++(0,16);
			\draw (12.5,0.5) -- ++(0,16);

			\draw (0.5, 4.5) -- ++(16,0);
			\draw (0.5, 8.5) -- ++(16,0);
			\draw (0.5,12.5) -- ++(16,0);
		\end{scope}
		
	\end{tikzpicture}
\caption{The $4 \times 4$ tilted-square permutation and another $4$-universal permutation of size $16$.}
\label{fig-tilted-square}
\end{figure}

The first to improve upon these trivial bounds provided by Arratia were Eriksson, Eriksson, Linusson, and W\"{a}stlund~\cite{eriksson:dense-packing-o:}, who used probabilistic methods to construct an $m$-universal permutation of size $(2/3 + \oo{1})m^2$. The next improvement came in~\cite{miller:asymptotic-boun:}, wherein Miller proves that there is a word over the alphabet $[m+1]$ of length $(m^2 + m)/2$ that contains subsequences order-isomorphic to every permutation of size $m$. Miller noted that by ``breaking ties'' between the letters of such a word, one obtains an $m$-universal permutation of size $(m^2 + m)/2$.

\begin{theorem}[Miller~\cite{miller:asymptotic-boun:}]
	\label{thm-miller-perms}
	For all $m \ge 1$, there is a word over the alphabet $[m+1]$ of length $(m^2+m)/2$ containing subsequences order-isomorphic to every permutation of size $m$.
\end{theorem}
	
To establish this result, define the \emph{infinite zigzag word} to be the word formed by alternating between ascending \emph{runs} of the odd positive integers $1357\cdots$ and descending \emph{runs} of the even positive integers $\cdots 8642$,
\[
	(1357\cdots)\ (\cdots 8642)\ 
	(1357\cdots)\ (\cdots 8642)\ 
	(1357\cdots)\ (\cdots 8642)\ \cdots.
\]
While this object does not conform to most definitions of the word \emph{word} in combinatorics, we hope the reader forgives the slight expansion of the definition adopted here. We are interested in the leftmost embeddings of words over $\mathbb{P}$ into the infinite zigzag word.

We also need two definitions. First, given a word $p\in\mathbb{P}^\ast$, we define the word $p^{+1}\in\mathbb{P}^\ast$ to be the word formed by adding $1$ to each letter of $p$, so $p^{+1}(i)=p(i)+1$ for all indices $i$ of $p$. Next we say that the word $p\in\mathbb{P}^\ast$ has an \emph{immediate repetition} if there is an index $i$ with $p(i)=p(i+1)$, i.e., if $p$ contains a factor equal to $\ell\ell$ for some letter $\ell\in\mathbb{P}$.

\begin{proposition}[Miller~\cite{miller:asymptotic-boun:}]
	\label{prop-miller-words}
	If the word $p\in\mathbb{P}^m$ has no immediate repetitions, then either $p$ or $p^{+1}$ occurs as a subsequence of the first $m$ runs of the infinite zigzag word.
\end{proposition}

Before proving Proposition~\ref{prop-miller-words}, note that permutations do not have immediate repetitions. Thus if $\pi$ is a permutation of size $m$, Proposition~\ref{prop-miller-words} implies that either $\pi$ or $\pi^{+1}$ occurs as a subsequence in the first $m$ runs of the infinite zigzag word. Since $\pi^{+1}$ is order-isomorphic to $\pi$ and both $\pi$ and $\pi^{+1}$ are words over $[m+1]$, this implies that the restriction of the first $m$ runs of the infinite zigzag word to the alphabet $[m+1]$ contains every permutation of size $m$. For example, in the case of $m=5$ we obtain the universal word
\[
	135\ 642\ 135\ 642\ 135
\]
of length $15$ over the alphabet $[6]$.

The restriction of the infinite zigzag word described above consists of $m$ runs of average length $(m+1)/2$: if $m$ is odd, then all runs are of this length, while if $m$ is even, then half are of length $m/2$ and half are of length $(m+2)/2$. Thus Proposition~\ref{prop-miller-words} implies Theorem~\ref{thm-miller-perms}. While Proposition~\ref{prop-miller-words} does not appear explicitly in \cite{miller:asymptotic-boun:}, its proof, presented below, is adapted from Miller's proof of Theorem~\ref{thm-miller-perms}.

\newenvironment{proof-of-prop-miller-words}{%
	\medskip\noindent {\it Proof of Proposition~\ref{prop-miller-words}.\/}%
}{%
	\qed\bigskip%
}
%
\begin{proof-of-prop-miller-words}
	We define the \emph{score} of the word $p\in\mathbb{P}^\ast$, denoted by $s(p)$, as the minimum number of runs that an initial segment of the infinite zigzag word must have in order to contain $p$, minus the length of $p$. Thus our goal is to show that for every word $p\in\mathbb{P}^\ast$ without immediate repetitions, either $s(p)\le 0$ or $s(p^{+1})\le 0$. In fact, we show that for such words we have $s(p)+s(p^{+1})=1$, which implies this.

	We prove this claim by induction on the length of $p$. For the base case, we see that words consisting of a single odd letter are contained in the first run of the infinite zigzag word (thus corresponding to scores of $0$) while words consisting of a single even letter are contained in the second run (corresponding to scores of $1$). Thus for every $\ell\in\mathbb{P}^1$ we have $s(\ell)+s(\ell^{+1})=1$, as desired. Now suppose that the claim is true for all words $p\in\mathbb{P}^m$ without immediate repetitions and let $\ell\in\mathbb{P}$ denote a letter. We see that, for any $p \in \mathbb{P}^m$,
	\[
		s(p\ell)-s(p)
		=
		\left\{
		\begin{array}{cl}
			-1&	\begin{array}{l}
				\text{if $p(n)<\ell$ and both entries are odd or}\\
				\text{if $p(n)>\ell$ and both entries are even;}
				\end{array}
			\\[12pt]
			0&	\begin{array}{l}
				\text{if $p(n)$ and $\ell$ are of different parity; or}
				\end{array}
			\\[8pt]
			+1&	\begin{array}{l}
				\text{if $p(n)<\ell$ and both entries are even,}\\
				\text{if $p(n)=\ell$, or}\\
				\text{if $p(n)>\ell$ and both entries are odd.}
				\end{array}
		\end{array}
		\right.
	\]
	Because our words do not have immediate repetitions, we can ignore the possibility that $\ell=p(m)$. In the other cases, it can be seen by inspection that
	\[
		\big(  s(p\ell)-s(p)  \big)
		+
		\big(  s\!\left((p\ell)^{+1}\right) - s\!\left(p^{+1}\right)\!  \big)
		=
		0.
	\]
	By rearranging these terms, we see that
	\[
		s(p\ell) + s\!\left((p\ell)^{+1}\right)
		=
		s(p)+s\!\left(p^{+1}\right).
	\]
	Since $s(p)+s\!\left(p^{+1}\right)=1$ by induction, this completes the proof of the inductive claim, and thus also of the proposition.
\end{proof-of-prop-miller-words}

Here, we give a new improvement to Miller's upper bound. In order to do so, we further restrict the infinite zigzag word, and then break ties between its letters to obtain a specific permutation $\zeta_m$. To this end, we define the word $z_m$ to be the restriction of the first $m$ runs of the infinite zigzag word to the alphabet $[m]$. When $m$ is even, each run of $z_m$ has length $m/2$. When $m$ is odd, $z_m$ consists of $(m+1)/2$ ascending odd runs, each of length $(m+1)/2$, and $(m-1)/2$ descending even runs, each of length $(m-1)/2$. Thus we have
\[
	\size{z_m}
	=
	\left\{
	\begin{array}{cl}
		\displaystyle\frac{m^2}{2}  &\text{if $m$ is even,}\\[12pt]
		\displaystyle\frac{m^2+1}{2}&\text{if $m$ is odd.}
	\end{array}
	\right.
\]

As one additional definition, we say that a word $u$ is \emph{order-homomorphic} to another word $v$ of the same length if for all indices $i$ and $j$, we have
\[
	u(i) > u(j)
	\implies
	v(i) > v(j).
\]
Next, we choose a specific permutation, $\zeta_m$, such that $z_m$ is order-homomorphic to $\zeta_m$. In constructing $\zeta_m$, we have the freedom to break ties between equal letters of $z_m$. That is to say, if $z_m(i)=z_m(j)$ for $i\neq j$, then in constructing $\zeta_m$ we may choose whether $\zeta_m(i)<\zeta_m(j)$ or $\zeta_m(i)>\zeta_m(j)$ arbitrarily without affecting any other pair of comparisons and thus without losing any occurrences of permutations. We choose to break these ties by replacing all instances of a given letter $k\in[m]$ in $z_m$ by a decreasing subsequence in $\zeta_m$. Thus for indices $i<j$, we have
\[
	z_m(i)=z_m(j)
	\implies
	\zeta_m(i)>\zeta_m(j).
\]
This choice uniquely determines $\zeta_m$ (up to order-isomorphism), as all comparisons between its letters are determined either in $z_m$, if the corresponding letters of $z_m$ differ, or by the rule above, if the corresponding letters of $z_m$ are the same. Figure~\ref{fig-z5-zeta5} shows the plots of $z_5$ and $\zeta_5$, where the \emph{plot} of a word $w$ over $\mathbb{P}$ is the set $\{(i,w(i))\}$ of points in the plane.

\begin{figure}[ht]
\captionsetup{justification=centering}
	\begin{footnotesize}
		\begin{tabular}{ccc}
			\begin{tikzpicture}[scale=0.35, baseline=(current bounding box.south)]
				\foreach \y/\val [count = \x] in {1/2, 3/7, 5/12, 4/9.5, 2/4.5, 1/2, 3/7, 5/12, 4/9.5, 2/4.5, 1/2, 3/7, 5/12} {
					\absdot{(\x,\val)}
					\node at (\x, 0) {\y};
					}
				\draw[darkgray, thick, line cap=round] (0.5,0.5) rectangle (13.5,13.5);
				\foreach \y in {3,5,8,10} {
					\draw[darkgray, thick, line cap=round] (0.5, \y+0.5) -- (13.5, \y+0.5);
					\draw[darkgray, thick, line cap=round] (\y+0.5, 0.5) -- (\y+0.5, 13.5);
				}
			\end{tikzpicture}
			&\quad\quad&
			\begin{tikzpicture}[scale=0.35, baseline=(current bounding box.south)]
				\plotperm{3,8,13,10,5,2,7,12,9,4,1,6,11}
				% The labels
				\node at (1,0) {3};
				\node at (2,0) {8};
				\node at (3,0) {1\!\!\:3};
				\node at (4,0) {1\!\!\:0};
				\node at (5,0) {5};
				\node at (6,0) {2};
				\node at (7,0) {7};
				\node at (8,0) {1\!\!\:2};
				\node at (9,0) {9};
				\node at (10,0) {4};
				\node at (11,0) {1};
				\node at (12,0) {6};
				\node at (13,0) {1\!\!\:1};
				\draw[darkgray, thick, line cap=round] (0.5,0.5) rectangle ++(13,13);
				\foreach \y in {3,5,8,10} {
					\draw[darkgray, thick, line cap=round] (0.5, \y+0.5) -- ++(13, 0);
					\draw[darkgray, thick, line cap=round] (\y+0.5, 0.5) -- ++(0, 13);
				}
				\node at (0,0) {\phantom{1}};
			\end{tikzpicture}
		\end{tabular}
	\end{footnotesize}
\caption{On the left, the further restriction we define of the infinite zigzag word, $z_5$. On the right, the normalized permutation formed by breaking ties, $\zeta_5$.}
\label{fig-z5-zeta5}
\end{figure}

In the following sequence of results, we show that $\zeta_m$ is \emph{almost} universal. In fact, we show that $\zeta_m$ fails to be universal only for even $m$, and in that case, the only missing permutation is the decreasing permutation $m\cdots 21$. The first of these results, Proposition~\ref{prop-distant-inv-desc}, covers almost all permutations. (In fact, Proposition~\ref{prop-distant-inv-desc-layered} shows that Proposition~\ref{prop-distant-inv-desc} handles all but $2^{m-1}$ permutations of size $m$.)

We say that two entries $\pi(j)$ and $\pi(k)$ form an \emph{inverse-descent} if $j<k$ and $\pi(j)=\pi(k)+1$. (As the name is meant to indicate, if a pair of entries forms an inverse-descent in $\pi$, then the corresponding entries of $\pi^{-1}$ form a descent.) If $\pi(j)$ and $\pi(k)$ form an inverse-descent and they are not adjacent in $\pi$ (so $k\ge j+2$), then we say that they form a \emph{distant} inverse-descent.

\begin{proposition}
	\label{prop-distant-inv-desc}
	If the permutation $\pi$ of size $m$ has a distant inverse-descent, then $\zeta_m$ contains a subsequence order-isomorphic to $\pi$.
\end{proposition}
\begin{proof}
	Suppose that the entries $\pi(a)$ and $\pi(b)$ form a distant inverse-descent in $\pi$, meaning that $\pi(a)=\pi(b)+1$ and $b\ge a+2$. We define the word $p\in [m-1]^m$ by
	\[
		p(i)
		=
		\left\{\begin{array}{ll}
			\pi(i)   &\text{if $\pi(i)\le\pi(b)$,}\\
			\pi(i)-1 &\text{if $\pi(i)\ge\pi(a)=\pi(b)+1$.}
		\end{array}\right.
	\]

	The word $p$ has two occurrences of the letter $\pi(b)$, but because $\pi(a)$ and $\pi(b)$ form a distant inverse-descent, these two occurrences of $\pi(b)$ in $p$ do not constitute an immediate repetition. Thus Proposition~\ref{prop-miller-words} shows that either $p$ or $p^{+1}$ occurs as a subsequence in the first $n$ runs of the infinite zigzag word. As $p$ and $p^{+1}$ are both words over $[m]$, whichever of these words occurs in the first $n$ runs of the infinite zigzag word  also occurs as a subsequence of $z_m$. Suppose that this subsequence occurs in the indices $1\le i_1<i_2<\cdots<i_m\le \size{z_m}$, so $z_m(i_1)z_m(i_2)\cdots z_m(i_m)$ is equal to either $p$ or $p^{+1}$, and thus for $j,k \in [m]$ we have
	\[
		z_m(i_j) > z_m(i_k)
		\iff 
		p(j) > p(k).
	\]
	Because $z_m$ is order-homomorphic to $\zeta_m$, this implies that for all pairs of indices $j,k\in[m]$ except the pair $\{a,b\}$, we have
	\[
		\zeta_m(i_j) > \zeta_m(i_k)
		\iff 
		p(j) > p(k)
		\iff 
		\pi(j) > \pi(k).
	\]
	Furthermore, since $p(a)=p(b)$, we have $z_m(a)=z_m(b)$, and so by our construction of $\zeta_m$ it follows that $\zeta_m(a)>\zeta_m(b)$, while we know that $\pi(a)>\pi(b)$ because those entries form an inverse-descent. This verifies that $\zeta_m(i_1)\zeta_m(i_2)\cdots \zeta_m(i_m)$ is order-isomorphic to $\pi$, completing the proof.
\end{proof}

\begin{figure}[ht]
	\captionsetup{justification=centering}
		\begin{tikzpicture}[scale=0.2, baseline=(current bounding box.center)]
			\draw[thick] (0.5, 0.5)
				rectangle ++(2,2)
				rectangle ++(1,1)
				rectangle ++(3,3)
				rectangle ++(2,2);
			% The permutation:
			\begin{scope}[shift={(0,-1pt)}]
				\plotperm{2,1,3,6,5,4,8,7};
			\end{scope}
		\end{tikzpicture}
	\caption{The plot of the layered permutation $21\ 3\ 654\ 87$ with layer lengths $2$, $1$, $3$, $2$.}
	\label{fig-layered}
\end{figure}

To describe the permutations that Proposition~\ref{prop-distant-inv-desc} does not apply to, we need the notion of a layered permutation. A permutation is said to be \emph{layered} if it can be expressed as a sum of decreasing permutations, and in this case, these decreasing permutations are themselves called the \emph{layers}. An example of a layered permutation is shown in Figure~\ref{fig-layered}.

\begin{proposition}
	\label{prop-distant-inv-desc-layered}
	The permutation $\pi$ is layered if and only if it does not have a distant inverse-descent.
\end{proposition}
\begin{proof}
	One direction is completely trivial: if $\pi$ is layered then all of its inverse-descents are between consecutive entries, so it does not have a distant inverse-descent. For the other direction, we use induction on the size of $\pi$. The empty permutation is layered, so the base case holds. If $\pi$ is a nonempty permutation without distant inverse-descents, then it must begin with the entries $\pi(1)$, $\pi(1)-1$, $\dots$, $2$, $1$ in that order. This means that $\pi=\delta\directsum\sigma$ where $\delta$ is a nonempty decreasing permutation and $\sigma$ is a permutation smaller than $\pi$ that also does not have any distant inverse-descents. By induction, $\sigma$ is layered, and thus $\pi$ is as well, completing the proof.
\end{proof}

Having characterized the permutations to which Proposition~\ref{prop-distant-inv-desc} does not apply, we now show that almost all of them are nevertheless contained in $\zeta_m$.

\begin{proposition}
	\label{prop-layered-zeta}
	If the permutation $\pi$ of size $n$ is layered and not a decreasing permutation of even size, then $\zeta_m$ contains a subsequence order-isomorphic to $\pi$.
\end{proposition}
\begin{proof}
	Let $\pi$ denote an arbitrary layered permutation of size $n$. To prove the result, we compute the score of $\pi$ as in the proof of Proposition~\ref{prop-miller-words}, show that this score can only take on the values $0$ or $\pm 1$, and then describe an alternative embedding of $\pi$ in $\zeta_m$ in the case where the score of $\pi$ is $1$, except when $\pi$ is a decreasing permutation of even size.

	Recall that the score of any word $\pi$, $s(\pi)$, is defined as the number of initial runs of the infinite zigzag word necessary to contain $\pi$ minus the size of $\pi$. As observed in the proof of Proposition~\ref{prop-miller-words}, the score of a word does not change upon reading a letter of opposite parity. This implies that, while reading a layered permutation, the score changes only when transitioning from one layer to the next, and thus we compute the score of $\pi$ layer-by-layer.

	\renewcommand{\OE}{\textsf{odd}\text{--}\textsf{even}}
	\newcommand{\OO}{\textsf{odd}\text{--}\textsf{odd}}
	\newcommand{\EE}{\textsf{even}\text{--}\textsf{even}}
	\newcommand{\EO}{\textsf{even}\text{--}\textsf{odd}}

	\begin{figure}[ht]
	\captionsetup{justification=centering}
		\begin{footnotesize}
			\begin{tikzpicture}[
				scale=1, 
				xscale=3, 
				node style/.style={thick, draw, ellipse, minimum width=68pt, minimum height = 16.66666pt, align=center}
			]

				\draw (-1, 0) node[node style] (oe) {$\OE$};
				\draw ( 0,-1) node[node style] (ee) {$\EE$};
				\draw ( 0, 1) node[node style] (oo) {$\OO$};
				\draw ( 1, 0) node[node style] (eo) {$\EO$};
				
				\draw [->] (0,-{1.6}) to (0,-{1.333333});
				
				% \draw [->] (eo) to (oo);
				\draw [-] (+0.9, +0.4) to (+0.9,+0.8);
				\draw [domain=0:90] plot ({+0.833333+0.066667*cos(\x)}, {+0.8+0.2*sin(\x)});
				\draw [->] (+0.833333,+1.0) to (+0.425, +1.0);
				\node at (0.8,0.8) {$-1$};
				
				% \draw [->] (oo) to (oe);
				\draw [-] (-0.425, 1.0) to (-0.833333,1.0);
				\draw [domain=90:180] plot ({-0.833333+0.066667*cos(\x)}, {0.8+0.2*sin(\x)});
				\draw [->] (-0.9,0.8) to (-0.9, 0.4);
				\node at (-0.8,0.8) {$-1$};
				
				% \draw [->] (oe) to (ee);
				\draw [-] (-0.9, -0.4) to (-0.9,-0.8);
				\draw [domain=180:270] plot ({-0.833333+0.066667*cos(\x)}, {-0.8+0.2*sin(\x)});
				\draw [->] (-0.833333,-1.0) to (-0.5, -1.0);
				\node at (-0.8,-0.8) {$+1$};
				
				% \draw [->] (ee) to (eo);
				\draw [-] (+0.5, -1.0) to (+0.833333,-1.0);
				\draw [domain=270:360] plot ({+0.833333+0.066667*cos(\x)}, {-0.8+0.2*sin(\x)});
				\draw [->] (+0.9,-0.8) to (+0.9, -0.4);
				\node at (+0.8,-0.8) {$+1$};
				
				% \draw [->] (oe) to (oe);
				\newcommand\hmargin{0.0666667}
				\newcommand\vmargin{0.2}
				\newcommand\leftloopleft{-1.45}
				\newcommand\leftloopright{-1.166667}
				\newcommand\loopupper{0.45}
				\newcommand\looplower{-\loopupper}
				
				\draw [-] (\leftloopright+\hmargin, 0.4) to (\leftloopright+\hmargin,\loopupper);
				\draw [domain=  0: 90] plot ({\leftloopright+\hmargin*cos(\x)}, {\loopupper+\vmargin*sin(\x)}); % NE corner
				\draw [-] (\leftloopright, \loopupper+\vmargin) to (\leftloopleft,\loopupper+\vmargin);
				\draw [domain= 90:180] plot ({\leftloopleft+\hmargin*cos(\x)}, {\loopupper+\vmargin*sin(\x)}); % NW corner
				\draw [-] (\leftloopleft-\hmargin,\loopupper) to (\leftloopleft-\hmargin, \looplower);
				\draw [domain=180:270] plot ({\leftloopleft+\hmargin*cos(\x)}, {\looplower+\vmargin*sin(\x)}); % SW corner
				\draw [-] (\leftloopright, \looplower-\vmargin) to (\leftloopleft,\looplower-\vmargin);
				\draw [domain=270:360] plot ({\leftloopright+\hmargin*cos(\x)}, {\looplower+\vmargin*sin(\x)}); % SE corner
				\draw [->] (\leftloopright+\hmargin,\looplower) to (\leftloopright+\hmargin, -0.4);
				\node at (\leftloopleft,\loopupper) {$0$};
				
				% \draw [->] (eo) to (eo);
				\newcommand\rightloopright{-\leftloopleft}
				\newcommand\rightloopleft{-\leftloopright}

				\draw [-] (\rightloopleft-\hmargin,-0.4) to (\rightloopleft-\hmargin, \looplower);
				\draw [domain=180:270] plot ({\rightloopleft+\hmargin*cos(\x)}, {\looplower+\vmargin*sin(\x)}); % NE corner
				\draw [-] (\rightloopright, \looplower-\vmargin) to (\rightloopleft,\looplower-\vmargin);
				\draw [domain=270:360] plot ({\rightloopright+\hmargin*cos(\x)}, {\looplower+\vmargin*sin(\x)}); % NW corner
				\draw [-] (\rightloopright+\hmargin,\looplower) to (\rightloopright+\hmargin, \loopupper);
				\draw [domain=  0: 90] plot ({\rightloopright+\hmargin*cos(\x)}, {\loopupper+\vmargin*sin(\x)}); % SW corner
				\draw [-] (\rightloopright, \loopupper+\vmargin) to (\rightloopleft,\loopupper+\vmargin);
				\draw [domain= 90:180] plot ({\rightloopleft+\hmargin*cos(\x)}, {\loopupper+\vmargin*sin(\x)}); % SE corner
				\draw [->] (\rightloopleft-\hmargin, \loopupper) to (\rightloopleft-\hmargin,0.4);
				\node at (\rightloopright,\looplower) {\footnotesize$0$};


				\node at (-0.8,-0.8) {$+1$};

				% \draw [->] (oo) to (ee);
				\draw [->] (-0.1,0.575) to (-0.1,-0.625);
				\node at (-0.15,0) {$0$};
				
				% \draw [->] (ee) to (oo);
				\draw [->] (+0.1,-0.625) to (+0.1,+0.575);
				\node at (+0.15,0) {$0$};
				
			\end{tikzpicture}
		\end{footnotesize}
		\caption{A directed graph describing the scoring of a layered permutation.}
		\label{fig-layered-zeta-automaton}
	\end{figure}

	The change in score when moving from one layer of $\pi$ to the next is determined by the parity of the last entry of the layer we are leaving and the first entry of the layer we are entering. Specifically, the score changes by $-1$ if both of these entries are odd and $+1$ if both are even. This shows that in order to compute the score of the layered permutation $\pi$, we simply need to know the parities of the first and last entries of each of its layers. This information is represented by the labels of the nodes of the directed graph shown in Figure~\ref{fig-layered-zeta-automaton}.

	Moreover, not all transitions between these nodes are possible, because the last entry of a layer is precisely $1$ greater than the first entry of the preceding layer. This is why there are only eight edges shown in Figure~\ref{fig-layered-zeta-automaton}. In this figure, each of those edges is labeled by the change in the score function. Note that the first layer must end with $1$ (an odd entry), and its first entry must be either odd (for a score of $0$) or even (for a score of $1$); this is equivalent to starting our walk on the graph in Figure~\ref{fig-layered-zeta-automaton} at the node labeled $(\EE)$ before any layers are read.

	From this graphical interpretation of the scoring process, it is apparent that the score of a layered permutation can take on only three values: $-1$ if it ends at the node $(\OE)$; $0$ if it ends at either node $(\EE)$ or $(\OO)$; or $1$ if it ends at the node $(\EO)$. Except in this final case, we are done.

	Now suppose that we are in the final case, so the ultimate layer of $\pi$ is of $(\EO)$ type. The first entry of this layer is the greatest entry of $\pi$, so we know that $\pi$ has even size. If $\pi$ were a decreasing permutation then there would be nothing to prove (as we have not claimed anything in this case), so let us further suppose that $\pi$ is not a decreasing permutation, and thus that $\pi$ has at least two layers. We further divide this case into two cases. In both cases, as in the proof of Proposition~\ref{prop-distant-inv-desc}, we construct a word $p\in[m-1]^m$ such that if $z_m$ contains $p$, then $\zeta_m$ contains $\pi$.

	First, suppose that the penultimate layer of $\pi$ is of $(\EO)$ type and that this layer begins with the entry $\pi(b)$. This implies that the penultimate layer of $\pi$ has at least two entries (because its first and last entries have different parities). In this case, we define $p$ by
	\[
		p(i)
		=
		\left\{\begin{array}{ll}
			\pi(i)&\text{if $\pi(i)<\pi(b)$,}\\
			\pi(i)-1&\text{if $\pi(i)\ge\pi(b)$.}
		\end{array}\right.
	\]
	In other words, to form $p$ from $\pi$ we decrement the first entry of the penultimate layer and all entries of the ultimate layer. Because the penultimate layer of $\pi$ has at least two entries, performing this operation creates an immediate repetition (of the entry $\pi(b)-1$) at the beginning of this layer. For example, if $\pi=21\ 6543\ 87$ then $\pi(b)=6$ and we decrement the $6$, $8$, and $7$ to obtain the word $p=21\ 5543\ 76$.

	As with our previous constructions, if $z_m$ contains an occurrence of $p$, then $\zeta_m$ will contain a copy of $\pi$. We establish that $z_m$ contains $p$ by showing that $s(p)=0$, which requires a further bifurcation into subcases. In both subcases, the scoring of $p$ is computed by considering its score in the antepenultimate layer (the layer immediately before the penultimate layer), the score change when reading the newly decremented first entry of the penultimate layer, the score penalty of $+1$ because $p$ contains an immediate repetition (namely, $\pi(b)-1$ occurs twice in a row), and finally the score change between the penultimate and ultimate layers. We label these cases by the final three nodes of the directed graph from Figure~\ref{fig-layered-zeta-automaton} visited while computing the score of $\pi$.
	\begin{itemize}
		\item The final three layers are of type $(\EE)(\EO)(\EO)$. Note that this case includes the possibility that $\pi$ has only two layers. If $p$ has an antepenultimate layer, then the score while reading that layer is $0$ and the ascent between its last entry and the newly decremented first entry of the penultimate layer is of different parity (even to odd), contributing $0$ to the score. If $p$ does not have an antepenultimate layer, then $p$ begins with the newly decremented first entry of its penultimate layer, which contributes $0$ to the score. In either case, the score of $p$ is $0$ upon reading the first entry of the penultimate layer. The immediate repetition in the penultimate layer contributes $+1$ to the score, while the ascent between the last entry of the penultimate layer and the newly decremented first entry of the ultimate layer is odd and thus contributes $-1$, so $s(p) = 0$.
		\item The final three layers are of type $(\EO)(\EO)(\EO)$. The score while reading the antepenultimate layer is $+1$. The ascent between the last entry of the antepenultimate layer and the newly decremented first entry of the penultimate layer is odd, so it contributes $-1$ to the score, the immediate repetition in the penultimate layer contributes $+1$, and the ascent between the last entry of the penultimate layer and the newly decremented first entry of the ultimate layer is odd and thus contributes $-1$, so $s(p) = 0$.
	\end{itemize}

	It remains to treat the case where the penultimate layer is of $(\EE)$ type. Note that this case includes the possibility that the penultimate layer consists of a single entry. Suppose that the penultimate layer ends with the entry $\pi(a)$. We define $p$ by
	\[
		p(i)
		=
		\left\{\begin{array}{ll}
			\pi(i)   &\text{if $\pi(i)<\pi(a)$ or $\pi(i)=m$,}\\
			\pi(i)+1 &\text{if $\pi(i)\ge\pi(a)$ and $\pi(i)\neq m$.}
		\end{array}\right.
	\]
	Thus in forming $p$ from $\pi$ we increment all entries of the penultimate layer and all but the first entry of the ultimate layer. For example, if $\pi=21\ 3\ 654\ 87$, then we increment the $6$, $5$, $4$, and $7$ to obtain the word $p=21\ 3\ 765\ 88$.

	As before, if $z_m$ contains an occurrence of $p$ then $\zeta_m$ will contain a copy of $\pi$. Thus we need only show that $s(p)=0$, which we do, as in the previous case, by considering the scoring of the final three layers. As in that case, we identify two subcases.
	\begin{itemize}
		\item The final three layers are of type $(\OE)(\EE)(\EO)$. The score while reading the antepenultimate layer is $-1$. The ascent between the last entry of the antepenultimate layer and the newly incremented first entry of the penultimate layer is of different parity (even to odd) and thus contributes $0$ to the score. The ascent between the newly incremented last entry of the penultimate and the first entry of the ultimate layer (which is $m$) is of different parity (odd to even) and thus contributes $0$ to the score. Finally, the immediate repetition at the beginning of the ultimate layer (the two entries equal to $m$) contributes $+1$ to the score, so $s(p)=0$.
		\item The final three layers are of type $(\OO)(\EE)(\EO)$. The score while reading the antepenultimate layer is $0$. The ascent between the last entry of the antepenultimate layer and the newly incremented first entry of the penultimate layer contributes $-1$ to the score (as both entries are now odd). The ascent between the newly incremented last entry of the penultimate layer and the first entry of the ultimate layer (which is $n$) is of different parity (odd to even) and thus contributes $0$ to the score. Finally, the immediate repetition at the beginning of the ultimate layer contributes $+1$ to the score, so $s(p)=0$.
	\end{itemize}

	As we have considered all of the cases, the proof is complete.
\end{proof}

It remains only to conclude. The size of $\zeta_m$ is $(m^2+1)/2$ when $m$ is odd and $m^2/2$ when $m$ is even. When $m$ is odd, we have established that $\zeta_m$ is $m$-universal. However, Proposition~\ref{prop-layered-zeta} shows that $\zeta_m$ need not be universal when $m$ is even. (Indeed, it can be checked that $\zeta_m$ is \emph{not} $m$-universal when $m$ is even.) However, in this case, we know that $\zeta_m$ contains the decreasing permutation $(m-1)\cdots 21$ (for instance because it contains the permutation $(m-1)\cdots 21\oplus 1$). Thus we obtain an $m$-universal permutation by prepending a new maximum entry to $\zeta_m$, giving us the following bound.

\begin{theorem}[Engen and Vatter~\cite{engen:containing-all-:}]
\label{thm-perm-universal}
There is an $m$-universal permutation of size $\ceil{(m^2+1)/2}$.
\end{theorem}

A computer search reveals that the bound in Theorem~\ref{thm-perm-universal} is best-possible for $m \le 5$. Alas, for $m=6$ the $6$-universal permutation of Theorem~\ref{thm-perm-universal}'s construction has size $19$, but Arnar Arnarson [private communication] has found that the permutation
\[
	6\ 14\ 10\ 2\ 13\ 17\ 5\ 8\ 3\ 12\ 9\ 16\ 1\ 7\ 11\ 4\ 15
\]
of size $17$ is $6$-universal, and computations have shown that no smaller permutation suffices.

Arratia~\cite[Conjecture 2]{arratia:on-the-stanley-:} conjectured that the size of the smallest $m$-universal permutations is asymptotic to $m^2/e^2$. In~\cite{chroman:lower-bounds:}, Chroman, Kwan, and Singhal prove that any $m$-universal permutation must have size at least $(1.000076/e^2)m^2$ asymptotically, refuting Arratia's conjecture. In~\cite{eriksson:dense-packing-o:}, Eriksson, Eriksson, Linusson, and W\"{a}stlund conjecture that the smallest $m$-universal permutations have size asymptotic to $m^2/2$, and in~\cite{arratia:on-the-stanley-:}, Arratia presents Alon's conjecture that, asymptotically, most permutations of size $m^2/4$ are $m$-universal.

%%%%%%%%%%%%%%%%%%%%%%%%%%%%%%%%%%%%%%%%%%%%%%%%%%%%%%%%%%%%%%%%
\section{Proper Permutation Classes}
\label{sec-perm-proper}
%%%%%%%%%%%%%%%%%%%%%%%%%%%%%%%%%%%%%%%%%%%%%%%%%%%%%%%%%%%%%%%%

In this section, we consider the problem of universality for proper permutation classes. For reference, some known or computed values of minimum sizes for universal permutations for various classes are presented in Appendix~\ref{appendix-permutations}. Every class may be described as the set of permutations that avoid each element of a set of permutations, and the minimal such set is called the \emph{basis} of the class. If $\P$ is the set of permutations that avoid the set of permutations $S$, then we write $\P = \Av(S)$.

The pattern containment order is closely related to the induced subgraph order, a connection that frequently proves useful. If $\pi$ is a permutation of size $n$, define the \emph{inversion graph} of $\pi$, denoted $\g(\pi)$, to be the graph with vertex set $[n]$ and edges $i \sim j$ if and only if $i < j$ and $\pi(i) > \pi(j)$. Then, if $\pi$ contains $\tau$ as a pattern, then $\g(\pi)$ contains $\g(\tau)$ as an induced subgraph.

For any collection of permutations $\P$, let $\g(\P) = \{\g(\pi) \st \pi \in \P\}$. Suppose that $\G$ is a class of graphs with $\G = \g(\P)$ for some permutation class $\P$. If the permutation $\pi$ is $\P_m$-universal, then $\g(\pi)$ is $\G(\P)_m$-universal. Moreover, if $\pi$ is a proper $\P_m$-universal permutation, then $\g(\pi)$ lies in $\G$, so $\g(\pi)$ is a proper $\G_m$-universal graph. This shows that $\u_\G^p(m) \le \u_\P^p(m)$, giving us the following rhombus of inequalities.

\begin{center}
\begin{tabular}{CCC}
	\u_{\G}(m) & \le & \u_{\G}^{p}(m) \\
	  \rotle   &     &     \rotle     \\
	\u_{\P}(m) & \le & \u_{\P}^{p}(m)
\end{tabular}
\end{center}

%%%%%%%%%%%%%%%%%%%%%%%%%%%%%%%%%%%%%%%%%%%%%%%%%%%%%%%%%%%%%%%%
\subsection{Layered Permutations}
\label{sec-perm-layered}
%%%%%%%%%%%%%%%%%%%%%%%%%%%%%%%%%%%%%%%%%%%%%%%%%%%%%%%%%%%%%%%%

Recall that a permutation is layered if it is the direct sum of decreasing permutations. Equivalently, a permutation is layered if it avoid the patterns $231$ and $312$. Let $\Lay = \Av(231,312)$ denote the class of layered permutations. Proposition~\ref{prop-clusterize} shows that any graph may be transformed into a cluster graph of the same size that contains each cluster graph that the original graph contained. Similarly, the following proposition, which appears in~\cite{albert:universal-layer:}, shows that any permutation may be transformed into a layered permutation of the same size while retaining each of its layered containments.

\begin{proposition}[Albert, Engen, Pantone, and Vatter~\cite{albert:universal-layer:}]
\label{prop-layerize}
	Given any permutation $\pi$ of size $n$, there is a layered permutation of size $n$ that contains every layered permutation contained in $\pi$.
\end{proposition}
\begin{proof}
	We prove the claim by induction on $n$. Note that the base case is trivial, and let $\pi$ be a permutation of size $n \ge 1$. Let $D$ denote a decreasing subsequence of $\pi$ of maximum possible size. Because $D$ is a maximal decreasing subsequence, every entry of $\pi$ that is not in $D$ must either lie to the southwest of an entry of $D$ or to the northeast of such an entry, but not both. Let $D^-$ denote the set of entries that lie to the southwest of an entry of $D$ and let $D^+$ denote the set of entries that lie to the northeast of such an entry, so that $D$, $D^-$, and $D^+$ together constitute a partition of the entries of $\pi$. An example of this decomposition is shown on the leftmost panel of Figure~\ref{fig-layerization}.

	Define $\pi^-$ (resp., $\pi^+$) to be the permutation in the same relative order as the entries of $D^-$ (resp., $D^+$). Let $\delta$ be the decreasing permutation of size $\size{D}$, and define $\pi^\ast = \pi^-\directsum\delta\directsum\pi^+$. Thus, in some sense, $\pi^\ast$ is a ``straightened-out'' version of $\pi$, and an example is shown in the central panel of Figure~\ref{fig-layerization}.

	\begin{figure}[h]
	\captionsetup{justification=centering}
		\begin{center}
			\begin{tikzpicture}[scale=0.25, baseline=(current bounding box.center)]
				\draw [darkgray, very thick, fill=lightgray] (0.5,11.5) rectangle (4,10);
				\draw [darkgray, very thick, fill=lightgray] (4,10) rectangle (6,9);
				\draw [darkgray, very thick, fill=lightgray] (6,9) rectangle (8,8);
				\draw [darkgray, very thick, fill=lightgray] (8,8) rectangle (9,7);
				\draw [darkgray, very thick, fill=lightgray] (9,7) rectangle (11,2);
				\draw [darkgray, very thick, fill=lightgray] (11,2) rectangle (11.5,0.5);
				\plotperm{3, 5, 4, 10, 1, 9, 6, 8, 7, 11, 2};
				\draw [darkgray, very thick] (0.5,0.5) rectangle (11.5,11.5);
				\node at (5.5,4.5) {$D^-$};
				\node at (10.25,9.25) {$D^+$};
			\end{tikzpicture}
			\quad\quad\quad
			\begin{tikzpicture}[scale=0.25, baseline=(current bounding box.center)]
				\plotperm{2,4,3,5,1,10,9,8,7,6,11};
				\draw [darkgray, very thick] (0.5,0.5) rectangle (11.5,11.5);
				\draw [darkgray, very thick] (0.5,0.5) rectangle (5.5,5.5);
				\draw [darkgray, very thick] (5.5,5.5) rectangle (10.5,10.5);
				\draw [darkgray, very thick] (10.5,10.5) rectangle (11.5,11.5);
			\end{tikzpicture}
			\quad\quad\quad
			\begin{tikzpicture}[scale=0.25, baseline=(current bounding box.center)]
				\plotperm{1,4,3,2,5,10,9,8,7,6,11};
				\draw [darkgray, very thick] (0.5,0.5) rectangle ++(11,11);
				\draw [darkgray, very thick] (0.5,0.5) %
					rectangle ++(5,5) % 
					rectangle ++(5,5) %
					rectangle ++(1,1);
			\end{tikzpicture}
		\end{center}
	\caption{The steps in the proof of Proposition~\ref{prop-layerize}. From left to right, the drawings show an example of $\pi$, of $\pi^\ast$, and of the layered permutation $\tau^- \directsum \delta \directsum \tau^+$.}
	\label{fig-layerization}
	\end{figure}

	We claim that every layered permutation contained in $\pi$ is also contained in $\pi^\ast$. Suppose $\lambda=\lambda_{1}\directsum\cdots\directsum\lambda_\ell$ is a layered permutation contained in $\pi$, where each $\lambda_{i}$ is a decreasing permutation, and fix an embedding of $\lambda$ into $\pi$. Choose $j$ maximally so that in this embedding of $\lambda$ into $\pi$, the layers $\lambda_{1}\directsum\cdots\directsum\lambda_{j-1}$ are embedded entirely using entries in $D^-$. It follows that the entries of $\lambda_{j+1}\directsum\cdots\directsum\lambda_\ell$ are embedded entirely using entries of $D^+$. Since $\lambda_j$ certainly embeds into $D$ and consequently into $\delta$, we have that $\lambda \le \pi^\ast$.

	Finally, by induction we see that there are layered permutations $\mu^-$ and $\mu^+$ that contain all of the layered permutations contained in $\pi^-$ and $\pi^+$, respectively. It follows that $\mu^- \directsum \delta \directsum \mu^+$ is layered and contains all of the layered permutations contained in $\pi^\ast$, which in turn contains all of the layered permutations contained in $\pi$, proving the proposition. An example of this final construction is shown in the rightmost panel of Figure~\ref{fig-layerization}.
\end{proof}

In particular, any $\Lay_m$-universal permutation may be transformed into a layered permutation of the same size that is still $\Lay_m$-universal, affirming a conjecture of Gray~\cite{gray:bounds-on-super:}. Thus, among the smallest $\Lay_m$-universal permutations there are proper $\Lay_m$-universal permutations, establishing following corollary, which appears in~\cite{albert:universal-layer:}.
\begin{corollary}[Albert, Engen, Pantone, and Vatter~\cite{albert:universal-layer:}]
	For all integers $m \ge 0$, the smallest proper $\Lay_m$-universal permutations are also smallest $\Lay_m$-universal permutations.
\end{corollary}

We note that if $\lambda$ and $\mu$ are layered permutations with layer lengths $\lambda_1, \cdots, \lambda_m$ and $\mu_1, \cdots, \mu_n$ respectively, then we have $\lambda \le \mu$ if and only if there are indices $i_1 < i_2 < \cdots < i_m$ so that $\lambda_{j} \le \mu_{i_j}$ for all $j$. Therefore the poset of layered permutations is isomorphic to the poset of compositions under the generalized subword order discussed in Chapter~\ref{chap-compositions}. As in any question of universality for classes of layered permutations we may assume that the containing permutation is layered, we may reduce each such question to the corresponding question of universality for composition classes. Theorem~\ref{thm-comp-universal}, adapted from~\cite{albert:universal-layer:}, gives a precise formula for the size of the smallest $m$-universal compositions, and thus a formula for the size of the smallest $\Lay_m$-universal permutations.
\begin{corollary}[Albert, Engen, Pantone, and Vatter~\cite{albert:universal-layer:}]
\label{cor-perm-layered}
	For all integers $m \ge 0$, the smallest (proper) $\Lay_m$-universal permutations have size $\ell(m) = (m+1)\ceil{\log_2 (m+1)} - 2^{\ceil{\log_2 (m+1)}} + 1$.
\end{corollary}

Corollary~\ref{cor-perm-layered} improves upon lower and upper bounds of $\oOmega{m \log m}$ and $m \log_2 m + m$ respectively proven by Bannister, Cheng, Devanny, and Eppstein~\cite{bannister:superpatterns-a:} as well as lower and upper bounds of $m \log m - m$ and $m \floor{\log_2 m} + m$ respectively proven by Gray~\cite{gray:bounds-on-super:}.

% %%%%%%%%%%%%%%%%%%%%%%%%%%%%%%%%
% %%%%%%%%%%%%%%%%%%%%%%%%%%%%%%%%
% \subsection{A Constructive Upper Bound}
% \label{subsec-a-constructive-upper-bound}
% %%%%%%%%%%%%%%%%%%%%%%%%%%%%%%%%
% %%%%%%%%%%%%%%%%%%%%%%%%%%%%%%%%

The recursive construction used to prove Theorem~\ref{thm-comp-universal} can be repurposed to construct universal permutations for classes using direct sums. Let $\P$ be a permutation class and let $\P^{\sumind}$ denote the set of those permutations in $\P$ that are sum-indecomposable. For each permutation $\pi \in \P$, decompose $\pi$ as $\pi_{1} \directsum \cdots \directsum \pi_{m}$, where each $\pi_i$ is a nonempty sum-indecomposable permutation. Define $\comp(\pi)$ to be the composition $\size{\pi_{1}} \size{\pi_{2}} \cdots \size{\pi_{m}}$ and let $\comp(\P) = \{\comp(\pi) \st \pi \in \P\}$.

\begin{proposition}
\label{prop-an-upper-bound}
	Suppose that for each $m \le n$ the permutation $\tau_m$ is $\P^{\sumind}_m$-universal. If the composition $c = c(1) \cdots c(n)$ is $\comp(\P)_m$-universal, then the permutation $\tau_{c(1)} \directsum \tau_{c(2)} \cdots \directsum \tau_{c(n)}$ is $\P_m$-universal.
\end{proposition}
\begin{proof}
	Let $\pi \in \P_n$, and write $\pi = \pi_{1} \directsum \cdots \directsum \pi_{k}$, where each $\pi_{i}$ is a nonempty sum-indecomposable permutation. Then as $c$ is $\comp(\P)_n$-universal, there must be some sequence of indices $1 \le i_1 < i_2 < \cdots < i_k \le n$ such that $\size{\pi_{j}} \le c(i_j)$ for all $j$. As $\tau_{c(i_j)}$ is $c(i_j)$-universal, $\pi_{j}$ embeds into $\tau_{c(i_j)}$, and we have that $\tau_{c(1)} \directsum \tau_{c(2)} \cdots \directsum \tau_{c(n)}$ contains $\pi$, as desired.
\end{proof}

We now present a two brief applications of Proposition~\ref{prop-an-upper-bound}. As our first example, let $\Q = \Av(132, 231, 321) = \{\varepsilon\} \cup \{(1 \skewsum \iota_a) \directsum \iota_b \st a, b \ge 0\}$. Then, using vocabulary introduced in Chapter~\ref{chap-compositions}, $\comp(\P) = \Age(\omega 1^\omega)$ is the class of compositions in which every part after the first is $1$. The composition $m 1^{m-1}$ is $\Age(\omega 1^\omega)_m$-universal for each $m$, and the permutation $1 \skewsum \iota_{m-1}$ is $m$-universal for the set $\Q^{\sumind} = \{1 \skewsum \iota_a \st a \ge 0\}$ of sum-indecomposable permutations in $\Q$. Thus, by Proposition~\ref{prop-an-upper-bound}, the permutation $(1 \skewsum \iota_{m-1}) \directsum \iota_{m-1}$ is $\Q_m$-universal.
\begin{proposition}
	The permutation $(1 \skewsum \iota_{m-1}) \directsum \iota_{m-1}$ is $m$-universal for $\Av(132, 231, 321)$.
\end{proposition}

For our second example, let $\P = \Av(231, 321)$. Then $\P^{\sumind} = \{1 \skewsum \iota_k \st k \ge 0\}$ contains permutations of all sizes, and as $\P$ is closed under direct sums, $\comp(\P)$ is the class of all compositions. The composition $w_m$ defined by $w_0 = \varepsilon$ and, for $m \ge 1$,
\[
	w_m
	=
	w_{\floor{(m-1)/2}} m w_{\ceil{(m-1)/2}}
\]
of length $m$ and size $\ell(m)$ defined in Chapter~\ref{chap-compositions} is $m$-universal, and the permutation $\tau_m = 1 \skewsum \iota_{m-1}$ is $\P^{\sumind}_m$-unviersal. Thus the permutation $\sigma_m$ defined as $\sigma_0 = \varepsilon$ and for $m \ge 1$
\[
	\sigma_m
	=
	\sigma_{\floor{(m-1)/2}}
	\directsum
	\tau_m
	\directsum
	\sigma_{\ceil{(m-1)/2}}
\]
of size $\ell(m)$ is $\P_m$-universal by Proposition~\ref{prop-an-upper-bound}. As $\P$ is closed under direct sums and $\tau_m \in \P$ for each $m$, each $\sigma_m$ lies in $\P$ and is thus a proper $\P_m$-universal permutation.
\begin{proposition}
\label{prop-perm-231-321-univ-proper}
	Let $\sigma_0 = \varepsilon$ and, for $m \ge 1$,
	\[
		\sigma_m
		=
		\sigma_{\floor{(m-1)/2}}
		\directsum
		(m 1 2 \cdots (m-1))
		\directsum
		\sigma_{\ceil{(m-1)/2}}.
	\]
	Then $\sigma_m$ of size $\ell(m)$ is $m$-universal for $\Av(231, 321)_m$-universal for all $m$.
\end{proposition}

The permutations $\sigma_m$ are in fact asymptotically optimal, in the sense that their size is asymptotic to the size of the smallest proper $\P_m$-universal permutations as the next result shows.

\begin{theorem}
\label{thm-perm-231-321-proper}
	Let $\P = \Av(231, 321)$, and let $S = \{\pi \in \P \st \pi(1) \neq 1\}$. Let $s(m)$ denote the minimum size of an $S_m$-universal permutation in $\P$. Then $s(0) = s(1) = 0$, and for $m \ge 2$,
	\[
		s(m)
		\ge
		m + \min\{s(k) + s(m-k-1) \st 1 \le k \le m-1\}.
	\]
\end{theorem}
The proof of Theorem~\ref{thm-perm-231-321-proper} is similar to that of the lower bound of Theorem~\ref{thm-comp-universal}, so it may serve the reader well to ``warm up'' with that proof first.
\begin{proof}
	We proceed by induction on $m$. Each of $S_0$ and $S_1$ are empty, so the empty pattern is both $S_0$- and $S_1$-universal, and thus $s(0) = s(1) = 0$. 

	Let $m \ge 2$. Suppose that $\pi \in \P$ is an $S_m$-universal permutation and write $\pi = \pi_{1} \directsum \cdots \directsum \pi_\ell$, where each $\pi_j$ is sum-indecomposable. Then at least one $\pi_j$ has size at least $m$, as $\pi$ must contain the pattern $m 1 2 \cdots (m-1)$ and must do so entirely within one summand $\pi_j$. Suppose $\size{\pi_j} \ge m$, and choose $k \ge 1$ so that
	\[
		s(k)
		\le
		\size{\pi_{1} \directsum \cdots \directsum \pi_{j-1}}
		<
		s(k+1).
	\]
	As $\size{\pi_{1} \directsum \cdots \directsum \pi_{j-1}} < s(k+1)$, there must be some $\sigma \in S_k$ that does not embed into $\pi_{1} \directsum \cdots \directsum \pi_{j-1}$. Therefore the earliest that $\sigma$ may embed into $\pi$ is into $\pi_{1} \directsum \cdots \directsum \pi_{j}$. In particular, the earliest that the last summand of $\sigma$ may embed into $\pi$ is into $\pi_{j}$. 
	
	For any pair of sum-indecomposable permutations $\tau, \rho \in \P$, the permutation $\tau \directsum \rho$ necessarily contains the pattern $132$ as $\rho$ necessarily contains the pattern $12$. Thus $\pi_j$ must avoid $\tau \directsum \rho$, as every sum-indecomposable permutation in $\P$ is of the form $1 \skewsum \iota_n$ for some $n$, and thus must avoid the pattern $132$. Thus, if $\rho^\ast \in S_{m-k-1}$, then its first sum-component $\rho$ has size at least $2$ (as its first sum-component cannot by $1$ by definition,) and so any embedding of $\sigma \directsum \rho^\ast$ into $\pi$ embeds $\rho^\ast$ entirely within $\pi_{j+1} \directsum \cdots \directsum \pi_\ell$. As $\rho^\ast$ is an abitrary permutation in $S_{m-k-1}$, it follows that $\pi_{j+1} \directsum \cdots \directsum \pi_\ell$ is a $S_{m-k-1}$-universal permutation in $\P$, and thus $\size{\pi_{j+1} \directsum \cdots \directsum \pi_\ell} \ge s(m-k-1)$. Together, we have
	\begin{align*}
		\size{\pi}
			&= \size{\pi_{1} \directsum \cdots \directsum \pi_{j-1} \directsum \pi_j \directsum \pi_{j+1} \directsum \cdots \directsum \pi_\ell} \\
			&= \size{\pi_{1} \directsum \cdots \directsum \pi_{j-1}} + \size{\pi_j} + \size{\pi_{j+1} \directsum \cdots \directsum \pi_\ell} \\
			& \ge s(k) + m + s(m-k-1) \\
			& \ge m + \min\{s(k) + s(m-k-1) \st 1 \le k \le m-1\},
	\end{align*}
	as desired.
\end{proof}

Let $t(0) = t(1) = 0$ and $t(m) = m + \min\{t(k)+t(m-k-1) \st 1 \le k \le m-1\}$, then one can show that $t(m) \ge \ell(m) - m$ by induction, and Theorem~\ref{thm-perm-231-321-proper} implies that every proper $\P_m$-universal permutations has size at least $t(m)$. Together with the upper bound of $\ell(m)$ provided by Proposition~\ref{prop-perm-231-321-univ-proper}, we have have that the smallest proper $\P_m$-universal permutations have size between $\ell(m) - m$ and $\ell(m)$, meaning they have size asymptotic to $\ell(m) \sim m \log_2 m$.

Supported by computational evidence, we conjecture that the size of the smallest (proper) $\P_m$-universal permutations is in fact equal to $\ell(m)$.
\begin{conjecture}
	For all $m \ge 0$, the smallest (proper) $m$-universal permutations for $\Av(231, 321)$ have size $\ell(m)$.
\end{conjecture}

%%%%%%%%%%%%%%%%%%%%%%%%%%%%%%%%
%%%%%%%%%%%%%%%%%%%%%%%%%%%%%%%%
\subsection{Grid Classes}
\label{subsec-grid-classes}
%%%%%%%%%%%%%%%%%%%%%%%%%%%%%%%%
%%%%%%%%%%%%%%%%%%%%%%%%%%%%%%%%

One well-studied family of permutation classes are the monotone grid classes. Roughly speaking, the grid class of a matrix $M$ is the collection of permutations whose plots may be divided, in a manner prescribed by $M$, into blocks each containing a monotone pattern. More formally, let $M$ be a $0/ \pm 1$ matrix with $t$ columns and $u$ rows. An \emph{$M$-gridding} of the permutation $\pi$ of size $n$ is a choice of column divisions $0 = c_0 \le c_1 \le \dots \le c_t = n$ and row divisions $0 = r_0 \le r_1 \le \dots \le r_u = n$ such that for all $i$ and $j$, the subsequence of $\pi$ with indices in the real interval $(c_{i-1}, c_i]$ and values in the real interval $(r_{j-1}, r_j]$ is increasing if $M_{i,j} = 1$, decreasing if $M_{i,j} = -1$, and empty if $M_{i,j} = 0$. The \emph{grid class} of $M$, denote $\Grid{M}$, is the collection of all permutations that admit an $M$-gridding. An example of an $M$-gridding of a permutation for some matrix $M$ is drawn in Figure~\ref{fig-m-gridding}.

\begin{figure}
\captionsetup{justification=centering, margin=0.5in}
	\begin{tikzpicture}[scale={1/4}]
		\plotperm{2, 4, 12, 11, 1, 3, 5, 6, 8, 7, 9, 10}
		\draw[draw=gray!75, very thick] (0.5,0.5) rectangle ++(12,12);
		\draw[draw=gray!75, very thick] (2.5,0.5) -- ++(0,12);
		\draw[draw=gray!75, very thick] (9.5,0.5) -- ++(0,12);
		\draw[draw=gray!75, very thick] (0.5,6.5) -- ++(12,0);
	\end{tikzpicture}
\caption{A $\left(\begin{smallmatrix} 0 & -1 & 1\\ 1 & 1 & 0 \end{smallmatrix}\right)$-gridding of a permutation $\pi$. As $\pi$ admits such a gridding, we write $\pi \in \gridVertThree{1,0}{1,-1}{0,1}$.}
\label{fig-m-gridding}
\end{figure}

To aid comprehension, we denote grid classes by their \emph{cell diagrams} rather than by their matrices. For example, we abbreviate $\Grid{\begin{smallmatrix} 0 & -1 & 1\\ 1 & 1 & 0 \end{smallmatrix}}$ as $\gridVertThree[{1/3}]{1,0}{1,-1}{0,1}$. Occasionally, where it is convenient, we use a cell diagram to stand for the matrix itself.

We begin the results portion of this subsection with an observation useful for providing lower bounds on the size of proper universal permutations for grid classes.

\begin{observation}
\label{obs-grid-points}
	Let $\P = \Grid{M}$ be a grid class, and let $\sigma, \pi \in \P$. If in every $M$-gridding of $\sigma$ there are $k$ points in cell $M_{i,j}$ and $\sigma \le \pi$, then in every $M$-gridding of $\pi$ there are at least $k$ points in the cell $M_{i,j}$.
\end{observation}

As one simple example where Observation~\ref{obs-grid-points} is useful, let $M = \gridHorizTwo[{1/3}]{1}{1}$ and consider the grid class $\P = \Grid{M}$. In the unique $M$-gridding of the permutation $m 12 \cdots (m-1)$, the lower cell of $M$ contains $m-1$ points, and thus any $M$-gridding of any proper $\P_m$-universal permutation must place at least $m-1$ points the lower cell of $M$. Likewise, the unique $M$-gridding of the permutation $2 \cdots m1$ places $m-1$ points in the upper cell of $M$, and thus any $M$-gridding of any proper $\P_m$-universal permutation must place at least $m-1$ points in the upper cell of $M$. Together, these observations show that any proper $\P_m$-universal permutation must have at least $2(m-1)$ points, which is nearly optimal as we will see shortly in Theorem~\ref{thm-perm-riffle}.

%%%%%%%%%%%%%%%%%%%%%%%%%%%%%%%%
%%%%%%%%%%%%%%%%%%%%%%%%%%%%%%%%
\subsection{Wedge Permutations and Riffle Shuffle Permutations}
\label{subsec-wedgle-riffle}
%%%%%%%%%%%%%%%%%%%%%%%%%%%%%%%%
%%%%%%%%%%%%%%%%%%%%%%%%%%%%%%%%

Up to symmetry, the two simplest non-trivial grid classes are $\gridVertOne[{1/3}]{-1,-1}$ and $\gridVertOne[{1/3}]{-1,1}$. In this section, we show that both classes have smallest (proper) $m$-universal permutations of size $2m-1$. To begin, let $\W = \Av(132, 312) = \gridVertOne[{1/3}]{1,-1}$ denote the class of non-empty \emph{wedge} permutations. In~\cite{bannister:superpatterns-a:}, Bannister, Cheng, Devanny, and Eppstein prove that $\W$ has smallest (proper) universal permutations of size $2m-1$, and below we provide a proof of this fact. To begin, we show that the poset of nonempty wedge permutations is isomorphic to the poset of words over a two-letter alphabet, discussed in generality in Section~\ref{sec-words}.

\begin{proposition}
\label{prop-perm-wedge-isomorphism}
Define the map $P: \{p, m\}^\ast \to \W$ recursively as follows:
\begin{enumerate}
	\item $P(\varepsilon) = 1$ is the permutation of size $1$.
	\item $P(vp) = P(v) \directsum 1$ is the direct sum of $P(v)$ with the pattern $1$.
	\item $P(vm) = P(v) \skewsum 1$ is the skew sum of $P(v)$ with the pattern $1$.
\end{enumerate}
Then $P$ is a poset isomorphism.
\end{proposition}

The map $P$ sends words of size $m$ to permutations of size $m+1$, so we omit the $0$-pattern from $\W$. Before proving Proposition~\ref{prop-perm-wedge-isomorphism}, we present a useful lemma.
\begin{lemma}
\label{lemma-perm-sum}
If $\pi$ and $\sigma$ are permutations, then
\begin{enumerate}
	\item $\pi \directsum 1 \le \sigma \directsum 1$ if and only if $\pi              \le \sigma$.
	\item $\pi \directsum 1 \le \sigma \skewsum   1$ if and only if $\pi \directsum 1 \le \sigma$.
	\item $\pi \skewsum   1 \le \sigma \directsum 1$ if and only if $\pi \skewsum   1 \le \sigma$.
	\item $\pi \skewsum   1 \le \sigma \skewsum   1$ if and only if $\pi              \le \sigma$.
\end{enumerate}
\end{lemma}

We now present the proof of Proposition~\ref{prop-perm-wedge-isomorphism}.

\newenvironment{proof-of-prop-perm-wedge-isomorphism}{%
	\medskip\noindent {\it Proof of Proposition~\ref{prop-perm-wedge-isomorphism}.\/}%
}{%
	\qed\bigskip%
}
\begin{proof-of-prop-perm-wedge-isomorphism}
	We begin by proving that if $v, w \in \{p, m\}^\ast$ with $v \le w$, then $P(v) \le P(w)$.

	Leveraging the recursive nature of the map $P$, we proceed by induction on the size of $v$. For our base case, note that if $v$ is the empty word, then $P(v) = 1$, which is contained in every non-empty permutation. Thus, assume that $v$ is a non-empty word, let $\ell$ be the final letter of $v$, and write $v = v_0 \ell$. Let $w \in \{p, m\}^\ast$ be any word with $v \le w$. As $v$ is a subsequence of $w$, we may write $w = w_0 \ell v_1$, where $v_0 \le w_0$. By induction, we have $P(v_0) \le P(w_0)$. By Lemma~\ref{lemma-perm-sum}, we have $P(v_0\ell) \le P(w_0\ell)$ no matter the value of $\ell$. Finally, as $P(w_0\ell) \le P(w_0\ell w_1)$, we have
	\[
		P(v) 
		=
		P(v_0\ell)
		\le
		P(w_0\ell)
		\le
		P(w_0\ell w_1)
		=
		P(w),
	\]
	as desired.

	For the converse, we show that if $P(v)$ and $P(w)$ are permutations with $P(v) \le P(w)$, then $v \le w$. Again, we proceed by induction on the size of $v$ and begin by noting that the permutation $P(\varepsilon) = 1$ is contained in every permutation in $\W$, and the empty word $\varepsilon$ is contained in every word in $\{p, m\}^\ast$, so the base case is satisfied.

	Let $P(v)$ and $P(w)$ be permutations with $P(v) \le P(w)$ and $\size{P(v)} \ge 2$, or equivalently, $\size{v} \ge 1$. Without loss of generality, assume that $v = v_0 p$ and $w = w_0 p m^k$ for some $k \ge 0$. Repeated applications of Lemma~\ref{lemma-perm-sum} mean that $P(v_0 p) \le P(w_0 p m^k)$ implies that $P(v_0) \le P(w_0)$. By induction we must have $v_0 \le w_0$, and thus $v_0 p \le w_0 p m^k$, completing the proof.
\end{proof-of-prop-perm-wedge-isomorphism}

One consequence of this isomorphism is that the smallest universal words constructed in Section~\ref{sec-words} translate into smallest proper universal permutations for $\W$.
\begin{corollary}
\label{cor-perm-wedge-proper}
	The smallest proper $\W_m$-universal permutations have size $2m-1$.
\end{corollary}

As $\W_m$ contains both $12 \cdots m$ and $m \cdots 21$, any $\W_m$-universal permutation must contain a length-$m$ increasing subsequence as well as a length-$m$ decreasing subsequence. As these subsequences may not intersect in more than one entry, any $\W_m$-universal permutation must have size at least $2m-1$, so no smaller $\W_m$-universal permutation exists outside $\W$.

\begin{corollary}
\label{cor-perm-wedge-improper}
	The smallest (proper) $\W_m$-universal permutations have size $2m-1$.
\end{corollary}

The other $2 \times 1$ grid class is $\R = \Av(123, 3412, 3142) = \gridVertOne{-1,-1}$, the class of \emph{riffle shuffle} permutations, so called as they are precisely those patterns that may be formed by a single riffle shuffle of a deck of $n$ cards. In~\cite{bannister:small-superpatt:}, Bannister, Devanny, and Eppstein prove that the permutation $(m+1)1(m+2)2\cdots(2m-1)(m-1) \in \R$ is $\R_m$-universal. This construction may be seen to be optimal, as we now show. Consider the layered permutation class $\P = \gridVertTwo[{1/3}]{-1,0}{0,-1}$. By Proposition~\ref{prop-layerize}, the minimum size of a $\P_m$-universal permutation is the same as the minimum size of an $m$-universal composition for the composition class $\Age(\omega^2)$, which is $2m-1$ by Proposition~\ref{prop-comp-length-2-improper}. As $\P \subseteq \R$, the smallest $\R_m$-universal permutation must have size at least $2m-1$, and thus the construction by Bannister, Devanny, and Eppstein is optimal.

\begin{theorem}
\label{thm-perm-riffle}
	Let $\R = \gridVertOne[{1/3}]{-1,-1}$. The smallest (proper) $\R_m$-universal permutations have size $2m-1$ for $m \ge 1$.
\end{theorem}

%%%%%%%%%%%%%%%%%%%%%%%%%%%%%%%%
%%%%%%%%%%%%%%%%%%%%%%%%%%%%%%%%
\subsection{\texorpdfstring{$\{123, 312\}$}{(123, 312)}-avoiding Permutations}
%%%%%%%%%%%%%%%%%%%%%%%%%%%%%%%%
%%%%%%%%%%%%%%%%%%%%%%%%%%%%%%%%

One immediate consequence of Theorem~\ref{thm-perm-riffle} is that any class of permutations that contains $\gridVertTwo[{1/3}]{-1,0}{0,-1}$ and is contained in $\gridVertOne[{1/3}]{-1,-1}$ must have smallest $m$-universal permutations of size $2m-1$. The class $\gridVertThree[{2/9}]{0,-1,0}{0,0,-1}{-1,0,0}$ meets this criteria, and thus the corollary below follows.
\begin{corollary}
\label{cor-perm-123-312-improper}
	Let $\P = \Av(123, 312) = \gridVertThree[{2/9}]{0,-1,0}{0,0,-1}{-1,0,0}$. For each $m \ge 1$, the smallest $\P_m$-universal permutations have size $2m-1$.
\end{corollary}

The size of the smallest proper $\Av(123, 312)_m$-universal permutations is slightly larger, as the following theorem shows.
\begin{theorem}
	Let $\P = \Av(123, 312) = \gridVertThree[{2/9}]{0,-1,0}{0,0,-1}{-1,0,0}$. For each $m \ge 3$, the smallest proper $\P_m$-universal permutations have size $3m-4$.
\end{theorem}
\begin{proof}
	To begin, we claim that the permutation $\pi = (\delta_{m-1} \directsum \delta_{m-1}) \skewsum \delta_{m-2}$ is $\P_m$-universal. Let $\sigma \in \P_m$. If $\sigma$ avoids $12$, then $\sigma$ is the decresing permutation of size $m$, and as $\pi$ contains $\delta_{m-2} \skewsum \delta_{m-1} = \delta_{2m-3}$ and $2m-3 \ge m$, we have that $\pi$ contains $\sigma$. Otherwise, $\sigma$ contains $12$, and we may decompose $\sigma$ as $\sigma = (\delta_a \directsum \delta_b) \skewsum \delta_c$ for nonnegative integers $a, b, c$ with $a, b \ge 1$ and $a + b + c = m$. In this case, both $a$ and $b$ must be at most $m-1$, and $c$ must be at most $m-2$, which implies that $\sigma$ is contained in $(\delta_{m-1} \directsum \delta_{m-1}) \skewsum \delta_{m-2} = \pi$.

	Let $M$ be the matrix $\gridVertThree[{2/9}]{0,-1,0}{0,0,-1}{-1,0,0}$. To establish a lower bound, we show that in any $M$-gridding of a proper $\P_m$-universal permutation, the lower-right cell must have at least $m-2$ points and both the leftmost cell and topmost cell must each have at least $m-1$ points.

	Consider the permutations $\sigma_1 = 1 \directsum \delta_{m-1}$, $\sigma_2 = \delta_{m-1} \directsum 1$, and $\sigma_3 = 12 \skewsum \delta_{m-2}$.
	\begin{enumerate}
		\item The permutation $\sigma_1 = 1 m \cdots 2$ has a unique $M$-gridding, where the $m-1$ entries $2, 3, \dots, m$ are placed in topmost cell and the entry $1$ is placed in the leftmost cell. 
		\item The permutation $\sigma_2 = (m-1)\cdots 2m$ has a unique $M$-gridding, where the entry $m$ is placed in the topmost cell and the $m-1$ entries $1, 2, \dots, (m-1)$ are placed in the leftmost cell. 
		\item The permutation $\sigma_3 = (m-1)m(m-2) \cdots 21$ has a unique gridding in $M$, where the $m-2$ entries $1, 2, \dots, (m-2)$ are placed in the lower-right cell, the entries $m-1$ and $m$ are placed in the leftmost and topmost cells, respectively.
	\end{enumerate}
	By Observation~\ref{obs-grid-points}, any $M$-gridding of any proper $\P_m$-universal permutation must contain at least $m-1$ points in the topmost cell, at least $m-1$ points in the leftmost cell, and at least $m-2$ points in the lower-left cell, and therefore must have size at least $(m-2) + (m-1) + (m-1) = 3m-4$, as desired.
\end{proof}

%%%%%%%%%%%%%%%%%%%%%%%%%%%%%%%%
%%%%%%%%%%%%%%%%%%%%%%%%%%%%%%%%
\subsection{\texorpdfstring{$\{231, 2143\}$}{231, 2143}-avoiding Permutations}
%%%%%%%%%%%%%%%%%%%%%%%%%%%%%%%%
%%%%%%%%%%%%%%%%%%%%%%%%%%%%%%%%

Let $\P = \Av(231, 2143) = \gridVertTwo[{1/3}]{1,-1}{0,1}$ be the class of $\{231, 2143\}$-avoiding permutations. In~\cite{bannister:superpatterns-a:}, Bannister, Cheng, Devanny, and Eppstein prove that there are $\P_m$-universal permutations of size $3m-4$ for $m \ge 3$. We provide a simple proof of this construction.

\begin{theorem}[Bannister, Cheng, Devanny, and Eppstein~\cite{bannister:superpatterns-a:}]
\label{thm-perm-231-2143-proper}
	For each $m \ge 3$, there is a proper $\P_m$-universal permutation of size $3m-4$.
\end{theorem}
\begin{proof}
	Let $\upsilon_3 = 51324$, and for each $m > 3$, let $\upsilon_{m+1} = 1 \skewsum (1 \directsum \upsilon_{m} \directsum 1)$. We claim that, for each $m \ge 3$, $\upsilon_m$ is (1) in $\P$ and (2) $\P_m$-universal. To begin, note that $\upsilon_3$ is in $\P$ and that for any permutation $\pi \in \P$, the permutations $1 \skewsum \pi$, $1 \directsum \pi$, and $\pi \directsum 1$ all lie in $\P$. As $\upsilon_{m+1}$ is constructed from $\upsilon_m$ using only these operations, we have that $\upsilon_m \in \P$ for all $m \ge 3$ by induction.

	To show that $\upsilon_m$ is $m$-universal for all $m \ge 3$, we begin by noting that $\upsilon_3 = 51324$ is $\P_3$-universal, meaning it contains the pattern $123$, $132$, $213$, $312$ and $321$, which we may check by hand. Given any permutation $\pi \in \P$, we may decompose $\pi$ as at least one of $\pi = 1 \directsum \pi^\ast$, $\pi = 1 \skewsum \pi^\ast$, or $\pi = \pi^\ast \directsum 1$, where $\pi^\ast \in \P_{m-1}$. As $\upsilon_m$ contains each of $1 \directsum \upsilon_{m-1}$, $1 \skewsum \upsilon_{m-1}$, and $\upsilon_{m-1} \directsum 1$, and $\upsilon_{m-1}$ contains $\pi^\ast$ necessarily by induction, we have that $\upsilon_m$ contains $\pi$, completing the proof.
\end{proof}

Moreover, Bannister, Cheng, Devanny, and Eppstein prove that, with $\Q = \Av(231, 312, 2143, 1324) = \gridHorizTwo[{1/3}]{1,0}{0,-1} \cup \gridVertTwo[{1/3}]{-1,0}{0,1}$, any $\Q_m$-universal permutation has size at least $3m-4$. As $\Q \subseteq \P$, any $\P_m$-universal permutation must have size at least $3m-4$, which together with the $\P_m$-universal permutations of size $3m-4$ of Theorem~\ref{thm-perm-231-2143-proper} are the best-possible even outside the class.
\begin{theorem}[Bannister, Cheng, Devanny, and Eppstein~\cite{bannister:superpatterns-a:}]
\label{thm-perm-231-2143-improper}
	The smallest $m$-universal permutations for $\Av(231, 2143)$ have size $3m-4$.
\end{theorem}

As a corollary, Bannister, Cheng, Devanny, and Eppstein note that any class contained in $\P$ that contains $\Q$ has smallest universal permutations of this size as well:
\begin{corollary}[Bannister, Cheng, Devanny, and Eppstein~\cite{bannister:superpatterns-a:}]
	If $\R$ is a permutation class with $\Q \subseteq \R \subseteq \P$, then the smallest $\R_m$-universal permutations have size $3m-4$.
\end{corollary}

%%%%%%%%%%%%%%%%%%%%%%%%%%%%%%
%%%%%%%%%%%%%%%%%%%%%%%%%%%%%%
\subsection{\texorpdfstring{$231$}{231}-avoiding Permutations}
%%%%%%%%%%%%%%%%%%%%%%%%%%%%%%
%%%%%%%%%%%%%%%%%%%%%%%%%%%%%%

In~\cite{knuth:the-art-of-comp:1}, Knuth shows that the permutations that may be sorted with a stack are precisely those that avoid the permutation $231$. Let $\P = \Av(231)$ denote the class of $231$-avoiding permutations. In~\cite{bannister:superpatterns-a:}, Bannister, Cheng, Devanny, and Eppstein construct $\P_m$-universal permutations of size $\floor{m^2/4} + m$ and construct, for every proper subclass of $\P$, $m$-universal permutations of size $\oO{m \log^{\oO{1}}m}$. The $\P_{10}$-universal permutation of their construction is presented in the left panel of Figure~\ref{fig-perm-231-univ}.

To present a construction of proper $\P_m$-universal permutations, we say the \emph{reverse-inverse-reverse} of a permutation $\pi$, denoted $\pi^{\rir}$, is defined as $\pi^{\rir} = ((\pi^{r})^{-1})^{r}$, where $\pi^r$ is the permutation formed by reversing $\pi$, and $\pi^{-1}$ is the function inverse of $\pi$. Pictorially, a plot of $\pi^{\rir}$ may be obtained by reflecting the plot of $\pi$ across the line $y = -x$. The reverse and inverse symmetries preserve containment, so the reverse-inverse-reverse symmetry does as well, meaning that $\sigma \le \pi$ if and only if $\sigma^{\rir} \le \pi^{\rir}$. As $231 = 231^{\rir}$, the class $\P$ is closed under the reverse-inverse-reverse symmetry, and thus $\pi$ is $\P_m$-universal if and only if $\pi^{\rir}$ is.

\begin{proposition}
\label{prop-perm-231-proper}
Let $\Pi_0 = \varepsilon$, and for $m \ge 1$, define
\[
	\Pi_m
	=
	\left(1 \skewsum \Pi_{m-1}^{\rir}\right) \directsum \Pi_{\floor{m/2}}.
\]
Then $\Pi_{m}$ is a proper $\P_m$-universal permutation for all $m$.
\end{proposition}

\begin{figure}
\captionsetup{justification=centering}
	\begin{tikzpicture}[scale={6/35}]
		\draw (0.25,0.35) rectangle ++(35.5, 35.5);
		\plotperm{34,23,14,7,2,1,33,22,13,6,3,32,21,12,5,4,31,20,11,8,30,19,10,9,29,18,15,28,17,16,27,24,26,25,35}

		\begin{scope}[shift={(40,0)}]
			\draw (0.25,0.35) rectangle ++(35.5, 35.5);
			\plotperm{26,1,5,3,2,4,25,19,7,6,8,18,14,9,13,11,10,12,16,15,17,20,24,22,21,23,32,27,31,29,28,30,34,33,35}
		\end{scope}
	\end{tikzpicture}
\caption{On the left, the $\P_{10}$-universal permutation of size $35$ of Bannister, Chung, Devanny, and Eppstein's construction~\cite{bannister:superpatterns-a:}. On the right, the proper $\P_8$-universal permutation $\Pi_{8}$ of size $35$.}
\label{fig-perm-231-univ}
\end{figure}

\begin{proof}
	Before showing that $\Pi_{m}$ is $\P_m$-universal for each $m$, we show that each $\P_m$ lies in $\P$ by a brief inductive arguemnt. The permutation $\Pi_0 = \varepsilon$ lies in every class of permutations, so our base case is satisfied. In addition to being closed under the reverse-inverse-reverse symmetry, the class $\P$ is closed under prepending a new largest element and under direct sums, ie. if $\pi, \sigma \in \P$, then both $1 \skewsum \pi$ and $\pi \directsum \sigma$ lie in $\P$. Thus, by induction, $\Pi_{m} = (1 \skewsum \Pi_{m-1}^{\rir}) \directsum \Pi_{\floor{m/2}} \in \P$ for all $m$.

	To show that $\Pi_{m}$ is $\P_m$-universal for all $m$, we again proceed by induction on $m$. The permutation $\Pi_0 = \varepsilon$ is clearly $0$-universal for every class, so assume that $\Pi_{m'}$ is $m'$-universal for $\P$ for all $m' < m$.

	Let $\pi \in \P_m$. If $\pi$ is sum-indecomposable, then $\pi = 1 \skewsum \tau$ for some $\tau \in \P_{m-1}$. As $\Pi_{m-1}$ is $\P_{m-1}$-universal, $\Pi^{\rir}_{m-1}$ is as well. Thus, as $\Pi_m$ contains $1 \skewsum \Pi^{\rir}_{m-1}$ and $\pi = 1 \skewsum \sigma$ for $\sigma \in \P_{m-1}$, we have that $\pi \le \Pi_m$. 
	
	Otherwise, we may write $\pi = \pi_{1} \directsum \pi_{2}$, where $\pi_{2}$ is a non-empty sum-indecomposable permutation, and we complete our analysis in two cases: when $\size{\pi_{2}} \le \floor{m/2}$ and when $\size{\pi_{2}} \ge \floor{m/2} + 1$. In the first case, assume that $\size{\pi_{2}} \le \floor{m/2}$. As $\size{\pi_{1}} \le m-1$, we have that $\pi_{1}$ embeds into $\Pi_{m-1}^{\rir}$, and as $\size{\pi_{2}} \le \floor{m/2}$, we have that $\pi_{2}$ embeds into $\Pi_{\floor{m/2}}$. Since $\Pi_{m}$ contains $\Pi_{m-1}^{\rir} \directsum \Pi_{\floor{m/2}}$, we may conclude that $\pi = \pi_{1} \directsum \pi_{2} \le \Pi_{m}$. In the second case, assume that $\size{\pi_{2}} \ge \floor{m/2}+1$, and thus $\size{\pi_{1}} \le \ceil{m/2}-1$. We claim that $\pi$ embeds into $\Pi_{m-1}^{\rir} = \Pi_{k-1}^{\rir} \directsum (1 \skewsum \Pi_{m-2})$. As $\pi_{2}$ is sum-indecomposable, we may write $\pi_{2} = 1 \skewsum \tau$ for some $\tau$ with $\size{\tau} \le m-2$. Thus we have $\pi = \pi_{1} \directsum (1 \skewsum \tau)$, where $\size{\pi_{1}} \le k-1$ and $\size{\tau} \le 2k-2$, and so by induction, $\pi$ embeds into $\Pi_{k-1}^{\rir} \directsum (1 \skewsum \Pi_{2k-2}) = \Pi_{2k-1}^{\rir}$, completing the proof.
\end{proof}

\begin{corollary}
\label{cor-231-ub-formula}
The sequence $\size{\Pi_{m}} = p(m)$ is given by $p(0) = 0$ and, for $m \ge 1$,
\[
	p(m) 
	= 
	p(m-1) + p\left(\floor{m/2}\right) + 1,
\]
and thus the smallest proper $\P_m$-universal permutations have size at most $p(m)$.
\end{corollary}

The proper $\P_m$-universal permutations constructed in Proposition~\ref{prop-perm-231-proper} are much larger than the $\P_m$-universal permutations of size $\floor{m^2/4} + m$ constructed by Bannister, Cheng, Devanny, and Eppstein in~\cite{bannister:superpatterns-a:}, as the next result shows.

\begin{theorem}[Knuth~\cite{knuth:an-almost-linear:}]
\label{thm-almost-linear}
Let $a(1) = 1$ and $a(n) = a(n-1) + a(\floor{n/2})$ for $n \ge 2$. Then $a(n)$ grows faster than any polynomial. That is, for any power $k$, there is some $N_k$ such that for all $n \ge N_k$ we have $a(n) > n^k$.
\end{theorem}

Before presenting the proof of Theorem~\ref{thm-almost-linear}, note that $s(n) \ge a(n)$, and thus $s$ grows faster than any polynomial.

\newenvironment{proof-of-thm-almost-linear}{%
	\medskip\noindent {\it Proof of Proposition~\ref{thm-almost-linear}.\/}%
}{%
	\qed\bigskip%
}
\begin{proof-of-thm-almost-linear}
	Fix $k$. We aim to show that for sufficiently large $n$, we have $a(n) > n^k$. Let $N$ be such that 
	\[
		2^{k+1} + 1 \ge \left(2 + \frac{1}{N}\right)^{k+1}
	\]
	and let
	\[
		c 
		=
		\min\left\{\frac{a(n)}{n^{k+1}} \st N \le n \le 2N\right\}.
	\]
	We claim that for all $n \ge N$, we have $a(n) \ge cn^{k+1}$. For $N \le n \le 2N$, this is clear as $c \le a(n)/n^{k+1}$. If $n > 2N$, induction shows that
	\begin{align*}
		a(n)
			&= a(n-1) + a(\floor{n/2}) \\
			&\ge c\cdot(n-1)^{k+1} + c\cdot\floor{n/2}^{k+1} \\
			&\ge c\left((n-1)^{k+1} + \left(\frac{n-1}{2}\right)^{k+1}\right) \\
			% &=   c\left( 1 + 1/2^{k+1}             \right)     (n-1)^{k+1} \\
			&=   c\left( \frac{2^{k+1}+1}{2^{k+1}} \right)     (n-1)^{k+1} \\
			&\ge c\left( \frac{2+\frac{1}{N}}{2} \right)^{k+1} (n-1)^{k+1} \\
			&=   c\left( 1+\frac{1}{2N}          \right)^{k+1} (n-1)^{k+1} \\
			&\ge c\left( 1+\frac{1}{n-1}         \right)^{k+1} (n-1)^{k+1} \\
			&=   c n^{k+1}.
	\end{align*}
	
	To conclude, choose $N_k$ such that $N_k \ge N$ and $N_k \ge 1/c$ (so that $cn \ge c N_k \ge 1$), and the proof is complete.
\end{proof-of-thm-almost-linear}

%%%%%%%%%%%%%%%%%%%%%%%%%%%%%%%%
%%%%%%%%%%%%%%%%%%%%%%%%%%%%%%%%
\subsection{\texorpdfstring{$321$}{321}-avoiding Permutations}
%%%%%%%%%%%%%%%%%%%%%%%%%%%%%%%%
%%%%%%%%%%%%%%%%%%%%%%%%%%%%%%%%

Let $\S$ be the class of $321$-avoiding permutations. Equivalently, $\S$ is the class of permutations that may be partitioned into two increasing subsequences. In~\cite{bannister:small-superpatt:}, Bannister, Devanny, and Eppstein construct $\S_m$-universal permutations of size $22m^{3/2} + \oTheta{m}$.

In~\cite{Atminas:Universal-graph:}, Atminas, Kitaev, Lozin, and Valyuzhenich construct a proper $\S_m$-universal permutation of size $m^2$. The corresponding permutation graph is thus a proper $\g(\S)_m$-universal graph, where in this case $\g(\S)$ is the class of bipartite permutation graphs. This corresponding $\g(\S)_m$-universal permutation graph is in fact equal to the size-$m^2$ graph that Lozin and Rudolf construct in \cite{lozin:minimal-univers:}. The question of whether $\S$ admits sub-quadratic sized proper universal permutations remains open, but evidence points towards ``no''. In~\cite{alecu:critical-properties:}, Alecu, Lozin, and Malyshev prove that any proper $m$-universal graph for the class of bipartite permutation graphs must have size $\oOmega{m^\alpha}$ for all $\alpha < 2$, implying that any proper $\S_m$-universal permutation must have size $\oOmega{m^\alpha}$ for all $\alpha < 2$. They conjecture that the smallest $m$-universal graphs for the class of bipartite permutation graphs have size $\oOmega{m^2}$, which would imply that the smallest proper $\S_m$-universal permutations have size $\oTheta{m^2}$.

%%%%%%%%%%%%%%%%
%%%%%%%%%%%%%%%%
%%%%%%%%%%%%%%%%
%%%%%%%%%%%%%%%%
\subsection{Subclasses of \texorpdfstring{$\Av(321)$}{Av(321)}: Truncated Staircases}
%%%%%%%%%%%%%%%%
%%%%%%%%%%%%%%%%
%%%%%%%%%%%%%%%%
%%%%%%%%%%%%%%%%

The class $\Av(321)$ may be represented as the grid class of an infinite matrix. First shown by Albert, Atkinson, Brignall, Ru\v{s}kuc, Smith, and West~\cite{albert:growth-rates-fo:}, we have
\[
	\Av(321)
	=
	\begin{tikzpicture}[grids]

		\foreach \x in {0, 1, 2, 3, 4} {
			\draw [thin, gray] (\x,0) -- ++(0,3.9);
		}

		\foreach \y in {0, 1, 2, 3} {
			\draw [thin, gray] (0,\y) -- ++(4.9,0);
		}

		\draw (0,0) rectangle ++(1,1);
		\draw[ultra thick, shorten <=4pt, shorten >=4pt] (0,0) -- ++(1,1);
		\draw (1,0) rectangle ++(1,1);
		\draw[ultra thick, shorten <=4pt, shorten >=4pt] (1,0) -- ++(1,1);
		\draw (1,1) rectangle ++(1,1);
		\draw[ultra thick, shorten <=4pt, shorten >=4pt] (1,1) -- ++(1,1);
		\draw (2,1) rectangle ++(1,1);
		\draw[ultra thick, shorten <=4pt, shorten >=4pt] (2,1) -- ++(1,1);
		\draw (2,2) rectangle ++(1,1);
		\draw[ultra thick, shorten <=4pt, shorten >=4pt] (2,2) -- ++(1,1);
		\draw (3,2) rectangle ++(1,1);
		\draw[ultra thick, shorten <=4pt, shorten >=4pt] (3,2) -- ++(1,1);
		\foreach \d in {0.25, 0.50, 0.75} {
			\filldraw[black] ({3+\d}, {3+\d}) circle [radius=0.03cm];
			\filldraw[black] ({4+\d}, {3+\d}) circle [radius=0.03cm];
		}
	\end{tikzpicture}
\]

The figure on the right is known as the infinite \emph{positive staircase}. One natural family of subclasses of $\Av(321)$ are the finite positive staircases. Let $\S^k$ be the positive staircase consisting of the first $k$ nonempty cells of the infinite staircase, e.g, $\S^1 = \gridCell[{1/3}]{1}$, $\S^2 = \gridHorizOne[{1/3}]{1,1}$, $\S^3 = \gridHorizTwo[{1/3}]{1,1}{0,1}$, and so on.

Then, as 
\[
	\S^1
	\subseteq
	\S^2
	\subseteq
	\S^3
	\subseteq
	\cdots
	\subseteq
	\S,
\]
we have
\[
	\u_{\S^1}(m)
	\le
	\u_{\S^2}(m)
	\le
	\u_{\S^3}(m)
	\le
	\cdots
	\le
	\u_{\S}(m).
\]

For all $m$, every permutation in $\S_m$ may be gridded using the first $\ceil{(m+1)/2}$ cells of the infinite positive, and there is some permutation in $\S_m$ that cannot be gridded using $\ceil{(m-1)/2}$ cells. Thus, any proper $\S_m$-universal permutation may not use fewer than the first $\ceil{(m+1)/2}$ cells, and computation suggests that among all smallest proper $\S_m$-universal permutations, there is at least one that may itself be gridded in $\ceil{(m+1)/2}$ cells. 
\begin{conjecture}
\label{conj-no-more-cells}
For all $m \ge 0$, there is a smallest proper $\S_m$-universal permutation that lies in $\S^{\ceil{(m+1)/2}}$.
\end{conjecture}

We now demonstrate, through Observation~\ref{obs-grid-points}, a lower bound on the size of proper $\S^{k}_m$-universal permutations. For $k \ge 2$, let $M$ be the matrix corresponding to the the finite positive staircase with $k$ nonempty cells. The permutation $\pi = 21 43 \cdots (2k-2)(2k-3)$ has a unique $M$-gridding, placing one point in each of the first and last cells and two points in every other cell. Consider some entry $\pi(i)$ of $\pi$ that is placed in the cell $C$ in the unique $M$-gridding of $\pi$. Replacing the entry $\pi(i)$ of $\pi$ with a contiguous increasing sequence of entries of size $m-2k+3$ and yields a permutation of size $m$ that has a unique $M$-gridding, placing $m-2k+3$ points in the cell $C$ if $C$ is the first or last cell of $M$ or $m-2k+4$ points in the cell $C$ otherwise. 

Thus for $m \ge 2k-2$, any $M$-gridding of a proper $\S^k_m$-universal permutation must have at least $m-2k+3$ points in its first and last cells, and $m-2k+4$ points in each of its other $k-2$ cells. This shows that any proper $\S^k_m$-universal permutation must have size at least $2(m-2k+3) + (k-2)(m-2k+4) = km - 2(k-1)^2$.

\begin{proposition}
\label{prop-cells-proper-lower}
	For $m \ge 2k-2$, any proper $\S^k_m$-universal permutation has size at least $km - 2(k-1)^2$.
\end{proposition}

Using Conjecture~\ref{conj-no-more-cells} and Proposition~\ref{prop-cells-proper-lower}, we have the conjollary\footnote{A \emph{conjollary} is a corollary that follows from a conjecture.} below.
\begin{conjollary}
	For all $m \ge 4$, any proper $\S_m$-universal permutation must have size at least $m^2/8$.
\end{conjollary}
\begin{proof}
	By Conjecture~\ref{conj-no-more-cells}, we have that the size of the smallest proper $\S_m$-universal permutations and the size of the smallest proper $\S^k_m$-universal permutations are the same for all $k \le (m+2)/2$. By Proposition~\ref{prop-cells-proper-lower}, every proper $\S^k_m$ universal permutation has size at least $km - 2(k-1)^2$, and thus
	\begin{align*}
		\u_{\S}^p(m) 
			&\ge \max\{mk - 2(k-1)^2 \st k \le (m+2)/2\} \\
			&\ge m \floor{\frac{m}{4}+1} - 2 \floor{\frac{m}{4}}^2 \\
			&> \frac{m^2}{8},
	\end{align*}
	as desired.
\end{proof}

%%%%%%%%%%%%%%%%%%%%%%%%%%%%%%%%
%%%%%%%%%%%%%%%%%%%%%%%%%%%%%%%%
\subsection{Skew riffle Permutations}
%%%%%%%%%%%%%%%%%%%%%%%%%%%%%%%%
%%%%%%%%%%%%%%%%%%%%%%%%%%%%%%%%

We say that $\pi$ is a \emph{skew riffle} permutation if it is the sum of riffle and antiriffle permutations. Let $\SR$ be the class of skew riffle permutations:
\[
	\P
	=
	\bigdirectsum 
	\left(
		\gridVertTwoDisplay{1}{1}
		\cup 
		\gridVertOneDisplay{1,1}
	\right).
\]

In~\cite{bannister:small-superpatt:}, Bannister, Devanny, and Eppstein construct $\P_m$-universal permutations of size $16m \log m + \oTheta{m}$. Let $\R = \gridVertTwo[{1/3}]{1}{1} \cup \gridVertOne[{1/3}]{1,1}$. The permutation 
\[
	\pi_m 
	= 
	1 3 5 \cdots (2m-1) 2 (2m) 4 (2m+1) 6 \cdots (3m-3) (2m-2) (3m-2)
\]
of size $3m-2$ is $\R_m$-universal, as the lowest $2m-1$ values form the permutation $1 3 5 \cdots (2m-1) 2 4 6 \cdots (2m-2)$, which is $m$-universal for $\gridVertTwo[{1/3}]{1}{1}$, and the rightmost $2m-1$ values form the permutation $m 1 (m+1) 2 \cdots (m-1) (2m-1)$, which is $m$-universal for $\gridVertOne[{1/3}]{1,1}$. In the language of Proposition~\ref{prop-an-upper-bound}, $\comp(\R)$ is the class of all compositions, and the recursively-defined composition $w_m = w_{\floor{(m-1)/2}} m w_{\ceil{(m-1)/2}}$ (with $w_0 = \varepsilon$) of length $m$ and size $\ell(m)$ is $m$-universal. Thus the permutation $\tau_m = \directsum_{i = 1}^{m} \pi_{w_m(i)}$ is $\R_m$-universal and has size at most $3\ell(m) \sim 3m \log_2 m$. For example, Figure~\ref{fig-perm-construction} shows the plots of $\pi_4$ and $\tau_3$.

\begin{figure}
\captionsetup{justification=centering}
	\begin{tikzpicture}[scale=0.25]
		\plotpermborder{1,3,5,7,2,8,4,9,6,10}
		\begin{scope}[shift={(12,0.5)}]
			\draw[thick, gray] (0.5,0.5) rectangle ++(1,1);
			\draw[thick, gray] (1.5,1.5) rectangle ++(7,7);
			\draw[thick, gray] (8.5,8.5) rectangle ++(1,1);
			\plotpermborder{1,2,4,6,3,7,5,8,9}
		\end{scope}
	\end{tikzpicture}
\caption{On the left, the permutation $\pi_4$, which is $4$-universal for $\gridVertTwo[{1/3}]{1}{1} \cup \gridVertOne[{1/3}]{1,1}$. On the right, the $\P_3$-universal permutation $\tau_3$ with its summands $\pi_1$, $\pi_3$, and $\pi_1$ outlined.}
\label{fig-perm-construction}
\end{figure}

%%%%%%%%%%%%%%%%%%%%%%%%%%%%%%%%
%%%%%%%%%%%%%%%%%%%%%%%%%%%%%%%%
\section{Concluding Remarks}
%%%%%%%%%%%%%%%%%%%%%%%%%%%%%%%%
%%%%%%%%%%%%%%%%%%%%%%%%%%%%%%%%

Unlike in the context of graphs, there is no clear guess about the ``optimal'' size of an $m$-universal permutation for a class based on the number of $m$-patterns in the class. Both the class of wedge permutations and the class of layered permutations contain precisely $2^{m-1}$ patterns of size $m$, but their smallest (proper) universal permutations have sizes $\ell(m) \sim m \log_2 m$ and $2m-1$, respectively. One immediately desirable characterization would be of those maximal permutation classes that admit linear-size universal permutations. 
\begin{question}
	What are the maximal permutations classes that admits linear-size universal permutations?
\end{question}

\chapter{Conclusion}
\label{chap-conclusion}

In this chapter, we present some additional problems related to universality. As remarked in Chapter~\ref{chap-introduction}, each $n$-structure considered in this work contains at most $\binom{n}{m}$ many $m$-structures, as any substructure of size $m$ corresponds to a subset of entries of size $m$. One may ask more generally, ``what is the maximum number of $m$-structures that an $n$-structure may contain?'' In~\cite{bevan:prolific-permut:}, Bevan, Homberger, and Tenner prove there are $n$-permutations that attain the above binomial bound, meaning they contain $\binom{n}{m}$ many $m$-permutations, if and only if $n \ge \ceil{(n-m)^2/2 + 2(n-m) + 1}$.  Focusing on permutations, let $\Cont(n,m)$ denote the maximum number of $m$-permutations that any $n$-permutation contains. Figure~\ref{fig-perm-max-contain} displays known values of $\Cont(m+k,m)$ for small $m$ and $k$. The cells with dark gray backgrounds contain values where $\Cont(m+k,m) = m!$, meaning there are $m$-universal permutations of size $m+k$. The cells with white backgrounds contain values so that $\Cont(m+k,m) = \binom{m+k}{m}$. We are grateful to Axel Bacher~[private communication] for computing many of these values.

\begin{figure}[ht]
\captionsetup{justification=centering}
    \begin{tikzpicture}[xscale = 1.0, yscale={2/3}]
        \def\height{11}
        \def\width{13}

        \draw[line cap = round] (-0.5,-0.5) -- (0.5,0.5);
        \node at (0.25,-0.25) {$\scriptstyle m$};
        \node at (-0.25,0.25) {$\scriptstyle k$};

        \foreach \x in {1,2,...,\width} {
            \node at (\x, 0) {$\scriptstyle \x$};
        }
        \foreach \y in {1,2,...,\height} {
            \node at (0, \y) {$\scriptstyle \y$};
        }
        \begin{scope}
            \clip (0.5,0.5) rectangle ++(\width,\height);
            \fill[fill=gray!20] (0.5,0.5) rectangle ++(\width,\height);
            
            % DARK GRAY
            \path[fill=gray!80] (0.5,0.5)
                -| ++(2,1)
                -| ++(1,3)
                -| ++(1,3)
                -| ++(1,3)
                -| ++(1,6)
                -| cycle;
            
            % WHITE
            % Bevan, Homberger, Tenner
            \path[fill=white] ( 0.5, 0.5)
                |- ++(2,0)
                |- ++(2,1)
                |- ++(4,1)
                |- ++(4,1)
                |- ++(6,1)
                |- cycle;

            \draw[shift={(0.5,0.5)}, black] (0,0) grid (\width,\height);

            % Known values:
            \foreach \val [count=\x] in {1, 2, 4,  5,   6,   7,    8,     9,    10,    11,    12,    13,   14,   15} {\node at (\x, 1) {$\scriptstyle\val$}; }
			\foreach \val [count=\x] in {1, 2, 6, 12,  21,  28,   36,    45,    55,    66,    78,    91,  105,  120} {\node at (\x, 2) {$\scriptstyle\val$}; }
			\foreach \val [count=\x] in {1, 2, 6, 19,  41,  76,  114,   162,   220,   286,   364,   455,  560,  680} {\node at (\x, 3) {$\scriptstyle\val$}; }
			\foreach \val [count=\x] in {1, 2, 6, 23,  71, 156,  291,   477,   699,   988,  1355,  1815, 2380, 3060} {\node at (\x, 4) {$\scriptstyle\val$}; }
			\foreach \val [count=\x] in {1, 2, 6, 24,  94, 273,  614,  1127,  1867,  2885,  4282,  6150, 8531,     } {\node at (\x, 5) {$\scriptstyle\val$}; }
			\foreach \val [count=\x] in {1, 2, 6, 24, 112, 408, 1094,  2356,  4368,  7405, 11944, 18109,     ,     } {\node at (\x, 6) {$\scriptstyle\val$}; }
			\foreach \val [count=\x] in {1, 2, 6, 24, 119, 526, 1728,  4402,  9070, 17150, 29422,      ,     ,     } {\node at (\x, 7) {$\scriptstyle\val$}; }
			\foreach \val [count=\x] in {1, 2, 6, 24, 120, 618, 2484,  7320, 17514, 35849,      ,      ,     ,     } {\node at (\x, 8) {$\scriptstyle\val$}; }
			\foreach \val [count=\x] in {1, 2, 6, 24, 120, 683, 3212, 11391, 30728,      ,      ,      ,     ,     } {\node at (\x, 9) {$\scriptstyle\val$}; }
			\foreach \val [count=\x] in {1, 2, 6, 24, 120, 710, 3931, 16215,      ,      ,      ,      ,     ,     } {\node at (\x,10) {$\scriptstyle\val$}; }
			\foreach \val [count=\x] in {1, 2, 6, 24, 120, 720, 4383,      ,      ,      ,      ,      ,     ,     } {\node at (\x,11) {$\scriptstyle\val$}; }
			\foreach \val [count=\x] in {1, 2, 6, 24, 120, 720,     ,      ,      ,      ,      ,      ,     ,     } {\node at (\x,12) {$\scriptstyle\val$}; }
			\foreach \val [count=\x] in {1, 2, 6, 24, 120, 720,     ,      ,      ,      ,      ,      ,     ,     } {\node at (\x,13) {$\scriptstyle\val$}; }
			\foreach \val [count=\x] in {1, 2, 6, 24, 120, 720,     ,      ,      ,      ,      ,      ,     ,     } {\node at (\x,14) {$\scriptstyle\val$}; }
        \end{scope}
    \end{tikzpicture}
\caption{An array of known values of $\Cont(m+k,m)$ for small values of $m, k$.}
\label{fig-perm-max-contain}
\end{figure}

Given a set of structures $\S$, and let $\Cont_{\S}(n, m)$ denote the maximum number of $\S_m$-structures that any $n$-structure may contain. Likewise, let $\Cont_{\S}^p(n, m)$ denote the maximum number of $\S_m$-structures that any $\S_n$-structure may contain. As a case study, consider the poset $\T$ of words over the two-letter alphabet $\{\worda, \wordb\}$ under the subword ordering. In this case, we may provide an exact formula for the maximum number of words of size $m$ that any word in $\T_n$ may contain.

\begin{proposition}
\label{prop-word-max-contain}
For all $n, m$, the maximum number of $m$-words contained in a $\T_n$-word is
\[
    \Cont_{\T}(n,m) 
    = 
    \sum_{\ell = 0}^{n-m} \binom{m}{\ell}.
\]
\end{proposition}
Before proving Proposition~\ref{prop-word-max-contain}, we note that the formula provided equals $2^{m}$ precisely when $n \ge 2m$, which is implied by Theorem~\ref{prop-word-universal}.
\begin{proof}
To begin, we obtain an recursive formula for the number of $m$-words contained in a word $w$ of size $n$ for $n > m$. If $w$ contains only $\worda$ or only $\wordb$, then $w$ contains only the constant word of size $m$. Otherwise, $w$ contains both $\worda$ and $\wordb$, and without loss of generality we can assume that $w$ ends in $\worda$ and write $w = w_0 \wordb \worda^k$. 

For each word $v$ that $w$ contains that ends in $\wordb$, there is some embedding of $v$ into $w$ that places the final letter of $v$ into the final $\wordb$ of $w$, and thus the set of $m$-words contained in $w$ that end in $\wordb$ is in bijection with the set of $(m-1)$-words contained in $w_0$. Likewise, the set of $m$-words contained in $w$ that end in $\worda$ is in bijection with the set of $(m-1)$-words contained in $w_0 \wordb \worda^{k-1}$. % Let $\cont_w(m)$ denote the number of $m$-words contained in $w$. 
We have shown that 
\[
    \binom{\text{$\#$ of $m$-words}}{\text{contained in $w$}}
    =
    \binom{\text{$\#$ of $(m-1)$-words}}{\text{contained in $w_0$}}
    +
    \binom{\text{$\#$ of $(m-1)$-words}}{\text{contained in $w_0 \wordb \worda^{k-1}$}}.
\]
% \[
%     \cont_{w}(m) 
%     = 
%     \cont_{w_0}(m-1) + \cont_{w_0 \wordb \worda^{k-1}}(m-1).
% \]

This formula implies an upper bound on the quantity $\Cont_{\T}(n,m)$. As $w_0$ has size at most $n-2$ and $w_0 \wordb \worda^{k-1}$ has size at most $n-1$, we have
\begin{align*}
    \binom{\text{$\#$ of $m$-words}}{\text{contained in $w$}}
        &= \binom{\text{$\#$ of $(m-1)$-words}}{\text{contained in $w_0$}}
         + \binom{\text{$\#$ of $(m-1)$-words}}{\text{contained in $w_0 \wordb \worda^{k-1}$}} \\
        &\le \Cont_{\T}(n-2,m-1) + \Cont_{\T}(n-1,m-1)
\end{align*}
for all words $w$. Thus the maximum number of $m$-words contained among all words in $\T_n$ obeys this bound as well, and thus we have
\[
    \Cont_{\T}(n,m)
    \le
    \Cont_{\T}(n-2,m-1) + \Cont_{\T}(n-1,m-1).
\]

To establish equality, we exhibit a word $w_n$ of size $n$ that attains this upper bound. For each $n \ge 0$, let $w_n$ be the word of size $n$ wherein each letter in an odd position is $\worda$ and each letter in an even position is $\wordb$, e.g. $w_0 = \varepsilon$, $w_1 = \worda$, $w_2 = \worda \wordb$, $w_3 = \worda \wordb \worda$, and so on. Let $\cont(n, m)$ denote the number of $m$-words contained in $w_n$.

We claim that for all $n$, $\cont(n, m) = \Cont_{\T}(n, m)$, and proceed by induction on $m$. For our base case of $m = 0$, every word contains the unique empty word, so $\cont(n, 0) = \Cont_{\T}(n, 0) = 1$. Now, assume that for all $n$ we have $\cont(n, m') = \Cont_{\T}(n, m')$ for all $m' < m$. By the recursive formula at the beginning of this proof, we have
\begin{align*}
    \cont(n, m)
        &= \cont(n-1, m-1) + \cont(n-2, m-1) \\
        &= \Cont_{\T}(n-1, m-1) + \Cont_{\T}(n-2, m-1) \\
        &\ge \Cont_{\T}(n,m).
\end{align*}
As $w_n$ is a word of size $n$, we have that $\cont(n,m) \le \Cont_{\T}(n,m)$, and thus we have equality:
\[
    \cont(n, m)
    =
    \Cont_{\T}(n,m)
    =
    \Cont_{\T}(n-1,m-1) + \Cont_{\T}(n-1,m-1).
\]

It only remains to show that for all $n$,
\[
    \Cont_{\T}(n,m)
    = 
    \sum_{\ell = 0}^{n-m} \binom{m}{\ell},
\]
which we show by induction on $m$.

For all $n$, we have $\Cont_{\T}(n,0) = \binom{0}{0} + \binom{0}{1} + \cdots + \binom{0}{n-m} = 1 + 0 + \cdots + 0 = 1$. Fix $m \ge 1$, and assume that the statement holds for all values less than $m$. Then we have
\begin{align*}
    \Cont_{\T}(n,m) 
        &= \Cont_{\T}(n-1,m-1) + \Cont_{\T}(n-2,m-1) \\
        &= \sum_{\ell = 0}^{n-m} \binom{m-1}{\ell} + \sum_{\ell = 0}^{n-m-1} \binom{m-1}{\ell} \\
        &= \binom{m-1}{0} + \sum_{\ell = 0}^{n-m-1} \left[ \binom{m-1}{\ell+1} + \binom{m-1}{\ell} \right] \\
        &= \binom{m}{0} + \sum_{\ell = 0}^{n-m-1} \binom{m}{\ell+1} \\
        &= \binom{m}{0} + \sum_{\ell = 1}^{n-m} \binom{m}{\ell} \\
        &= \sum_{\ell = 0}^{n-m} \binom{m}{\ell},
\end{align*}
as desired.
\end{proof}

Values of $\Cont_{\T}(m+k,m)$ for small values of $m$ and $k$ are displayed in Figure~\ref{fig-max-contain-words}. The recurrence proven in the proof of Proposition~\ref{prop-word-max-contain} may be interpreted in this context as ``the value in a cell is equal to the sum of the values in the cells immediately to its west and to its south-west''. As in Figure~\ref{fig-perm-max-contain}, the cells with dark gray backgrounds correspond to universality, and the cells with white backgrounds contain values so that $\Cont_{\T}(m+k,m) = \binom{m+k}{m}$.

\begin{figure}[ht]
\captionsetup{justification=centering}
\begin{tikzpicture}[xscale=1, yscale={2/3}]
    \def\height{10}
    \def\width{10}
    
    \draw[line cap = round] (-0.5,-0.5) -- (0.5,0.5);
    \node at (0.25,-0.25) {$\scriptstyle m$};
    \node at (-0.25,0.25) {$\scriptstyle k$};
    
    \foreach \x in {1,2,...,\width} {
        \node at (\x, 0) {$\scriptstyle \x$};
    }
    \foreach \y in {1,2,...,\height} {
        \node at (0, \y) {$\scriptstyle \y$};
    }
    
    \begin{scope}
        \clip (0.5,0.5) rectangle ++(\width,\height);
        \fill[fill=gray!20] (0.5,0.5) rectangle ++(\width,\height);
        
        % White
        \path[fill=white] ( 0.5, 0.5)
                |- ++(10,1) |- cycle;
        
        % Dark gray
        \path[fill=gray!80] (0.5,1.5)
                -| ++(1,0)
                -| ++(1,1) -| ++(1,1) -| ++(1,1) -| ++(1,1)
                -| ++(1,1) -| ++(1,1) -| ++(1,1) -| ++(1,1)
                -| ++(1,1) -| cycle;
        
        % Hatched dark gray
        \path[pattern = north east lines, pattern color = gray!80] (0.5, 0.5) 
            |- ++(1,1)
            |- cycle;
    
        \draw[shift={(0.5,0.5)}, black] (0,0) grid (\width,\height);

        % \foreach \val [count=\x] in { 1, 1,  1,  1,  1,  1,  1,   1,   1,   1} {\node at ( 0,\x) {$\scriptstyle\val$};}
        \foreach \val [count=\x] in { 2, 2,  2,  2,  2,  2,  2,   2,   2,   2} {\node at ( 1,\x) {$\scriptstyle\val$};}
        \foreach \val [count=\x] in { 3, 4,  4,  4,  4,  4,  4,   4,   4,   4} {\node at ( 2,\x) {$\scriptstyle\val$};}
        \foreach \val [count=\x] in { 4, 7,  8,  8,  8,  8,  8,   8,   8,   8} {\node at ( 3,\x) {$\scriptstyle\val$};}
        \foreach \val [count=\x] in { 5,11, 15, 16, 16, 16, 16,  16,  16,  16} {\node at ( 4,\x) {$\scriptstyle\val$};}
        \foreach \val [count=\x] in { 6,16, 26, 31, 32, 32, 32,  32,  32,  32} {\node at ( 5,\x) {$\scriptstyle\val$};}
        \foreach \val [count=\x] in { 7,22, 42, 57, 63, 64, 64,  64,  64,  64} {\node at ( 6,\x) {$\scriptstyle\val$};}
        \foreach \val [count=\x] in { 8,29, 64, 99,120,127,128, 128, 128, 128} {\node at ( 7,\x) {$\scriptstyle\val$};}
        \foreach \val [count=\x] in { 9,37, 93,163,219,247,255, 256, 256, 256} {\node at ( 8,\x) {$\scriptstyle\val$};}
        \foreach \val [count=\x] in {10,46,130,256,382,466,502, 511, 512, 512} {\node at ( 9,\x) {$\scriptstyle\val$};}
        \foreach \val [count=\x] in {11,56,176,386,638,848,968,1013,1023,1024} {\node at (10,\x) {$\scriptstyle\val$};}
        
    \end{scope}
    
\end{tikzpicture}
\caption{An array containing the values $\Cont_{\T}(m+k, m)$ for small values of $m$ and $k$.}
\label{fig-max-contain-words}
\end{figure}

By the isomorphism between the poset of words over a two-letter alphabet and the poset $\W$ of non-empty wedge permutations in Proposition~\ref{prop-perm-wedge-isomorphism}, the maximum number of $\W_m$-permutations that any $\W_n$-permutation may contain is $\sum_{\ell = 0}^{n-m} \binom{m-1}{\ell}$. Computation suggests that the maximum number of $\W_m$-permutations contained by \emph{any} permutation of size $n$ is $\sum_{\ell = 0}^{n-m} \binom{m-1}{\ell}$ as well.

\begin{conjecture}
\label{conj-max-contain-w}
For all $m$ and $n$, the maximum number of $\W_m$-permutations contained by any permutation of size $n$ is
\[
    \sum_{\ell = 0}^{n-m} \binom{m-1}{\ell}.
\]
\end{conjecture}

\end{document}
