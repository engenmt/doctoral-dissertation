A combinatorial structure is said to be \emph{$m$-universal} if it contains all other structures of the same kind of size $m$. Rado first introduced the notion of universality for infinite graphs in 1964, and in the same year, Moon first studied the problem of seeking smallest universal graphs under the induced subgraph order. Since then, universality has been studied in a variety of other contexts, with connections to information theory, probability theory, and more.

This work exhibits the problem of seeking smallest universal structures for words, integer partitions, compositions, graphs, and permutations. For partitions and compositions, we show that the smallest $m$-universal structures have size $\oTheta{m \log m}$. In the context of graphs, we present the state of the art, culminating in Alon's 2017 proof that the smallest $m$-universal graphs have $2^{(m-1)/2}$ vertices asymptotically. The problem for permutations was first studied in 1999 by Arratia, who observed that the smallest $m$-universal permutations have size $\oTheta{m^2}$, but researchers hold conflicting conjectures about their precise smallest size. We present the best-known lower and upper bounds on this quantity, including a recent improvement on the lower bound by Chroman, Kwan, and Singhal as well as a recent improvement on the upper bound in joint work with Vatter.

In each of these contexts, we also present results old and new for structures that are universal for proper classes of structures. We show that for both partitions and compositions, any proper class admits universal structures of linear size, whereas for both graphs and permutations, the answers vary widely.